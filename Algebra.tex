\documentclass{article}
\usepackage[utf8]{inputenc}
\usepackage{graphicx}
\graphicspath{ {./images/} }
\usepackage{amsmath}
\usepackage{amssymb}
\usepackage{amsfonts}
\usepackage{amsthm}
\usepackage[sorting=none]{biblatex}
\usepackage{adjustbox}
\usepackage{array}
\usepackage{enumitem}
\usepackage{pdfpages}
\usepackage{setspace}
\usepackage{hyperref}
\usepackage{minted}
\usepackage{mathrsfs}
\newcolumntype{C}[1]{>{\centering\arraybackslash}m{#1}}
\usepackage[table]{xcolor}
\addbibresource{references.bib}
\newcommand{\Mod}[1]{\ (\mathrm{mod}\ #1)}
\newcommand*{\Perm}[2]{{}^{#1}\!P_{#2}}
\newcommand*{\Comb}[2]{{}^{#1}C_{#2}}
\DeclareMathOperator{\csch}{csch}
\DeclareMathOperator{\sech}{sech}
\DeclareMathOperator{\arsinh}{arsinh}
\DeclareMathOperator{\arcosh}{arcosh}
\DeclareMathOperator{\artanh}{artanh}
\DeclareMathOperator{\arcsch}{arcsch}
\DeclareMathOperator{\arsech}{arsech}
\DeclareMathOperator{\arcoth}{arcoth}
\DeclareMathOperator{\E}{E}
\DeclareMathOperator{\Var}{Var}
\DeclareMathOperator{\tr}{tr}
\DeclareMathOperator{\grad}{grad}
\DeclareMathOperator{\lcm}{lcm}
\DeclareMathOperator{\disc}{disc}
\DeclareMathOperator{\ord}{ord}
\DeclareMathOperator{\Cl}{Cl}
\DeclareMathOperator{\im}{im}
\DeclareMathOperator{\N}{N}
\DeclareMathOperator{\Aut}{Aut}
\DeclareMathOperator{\Inn}{Inn}
\DeclareMathOperator{\Syl}{Syl}
\DeclareMathOperator{\Hom}{Hom}
\DeclareMathOperator{\End}{End}
\DeclareMathOperator{\Sym}{Sym}
\DeclareMathOperator{\Alt}{Alt}
\DeclareMathOperator{\Tor}{Tor}
\DeclareMathOperator{\Ann}{Ann}
\DeclareMathOperator{\ch}{ch}
\DeclareMathOperator{\Gal}{Gal}
\DeclareMathOperator{\GL}{GL}
\DeclareMathOperator{\Cent}{Cent}
\DeclareMathOperator{\Rad}{Rad}
\newcommand{\characteristic}{\mathrel{\textrm{char}}}
\newcommand{\norm}[1]{\left\lVert #1 \right\rVert}
\theoremstyle{plain}
\newtheorem{theorem}{Theorem}[section]
\newtheorem{lemma}[theorem]{Lemma}
\newtheorem{prop}[theorem]{Proposition}
\newtheorem{corollary}[theorem]{Corollary}
\theoremstyle{definition}
\newtheorem{exmp}[theorem]{Example}
\newtheorem{defn}[theorem]{Definition}
\theoremstyle{remark}
\newtheorem*{remark}{Remark}
\def\lc{\left\lceil}   
\def\rc{\right\rceil}
\def\lf{\left\lfloor}   
\def\rf{\right\rfloor}

\title{Algebra}
\author{Wilkie Hoare}
\date{}

\begin{document}

\maketitle

\newpage
\tableofcontents

%%CONTENT STARTS HERE

\newpage
\section{Dummit and Foote: Abstract Algebra}
\subsection{Group Theory}
\subsubsection{Introduction to Groups}
\begin{defn}(Binary operation, associative, commutative).
    \begin{enumerate}
        \item A \textit{binary operation} $\star$ on a set $G$ is a function $\star:G\times G\rightarrow G$. For any $a,b\in G$ we shall write $a\star b$ for $\star(a,b)$.
        \item A binary operation $\star$ on a set $G$ is \textit{associative} if for all $a,b,c\in G$ we have $a\star(b\star c)=(a\star b)\star c$.
        \item If $\star$ is a binary operation on a set $G$ we say elements $a$ and $b$ of $G$ \textit{commute} if $a\star b=b\star a$. We say $\star$ (or $G$) is \textit{commutative} if for all $a,b\in G$, $a\star b=b\star a$.
    \end{enumerate}
\end{defn}
\begin{defn}(Group, identity, inverse, abelian group).
    \begin{enumerate}
        \item A \textit{group} is an ordered pair $(G,\star)$ where $G$ is a set and $\star$ is a binary operation on $G$ satisfying the following axioms:
        \begin{itemize}
            \item $(a\star b)\star c=a\star(b\star c)$, for all $a,b,c\in G$, i.e., $\star$ is \textit{associative}.
            \item There exists an element $e$ in $G$, called an \textit{identity} of $G$, such that for all $a\in G$ we have $a\star e=e\star a=a$.
            \item For each $a\in G$ there is an element $a^{-1}$ of $G$, called an \textit{inverse} of $a$, such that $a\star a^{-1}=a^{-1}\star a=e$.
        \end{itemize}
        \item The group $(G,\star)$ is called \textit{abelian} (or \textit{commutative}) if $a\star b=b\star a$ for all $a,b\in G$.
    \end{enumerate}
\end{defn}
\begin{prop}
    If $G$ is a group under the operation $\star$, then
    \begin{enumerate}
        \item The identity of $G$ is unique.
        \item For each $a\in G$, $a^{-1}$ is uniquely determined.
        \item $(a^{-1})^{-1}=a$ for all $a\in G$.
        \item $(a\star b)^{-1}=(b^{-1})\star(a^{-1})$.
        \item For any $a_1,a_2,\ldots,a_n\in G$ the value of $a_1\star a_2\star\cdots\star a_n$ is independent of how the expression is bracketed (this is called the \textit{generalized associative law}).
    \end{enumerate}
\end{prop}
\begin{prop}
    Let $G$ be a group and let $a,b\in G$. The equations $ax=b$ and $ya=b$ have unique solutions for $x,y\in G$. In particular, the left and right cancellation laws hold in $G$, i.e.,
    \begin{enumerate}
        \item If $au=av$, then $u=v$.
        \item If $ub=vb$, then $u=v$.
    \end{enumerate}
\end{prop}
\begin{defn}(Order).
    For $G$ a group and $x\in G$ define the \textit{order} of $x$ to be the smallest positive integer $n$ such that $x^n=1$, and denote this integer by $|x|$. In this case $x$ is said to be of order $n$. If no positive power of $x$ is the identity, the order of $x$ is defined to be infinity and $x$ is said to be of infinite order.
\end{defn}
\begin{defn}(Multiplication/group table).
    Let $G=\{g_1,g_2,\ldots,g_n\}$ be a finite group with $g_1=1$. The \textit{multiplication table} or \textit{group table} of $G$ is the $n\times n$ matrix whose $i,j$ entry is the group elements $g_ig_j$.
\end{defn}
\begin{defn}(Field).
    \begin{enumerate}
        \item A \textit{field} is a set $F$ together with two binary operations $+$ and $\cdot$ on $F$ such that $(F,+)$ is an abelian group (call its identity 0) and $(F-\{0\},\cdot)$ is also an abelian group, and the following \textit{distributive} law holds: $a\cdot(b+c)=(a\cdot b)+(a\cdot c)$ for all $a,b,c\in F$.
        \item For any field $F$ let $F^\times=F-\{0\}$.
    \end{enumerate}
\end{defn}
\begin{defn}(Homomorphism).
    Let $(G,\star)$ and $(H,\diamond)$ be groups. A map $\varphi:G\rightarrow H$ such that $\varphi(x\star y)=\varphi(x)\diamond\varphi(y)$ for all $x,y\in G$ is called a \textit{homomorphism}.
\end{defn}
\begin{defn}(Isomorphism).
    The map $\varphi:G\rightarrow H$ is called an \textit{isomorphism} and $G$ and $H$ are said to be \textit{isomorphic} or of the same \textit{isomorphism type}, written $G\cong H$, if
    \begin{enumerate}
        \item $\varphi$ is a homomorphism.
        \item $\varphi$ is a bijection.
    \end{enumerate}
\end{defn}
\begin{defn}(Group action).
    A \textit{group action} of a group $G$ on a set $A$ is a map from $G\times A$ to $A$ (written as $g\cdot a$ for all $g\in G$ and $a\in A$) satisfying the following properties:
    \begin{enumerate}
        \item $g_1\cdot(g_2\cdot a)=(g_1g_2)\cdot a$ for all $g_1,g_2\in G,a\in A$.
        \item $1\cdot a=a$ for all $a\in A$.
    \end{enumerate}
\end{defn}
\subsubsection{Subgroups}
\begin{defn}(Subgroup).
    Let $G$ be a group. The subset $H$ of $G$ is a \textit{subgroup} of $G$ if $H$ is nonempty and $H$ is closed under products and inverses (i.e., $x,y\in H$ implies $x^{-1}\in H$ and $xy\in H$. If $H$ is a subgroup of $G$ we shall write $H\leq G$.
\end{defn}
\begin{prop}(The Subgroup Criterion).
    A subset $H$ of a group $G$ is a subgroup if and only if
    \begin{enumerate}
        \item $H\neq\emptyset$.
        \item For all $x,y\in H$, $xy^{-1}\in H$.
    \end{enumerate}
\end{prop}
\begin{defn}(Centralizer).
    Define $C_G(A)=\{g\in G:gag^{-1}=a\;\forall a\in A\}$. This subset of $G$ is called the \textit{centralizer} of $A$ in $G$. Since $gag^{-1}=a$ if and only if $ga=ag$, $C_G(A)$ is the set of elements of $G$ which commute with every element of $A$.
\end{defn}
\begin{defn}(Center).
    Define $Z(G)=\{g\in G:gx=xg\;\forall x\in G\}$, the set of elements commuting with all the elements of $G$. This subset of $G$ is called the \textit{center} of $G$.
\end{defn}
\begin{defn}(Normalizer).
    Define $gAg^{-1}=\{gag^{-1}:a\in A\}$. Define the \textit{normalizer} of $A$ in $G$ to be the set $N_G(A)=\{g\in G:gAg^{-1}=A\}$.
\end{defn}
\begin{defn}(Cyclic group).
    A group $H$ is \textit{cyclic} if $H$ can be generated by a single element, i.e., there is some element $x\in H$ such that $H=\{x^n:n\in\mathbb{Z}\}$ (where as usual the operation is multiplication).
\end{defn}
\begin{prop}
    If $H=\langle x\rangle$, then $|H|=|x|$ (where if one side of this equality is infinite, so is the other). More specifically
    \begin{enumerate}
        \item If $|H|=n<\infty$, then $x^n=1$ and $1,x,x^2,\ldots,x^{n-1}$ are all the distinct elements of $H$.
        \item If $|H|=\infty$, then $x^n\neq1$ for all $n\neq0$ and $x^a\neq x^b$ for all $a\neq b$ in $\mathbb{Z}$.
    \end{enumerate}
\end{prop}
\begin{prop}
    Let $G$ be an arbitrary group, $x\in G$ and let $m,n\in\mathbb{Z}$. If $x^n=1$ and $x^m=1$, then $x^d=1$, where $d=(m,n)$. In particular, if $x^m=1$ for some $m\in\mathbb{Z}$, then $|x|$ divides $m$.
\end{prop}
\begin{theorem}
    Any two cyclic groups of the same order are isomorphic. More specifically,
    \begin{enumerate}
        \item If $n\in\mathbb{Z}^+$ and $\langle x\rangle$ and $\langle y\rangle$ are both cyclic groups of order $n$, then the map $\varphi:\langle x\rangle\rightarrow\langle y\rangle,x^k\mapsto y^k$ is well defined and is an isomorphism.
        \item If $\langle x\rangle$ is an infinite cyclic group, the map $\varphi:\mathbb{Z}\rightarrow\langle x\rangle,k\mapsto x^k$ is well defined and is an isomorphism.
    \end{enumerate}
\end{theorem}
\begin{prop}
    Let $G$ be a group, let $x\in G$ and let $a\in\mathbb{Z}-\{0\}$.
    \begin{enumerate}
        \item If $|x|=\infty$, then $|x^a|=\infty$.
        \item If $|x|=n<\infty$, then $|x^a|=n/(n,a)$.
        \item In particular, if $|x|=n<\infty$ and $a$ is a positive integer dividing $n$, then $|x^a|=n/a$.
    \end{enumerate}
\end{prop}
\begin{prop}
    Let $H=\langle x\rangle$.
    \begin{enumerate}
        \item Assume $|x|=\infty$. Then $H=\langle x^a\rangle$ if and only if $a=\pm1$.
        \item Assume $|x|=n<\infty$. Then $H=\langle x^a\rangle$ if and only if $(a,n)=1$. In particular, the number of generators of $H$ is $\varphi(n)$ (where $\varphi$ is Euler's $\varphi$-function).
    \end{enumerate}
\end{prop}
\begin{theorem}
    Let $H=\langle x\rangle$ be a cyclic group.
    \begin{enumerate}
        \item Every subgroup of $H$ is cyclic. More precisely, if $K\leq H$, then either $K=\{1\}$ or $K=\langle x^d\rangle$, where $d$ is the smallest positive integer such that $x^d\in K$.
        \item If $|H|=\infty$, then for any distinct nonnegative integers $a$ and $b$, $\langle x^a\rangle\neq\langle x^b\rangle$. Furthermore, for every integer $m$, $\langle x^m\rangle=\langle x^{|m|}\rangle$, where $|m|$ denotes the absolute value of $m$, so that the nontrivial subgroups of $H$ correspond bijectively with the natural numbers.
        \item If $|H|=n<\infty$, then for each positive integer $a$ dividing $n$ there is a unique subgroup of $H$ of order $a$. This subgroup is the cyclic group $\langle x^d\rangle$, where $d=n/a$. Furthermore, for every integer $m$, $\langle x^m\rangle=\langle x^{(n,m)}\rangle$, so that the subgroups of $H$ correspond bijectively with the positive divisors of $n$.
    \end{enumerate}
\end{theorem}
\begin{prop}
    If $\mathcal{A}$ is any nonempty collection of subgroups of $G$, then the intersection of all members of $\mathcal{A}$ is also a subgroup of $G$.
\end{prop}
\begin{defn}(Subgroup generated by a subset).
    If $A$ is any subset of the group $G$ define \[\langle A\rangle=\bigcap_{\substack{A\subseteq H\\H\leq G}}H\] This is called the \textit{subgroup of $G$ generated by $A$}.
\end{defn}
\begin{prop}
    $\overline{A}=\langle A\rangle$.
\end{prop}
\subsubsection{Quotient Groups and Homomorphisms}
\begin{defn}(Kernel).
    If $\varphi$ is a homomorphism $\varphi:G\rightarrow H$, the kernel of $\varphi$ is the set $\{g\in G:\varphi(g)=1\}$ and will be denoted by $\ker{\varphi}$ (here 1 is the identity of $H$).
\end{defn}
\begin{prop}
    Let $G$ and $H$ be groups and let $\varphi:G\rightarrow H$ be a homomorphism.
    \begin{enumerate}
        \item $\varphi(1_G)=1_H$, where $1_G$ and $1_H$ are the identities of $G$ and $H$, respectively.
        \item $\varphi(g^{-1})=\varphi(g)^{-1}$ for all $g\in G$.
        \item $\varphi(g^n)=\varphi(g)^n$ for all $n\in\mathbb{Z}$.
        \item $\ker{\varphi}$ is a subgroup of $G$.
        \item $\im(\varphi)$, the image of $G$ under $\varphi$, is a subgroup of $H$.
    \end{enumerate}
\end{prop}
\begin{defn}(Quotient/factor group).
    Let $\varphi:G\rightarrow H$ be a homomorphism with kernel $K$. The \textit{quotient group} or \textit{factor group}, $G/K$ (read \textit{$G$ modulo $K$} or simply \textit{$G$ mod $K$}), is the group whose elements are the fibers of $\varphi$ with group operation defined above: namely if $X$ is the fiber above $a$ and $Y$ is the fiber above $b$ then the product of $X$ with $Y$ is defined to be the fiber above the product $ab$.
\end{defn}
\begin{prop}
    Let $\varphi:G\rightarrow H$ be a homomorphism of groups with kernel $K$. Let $x\in G/K$ be the fiber above, i.e., $X=\varphi^{-1}(a)$. Then
    \begin{enumerate}
        \item For any $u\in X$, $X=\{uk:k\in K\}$.
        \item For any $u\in X$, $X=\{ku:k\in K\}$.
    \end{enumerate}
\end{prop}
\begin{defn}(Left/right coset, representative).
    For any $N\leq G$ and any $g\in G$ let $gN=\{gn:n\in N\}$ and $Ng=\{ng:n\in N\}$ called respectively a \textit{left coset} and a \textit{right coset} of $N$ in $G$. Any element of a coset is called a \textit{representative} for the coset.
\end{defn}
\begin{theorem}
    Let $G$ be a group and let $K$ be the kernel of some homomorphism from $G$ to another group. Then the set whose elements are the left (or right) cosets of $K$ in $G$ with operation defined by $uK\circ vK=(uv)K$ forms a group, $G/K$. In particular, this operation is well defined in the sense that if $u_1$ is any element in $uK$ and $v_1$ is any element in $vK$, then $u_1v_1\in uvK$, i.e., $u_1v_1K=uvK$ so that the multiplication does not depend on the choice of representatives for the cosets.
\end{theorem}
\begin{prop}
    Let $N$ be any subgroup of the group $G$. The set of left cosets of $N$ in $G$ form a partition of $G$. Furthermore, for all $u,v\in G$, $uN=vN$ if and only if $v^{-1}u\in N$ and in particular, $uN=vN$ if and only if $u$ and $v$ are representatives of the same coset.
\end{prop}
\begin{prop}
    Let $G$ be a group and let $N$ be a subgroup of $G$.
    \begin{enumerate}
        \item The operation on the set of left cosets of $N$ in $G$ described by $uN\cdot vN=(uv)N$ is well defined if and only if $gng^{-1}\in N$ for all $g\in G$ and all $n\in N$.
        \item If the above operation is well defined, then it makes the set of left cosets of $N$ in $G$ into a group. In particular the identity of this group is the coset $1N$ and the inverse of $gN$ is the coset $g^{-1}N$ i.e., $(gN)^{-1}=g^{-1}N$.
    \end{enumerate}
\end{prop}
\begin{defn}(Conjugate, normalize, normal subgroup).
    The element $gng^{-1}$ is called the \textit{conjugate} of $n\in G$ by $g$. The set $gNg^{-1}=\{gng^{-1}:n\in N\}$ is called the \textit{conjugate} of $N$ by $g$. The element $g$ is said to \textit{normalize} $N$ if $gNg^{-1}=N$. A subgroup $N$ of a group $G$ is called \textit{normal} if every element of $G$ normalizes $N$, i.e., if $gNg^{-1}=N$ for all $g\in G$. If $N$ is a normal subgroup of $G$ we shall write $N\trianglelefteq G$.
\end{defn}
\begin{theorem}
    Let $N$ be a subgroup of the group $G$. The following are equivalent:
    \begin{enumerate}
        \item $N\trianglelefteq G$.
        \item $N_G(N)=G$ (recall $N_G(N)$ is the normalizer in $G$ of $N$).
        \item $gN=Ng$ for all $g\in G$.
        \item The operation on left cosets of $N$ in $G$ described in Proposition 1.33 makes the set of left cosets into a group.
        \item $gNg^{-1}\subseteq N$ for all $g\in G$.
    \end{enumerate}
\end{theorem}
\begin{prop}
    A subgroup $N$ of the group $G$ is normal if and only if it is the kernel of some homomorphism.
\end{prop}
\begin{defn}(Natural projection, complete preimage).
    Let $N\trianglelefteq G$. The homomorphism $\pi:G\rightarrow G/N$ defined by $\pi(g)=gN$ is called the \textit{natural projection (homomorphism)} of $G$ onto $G/N$. If $\overline{H}\leq G/N$ is a subgroup of $G/N$, the \textit{complete preimage} of $\overline{H}$ in $G$ is the preimage of $\overline{H}$ under the natural projection homomorphism.
\end{defn}
\begin{theorem}(Lagrange's Theorem).
    If $G$ is a finite group and $H$ is a subgroup of $G$, then the order of $H$ divides the order of $G$ (i.e., $|H|\mid|G|$) and the number of left cosets of $H$ in $G$ equals $|G|/|H|$.
\end{theorem}
\begin{defn}(Index).
    If $G$ is a group (possibly infinite) and $H\leq G$, the number of left cosets of $H$ in $G$ is called the \textit{index} of $H$ in $G$ and is denoted by $|G:H|$.
\end{defn}
\begin{corollary}
    If $G$ is a finite group and $x\in G$, then the order of $x$ divides the order of $G$. In particular $x^{|G|}=1$ for all $x\in G$.
\end{corollary}
\begin{corollary}
    If $G$ is a group of prime order $p$, then $G$ is cyclic, hence $G\cong\mathbb{Z}_p$.
\end{corollary}
\begin{theorem}(Cauchy's Theorem).
    If $G$ is a finite group and $p$ is a prime dividing $|G|$, then $G$ has an element of order $p$.
\end{theorem}
\begin{theorem}(Sylow).
    If $G$ is a finite group of order $p^{\alpha}m$, where $p$ is a prime and $p$ does not divide $m$, then $G$ has a subgroup of order $p^{\alpha}$.
\end{theorem}
\begin{defn}
    Let $H$ and $K$ be subgroups of a group and define $HK=\{hk:h\in H,k\in K\}$.
\end{defn}
\begin{prop}
    If $H$ and $K$ are finite subgroups of a group then $|HK|=|H||K|/|H\cap K|$.
\end{prop}
\begin{prop}
    If $H$ and $K$ are subgroups of a group, $HK$ is a subgroup if and only if $HK=KH$.
\end{prop}
\begin{corollary}
    If $H$ and $K$ are subgroups of $G$ and $H\leq N_G(K)$, then $HK$ is a subgroup of $G$. In particular, if $K\trianglelefteq G$ then $HK\leq G$ for any $H\leq G$.
\end{corollary}
\begin{defn}(Normalizes, centralizes).
    If $A$ is any subset of $N_G(K)$ (or $C_G(K)$), we shall say $A$ \textit{normalizes} $K$ (\textit{centralizes} $K$, respectively).
\end{defn}
\begin{theorem}(The First Isomorphism Theorem for Groups).
    If $\varphi:G\rightarrow H$ is a homomorphism of groups, then $\ker{\varphi}\trianglelefteq G$ and $G/\ker{\varphi}\cong\varphi(G)$.
\end{theorem}
\begin{corollary}
    Let $\varphi:G\rightarrow H$ be a homomorphisms of groups.
    \begin{enumerate}
        \item $\varphi$ is injective if and only if $\ker{\varphi}=1$.
        \item $|G:\ker{\varphi}|=|\varphi(G)|$.
    \end{enumerate}
\end{corollary}
\begin{theorem}(The Second or Diamond Isomorphism Theorem for Groups).
    Let $G$ be a group, let $A$ and $B$ be subgroups of $G$ and assume $A\leq N_G(B)$. Then $AB$ is a subgroup of $G$, $B\trianglelefteq AB$, $A\cap B\trianglelefteq A$ and $AB/B\cong A/A\cap B$.
\end{theorem}
\begin{theorem}(The Third Isomorphism Theorem for Groups).
    Let $G$ be a group and let $H$ and $K$ be normal subgroups of $G$ with $H\leq K$. Then $K/H\trianglelefteq G/H$ and $(G/H)/(K/H)\cong G/K$.
\end{theorem}
\begin{theorem}(The Fourth or Lattice Isomorphism Theorem for Groups).
    Let $G$ be a group and let $N$ be a normal subgroup of $G$. Then there is a bijection from the set of subgroups $A$ of $G$ which contain $N$ onto the set of subgroups $\overline{A}=A/N$ of $G/N$. In particular, every subgroup of $\overline{G}$ is of the form $A/N$ for some subgroup $A$ of $G$ containing $N$ (namely, its preimage in $G$ under the natural projection homomorphism from $G$ to $G/N$). This bijection has the following properties: for all $A,B\leq G$ with $N\leq A$ and $N\leq B$,
    \begin{enumerate}
        \item $A\leq B$ if and only if $\overline{A}\leq\overline{B}$.
        \item If $A\leq B$, then $|B:A|=|\overline{B}:\overline{A}|$.
        \item $\overline{\langle A,B\rangle}=\langle\overline{A},\overline{B}\rangle$.
        \item $\overline{A\cap B}=\overline{A}\cap\overline{B}$.
        \item $A\trianglelefteq G$ if and only if $\overline{A}\trianglelefteq\overline{G}$.
    \end{enumerate}
\end{theorem}
\begin{prop}
    If $G$ is a finite abelian group and $p$ is a prime dividing $|G|$, then $G$ contains an element of order $p$.
\end{prop}
\begin{defn}(Simple group).
    A (finite or infinite) group $G$ is called \textit{simple} if $|G|>1$ and the only normal subgroups of $G$ are 1 and $G$.
\end{defn}
\begin{defn}(Composition series, composition factors).
    In a group $G$ a sequence of subgroups $1=N_0\leq N_1\leq N_2\leq\cdots\leq N_{k-1}\leq N_k=G$ is called a \textit{composition series} if $N_i\trianglelefteq N_{i+1}$ and $N_{i+1}/N_i$ a simple group, $0\leq i\leq k-1$. If the above sequence is a composition series, the quotient groups $N_{i+1}/N_i$ are called \textit{composition factors} of $G$.
\end{defn}
\begin{theorem}(Jordan-Hölder).
    Let $G$ be a finite group with $G\neq1$. Then
    \begin{enumerate}
        \item $G$ has a composition series.
        \item The composition factors in a composition series are unique, namely, if $1=N_0\leq N_1\leq\cdots\leq N_r=G$ and $1=M_0\leq M_1\leq\cdots M_s=G$ are two composition series for $G$, then $r=s$ and there is some permutation, $\pi$, of $\{1,2,\ldots,r\}$ such that $M_{\pi(i)}/M_{\pi(i)-1}\cong N_i/N_{i-1},1\leq i\leq r$.
    \end{enumerate}
\end{theorem}
\begin{theorem}
    There is a list consisting of 18 (infinite) families of simple groups and 26 simple groups not belonging to these families (the \textit{sporadic} simple groups) such that every finite simple group is isomorphic to one of the groups in this list.
\end{theorem}
\begin{theorem}(Feit-Thompson).
    If $G$ is a simple group of odd order, then $G\cong\mathbb{Z}_p$ for some prime $p$.
\end{theorem}
\begin{defn}(Solvable).
    A group $G$ is \textit{solvable} if there is a chain of subgroups $1=G_0\trianglelefteq G_1\trianglelefteq G_2\trianglelefteq\ldots\trianglelefteq G_s=G$ such that $G_{i+1}/G_i$ is abelian for $i=0,1,\ldots,s-1$.
\end{defn}
\begin{theorem}
    The finite group $G$ is solvable if and only if for every divisor $n$ of $|G|$ such that $(n,|G|/n)=1$, $G$ has a subgroup of order $n$.
\end{theorem}
\begin{defn}(Transposition).
    A 2-cycle is called a \textit{transposition}.
\end{defn}
\begin{defn}(Sign, even/odd permutation).
    \begin{enumerate}
        \item $\epsilon(\sigma)$ is called the \textit{sign} of $\sigma$.
        \item $\sigma$ is called an \textit{even permutation} if $\epsilon(\sigma)=1$ and an \textit{odd permutation} if $\epsilon(\sigma)=-1$.
    \end{enumerate}
\end{defn}
\begin{prop}
    The map $\epsilon:S_n\rightarrow\{\pm1\}$ is a homomorphism (where $\{\pm1\}$ is a multiplicative version of the cyclic group of order 2).
\end{prop}
\begin{prop}
    Transpositions are all odd permutations and $\epsilon$ is a surjective homomorphism.
\end{prop}
\begin{defn}(Alternating group).
    The \textit{alternating group of degree $n$}, denoted by $A_n$, is the kernel of the homomorphism $\epsilon$ (i.e., the set of even permutations).
\end{defn}
\begin{prop}
    The permutation $\sigma$ is odd if and only if the number of cycles of even length in its cycle decomposition is odd.
\end{prop}
\subsubsection{Group Actions}
\begin{defn}(Kernel, stabilizer, faithful).
    \begin{enumerate}
        \item The \textit{kernel} of the action is the set of elements of $G$ that act trivially on every element of $A$: $\{g\in G:g\cdot a=a\;\forall a\in A\}$.
        \item For each $a\in A$ the \textit{stabilizer} of $a$ in $G$ is the set of elements of $G$ that fix the element $a$: $\{g\in G:g\cdot a=a\}$ and is denoted by $G_a$.
        \item An action is \textit{faithful} if its kernel is the identity.
    \end{enumerate}
\end{defn}
\begin{prop}
    For any group $G$ and any nonempty set $A$ there is a bijection between the actions of $G$ on $A$ and the homomorphisms of $G$ into $S_A$.
\end{prop}
\begin{defn}(Permutation representation, affords/induces).
    If $G$ is a group, a \textit{permutation representation} of $G$ is any homomorphism of $G$ into the symmetric group $S_A$ for some nonempty set $A$. We shall say a given action of $G$ on $A$ \textit{affords} or \textit{induces} the associated permutation representation of $G$.
\end{defn}
\begin{prop}
    Let $G$ be a group acting on the nonempty set $A$. The relation on $A$ defined by $a\sim b$ if and only if $a=g\cdot b$ for some $g\in G$ is an equivalence relation. For each $a\in A$, the number of elements in the equivalence class containing $a$ is $|G:G_a|$, the index of the stabilizer of $a$.
\end{prop}
\begin{defn}(Orbit, transitive).
    Let $G$ be a group acting on the nonempty set $A$.
    \begin{enumerate}
        \item The equivalence class $\{g\cdot a:g\in G\}$ is called the \textit{orbit} of $G$ containing $a$.
        \item The action of $G$ on $A$ is called \textit{transitive} if there is only one orbit, i.e., given any two elements $a,b\in A$ there is some $g\in G$ such that $a=g\cdot b$.
    \end{enumerate}
\end{defn}
\begin{theorem}
    Let $G$ be a group, let $H$ be a subgroup of $G$ and let $G$ act by left multiplication on the set $A$ of left cosets of $H$ in $G$. Let $\pi_H$ be the associated permutation representation afforded by this action. Then
    \begin{enumerate}
        \item $G$ acts transitively on $A$.
        \item The stabilizer in $G$ of the point $1H\in A$ is the subgroup $H$.
        \item The kernel of the action (i.e., the kernel of $\pi_H$) is $\cap_{x\in G}xHx^{-1}$, and $\ker{\pi_H}$ is the largest normal subgroup of $G$ contained in $H$.
    \end{enumerate}
\end{theorem}
\begin{corollary}(Cayley's Theorem).
    Every group is isomorphic to a subgroup of some symmetric group. If $G$ is a group of order $n$, then $G$ is isomorphic to a subgroup of $S_n$.
\end{corollary}
\begin{corollary}
    If $G$ is a finite group of order $n$ and $p$ is the smallest prime dividing $|G|$, then any subgroup of index $p$ is normal.
\end{corollary}
\begin{defn}(Conjugate elements, conjugacy classes).
    Two elements $a$ and $b$ of $G$ are said to be \textit{conjugate in} $G$ if there is some $g\in G$ such that $b=gag^{-1}$ (i.e., if and only if they are in the same orbit of $G$ acting on itself by conjugation). The orbits of $G$ acting on itself by conjugation are called the \textit{conjugacy classes} of $G$.
\end{defn}
\begin{defn}(Conjugate subsets).
    Two subsets $S$ and $T$ of $G$ are said to be \textit{conjugate in} $G$ if there is some $g\in G$ such that $T=gSg^{-1}$ (i.e., if and only if they are in the same orbit of $G$ acting on its subsets by conjugation).
\end{defn}
\begin{prop}
    The number of conjugates of a subset $S$ in a group $G$ is the index of the normalizer of $S$, $|G:N_G(S)|$. In particular, the number of conjugates of an element $s$ of $G$ is the index of the centralizer of $s$, $|G:C_G(s)|$.
\end{prop}
\begin{theorem}(The Class Equation).
    Let $G$ be a finite group and let $g_1,g_2,\ldots,g_r$ be representatives of the distinct conjugacy classes of $G$ not contained in the center $Z(G)$ of $G$. Then \[|G|=|Z(G)|+\sum_{i=1}^r|G:C_G(g_i)|\]
\end{theorem}
\begin{theorem}
    If $p$ is a prime and $P$ is a group of prime power order $p^\alpha$ for some $\alpha\geq1$, then $P$ has a nontrivial center: $Z(P)\neq1$.
\end{theorem}
\begin{corollary}
    If $|P|=p^2$ for some prime $p$, then $P$ is abelian. More precisely, $P$ is isomorphic to either $\mathbb{Z}_{p^2}$ or $\mathbb{Z}_p\times\mathbb{Z}_p$.
\end{corollary}
\begin{prop}
    Let $\sigma,\tau$ be elements of the symmetric group $S_n$ and suppose $\sigma$ has cycle decomposition $(a_1a_2\ldots a_{k_1})(b_1b_2\ldots b_{k_2})\ldots$. Then $\tau\sigma\tau^{-1}$ has cycle decomposition $(\tau(a_1)\tau(a_2)\ldots\tau(a_{k_1}))(\tau(b_1)\tau(b_2)\ldots\tau(b_{k_2}))\ldots$, that is, $\tau\sigma\tau^{-1}$ is obtained from $\sigma$ by replacing each entry $i$ in the cycle decomposition for $\sigma$ by the entry $\tau(i)$.
\end{prop}
\begin{defn}(Cycle type, partition).
    \begin{enumerate}
        \item If $\sigma\in S_n$ is the product of disjoint cycles of lengths $n_1,n_2,\ldots,n_r$ with $n_1\leq n_2\leq\cdots\leq n_r$ (including its 1-cycles) then the integers $n_1,n_2,\ldots,n_r$ are called the \textit{cycle type} of $\sigma$.
        \item If $n\in\mathbb{Z}^+$, a \textit{partition} of $n$ is any nondecreasing sequence of positive integers whose sum is $n$.
    \end{enumerate}
\end{defn}
\begin{prop}
    Two elements of $S_n$ are conjugate in $S_n$ if and only if they have the same cycle type. The number of conjugacy classes of $S_n$ equals the number of partitions of $n$.
\end{prop}
\begin{defn}(Automorphism of a group).
    Let $G$ be a group. An isomorphism from $G$ onto itself is called an \textit{automorphism} of $G$. The set of all automorphisms of $G$ is denoted by $\Aut(G)$.
\end{defn}
\begin{prop}
    Let $H$ be a normal subgroup of the group $G$. Then $G$ acts by conjugation on $H$ as automorphisms of $H$. More specifically, the action of $G$ on $H$ by conjugation is defined for each $g\in G$ by $h\mapsto ghg^{-1}$ for each $h\in H$. For each $g\in G$, conjugation by $g$ is an automorphism of $H$. The permutation representation afforded by this action is a homomorphism of $G$ into $\Aut(H)$ with kernel $C_G(H)$. In particular, $G/C_G(H)$ is isomorphic to a subgroup of $\Aut(H)$.
\end{prop}
\begin{corollary}
    If $K$ is any subgroup of the group $G$ and $g\in G$, then $K\cong gKg^{-1}$. Conjugate elements and conjugate subgroups have the same order.
\end{corollary}
\begin{corollary}
    For any subgroup $H$ of a group $G$, the quotient group $N_G(H)/C_G(H)$ is isomorphic to a subgroup of $\Aut(H)$. In particular, $G/Z(G)$ is isomorphic to a subgroup of $\Aut(G)$.
\end{corollary}
\begin{defn}(Inner automorphism).
    Let $G$ be a group and let $g\in G$. Conjugation by $g$ is called an \textit{inner automorphism} of $G$ and the subgroup of $\Aut(G)$ consisting of all inner automorphisms is denoted by $\Inn(G)$.
\end{defn}
\begin{defn}(Characteristic subgroup).
    A subgroup $H$ of a group $G$ is called \textit{characteristic} in $G$, denoted $H\characteristic G$, if every automorphism of $G$ maps $H$ to itself, i.e., $\sigma(H)=H$ for all $\sigma\in\Aut(G)$.
\end{defn}
\begin{prop}
    The automorphism group of the cyclic group of order $n$ is isomorphic to $(\mathbb{Z}/n\mathbb{Z})^\times$, an abelian group of order $\varphi(n)$ (where $\varphi$ is Euler's function).
\end{prop}
\begin{prop}
    \begin{enumerate}
        \item If $p$ is an odd prime and $n\in\mathbb{Z}^+$, then the automorphism group of the cyclic group of order $p$ is cyclic of order $p-1$. More generally, the automorphism group of the cyclic group of order $p^n$ is cyclic of order $p^{n-1}(p-1)$.
        \item For all $n\geq3$ the automorphism group of the cyclic group of order $2^n$ is isomorphic to $\mathbb{Z}_2\times\mathbb{Z}_{2^{n-2}}$, and in particular is not cyclic but has a cyclic subgroup of index 2.
        \item Let $p$ be a prime and let $V$ be an abelian group (written additively) with the property that $pv=0$ for all $v\in V$. If $|V|=p^n$, then $V$ is an $n$-dimensional vector space over the field $\mathbb{F}_p=\mathbb{Z}/p\mathbb{Z}$. The automorphisms of $V$ are precisely the nonsingular linear transformations from $V$ to itself, that is $\Aut(V)\cong GL(V)\cong GL_n(\mathbb{F}_p)$.
        \item For all $n\neq6$ we have $\Aut(S_n)=\Inn(S_n)\cong S_n$. For $n=6$ we have $|\Aut(S_6):\Inn(S_6)|=2$.
        \item $\Aut(D_8)\cong D_8$ and $\Aut(Q_8)\cong S_4$.
    \end{enumerate}
\end{prop}
\begin{defn}($p$-(sub)group, Sylow $p$-subgroup).
    Let $G$ be a group and let $p$ be a prime.
    \begin{enumerate}
        \item A group of order $p^{\alpha}$ for some $\alpha\geq1$ is called a \textit{$p$-group}. Subgroups of $G$ which are $p$-groups are called \textit{$p$-subgroups}.
        \item If $G$ is a group of order $p^{\alpha}m$, where $p\nmid m$, then a subgroup of order $p^{\alpha}$ is called a \textit{Sylow $p$-subgroup} of $G$.
        \item The set of Sylow $p$-subgroups of $G$ will be denoted by $\Syl_p(G)$ and the number of Sylow $p$-subgroups of $G$ will be denoted by $n_p(G)$ (or just $n_p$ when $G$ is clear from the context).
    \end{enumerate}
\end{defn}
\begin{theorem}(Sylow's Theorem).
    Let $G$ be a group of order $p^{\alpha}m$, where $p$ is a prime not dividing $m$.
    \begin{enumerate}
        \item Sylow $p$-subgroups of $G$ exist, i.e., $\Syl_p(G)\neq\emptyset$.
        \item If $P$ is a Sylow $p$-subgroup of $G$ and $Q$ is any $p$-subgroup of $G$, then there exists $g\in G$ such that $Q\leq gPg^{-1}$, i.e., $Q$ is contained in some conjugate of $P$. In particular, any two Sylow $p$-subgroups of $G$ are conjugate in $G$.
        \item The number of Sylow $p$-subgroups of $G$ is of the form $1+kp$, i.e., $n_p\equiv1\mod p$. Further, $n_p$ is the index in $G$ of the normalizer $N_G(P)$ for any Sylow $p$-subgroup $P$, hence $n_p$ divides $m$.
    \end{enumerate}
\end{theorem}
\begin{lemma}
    Let $P\in\Syl_p(G)$. If $Q$ is any $p$-subgroup of $G$, then $Q\cap N_{G}(P)=Q\cap P$.
\end{lemma}
\begin{corollary}
    Let $P$ be a Sylow $p$-subgroup of $G$. Then the following are equivalent:
    \begin{enumerate}
        \item $P$ is the unique Sylow $p$-subgroup of $G$, i.e., $n_p=1$.
        \item $P$ is normal in $G$.
        \item $P$ is characteristic in $G$.
        \item All subgroups generated by elements of $p$-power order are $p$-groups, i.e., if $X$ is any subset of $G$ such that $|x|$ is a power of $p$ for all $x\in X$, then $\langle X\rangle$ is a $p$-group.
    \end{enumerate}
\end{corollary}
\begin{prop}
    If $|G|=60$ and $G$ has more than one Sylow 5-subgroup, then $G$ is simple.
\end{prop}
\begin{corollary}
    $A_5$ is simple.
\end{corollary}
\begin{prop}
    If $G$ is a simple group of order 60, then $G\cong A_5$.
\end{prop}
\begin{theorem}
    $A_n$ is simple for all $n\geq5$.
\end{theorem}
\subsubsection{Direct and Semidirect Products and Abelian Groups}
\begin{defn}(Direct product of groups).
    \begin{enumerate}
        \item The \textit{direct product} $G_1\times G_2\times\cdots\times G_n$ of the groups $G_1,G_2,\ldots,G_n$ with operations $\star_1,\star_2,\ldots,\star_n$, respectively, is the set of $n$-tuples $(g_1,g_2,\ldots,g_n)$ where $g_i\in G_i$ with operation defined componentwise: $(g_1,g_2,\ldots,g_n)\star(h_1,h_2,\ldots,h_n)=(g_1\star_1h_1,g_2\star_2h_2,\ldots,g_n\star_nh_n)$.
        \item Similarly, the \textit{direct product} $G_1\times G_2\times\cdots$ of the groups $G_1,G_2,\ldots$ with operations $\star_1,\star_2,\ldots$, respectively, is the set of sequences $(g_1,g_2,\ldots)$ where $g_i\in G_i$ with operation defined componentwise: $(g_1,g_2,\ldots)\star(h_1,h_2,\ldots)=(g_1\star_1h_1,g_2\star_2h_2,\ldots)$.
    \end{enumerate}
\end{defn}
\begin{prop}
    If $G_1,\ldots,G_n$ are groups, their direct product is a group of order $|G_1||G_2|\cdots|G_n|$ (if any $G_i$ is infinite, so is the direct product).
\end{prop}
\begin{prop}
    Let $G_1,G_2,\ldots,G_n$ be groups and let $G=G_1\times\cdots\times G_n$ be their direct product.
    \begin{enumerate}
        \item For each fixed $i$ the set of elements of $G$ which have the identity of $G_j$ in the $j$th position for all $j\neq i$ and arbitrary elements of $G_i$ in position $i$ is a subgroup of $G$ isomorphic to $G_i$: $G_i\cong\{(1,1,\ldots,1,g_i,1,\ldots,1):g_i\in G_i\}$ (here $g_i$ appears in the $i$th position). If we identify $G_i$ with this subgroup, then $G_i\trianglelefteq G$ and $G/G_i\cong G_1\times\cdots\times G_{i-1}\times G_{i+1}\times\cdots\times G_n$.
        \item For each fixed $i$ define $\pi_i:G\rightarrow G_i$ by $\pi_i((g_1,g_2,\ldots,g_n))=g_i$. Then $\pi_i$ is a surjective homomorphism with $\ker{\pi_i}=\{(g_1,\ldots,g_{i-1},1,g_{i+1},\ldots,g_n):g_j\in G_j\;\forall j\neq i\}\cong G_1\times\cdots\times G_{i-1}\times G_{i+1}\times\cdots\times G_n$ (here the 1 appears in position $i$).
        \item Under the identifications in part (1), if $x\in G_i$ and $y\in G_j$ for some $i\neq j$, then $xy=yx$.
    \end{enumerate}
\end{prop}
\begin{defn}(Finitely generated group, free abelian group of rank).
    \begin{enumerate}
        \item A group $G$ is \textit{finitely generated} if there is a finite subset $A$ of $G$ such that $G=\langle A\rangle$.
        \item For each $r\in\mathbb{Z}$ with $r\geq0$, let $\mathbb{Z}^r=\mathbb{Z}\times\mathbb{Z}\times\cdots\times\mathbb{Z}$ be the direct product of $r$ copies of the group $\mathbb{Z}$, where $\mathbb{Z}^0=1$. The group $\mathbb{Z}^r$ is called the \textit{free abelian group of rank $r$}.
    \end{enumerate}
\end{defn}
\begin{theorem}(Fundamental Theorem of Finitely Generated Abelian Groups).
    Let $G$ be a finitely generated abelian group. Then
    \begin{enumerate}
        \item $G\cong\mathbb{Z}^r\times\mathbb{Z}_{n_1}\times\mathbb{Z}_{n_2}\times\cdots\times\mathbb{Z}_{n_s}$, for some integers $r,n_1,n_2,\ldots,n_s$ satisfying the following conditions:
        \begin{itemize}
            \item $r\geq0$ and $n_j\geq2$ for all $j$.
            \item $n_{i+1}\mid n_i$ for $1\leq i\leq s-1$.
        \end{itemize}
        \item The expression in (1) is unique: if $G\cong\mathbb{Z}^t\times\mathbb{Z}_{m_1}\times\mathbb{Z}_{m_2}\times\cdots\times\mathbb{Z}_{m_u}$, where $t$ and $m_1,m_2,\ldots,m_u$ satisfy the above conditions (i.e., $t\geq0$, $m_j\geq2$ for all $j$ and $m_{i+1}\mid m_i$ for $1\leq i\leq u-1$), then $t=r$, $u=s$ and $m_i=n_i$ for all $i$.
    \end{enumerate}
\end{theorem}
\begin{defn}(Free rank/Betti number of a group, invariant factors, invariant factor decomposition).
    The integer $r$ in Theorem 1.105 is called the \textit{free rank} or \textit{Betti number} of $G$ and the integers $n_1,n_2,\ldots,n_s$ are called the \textit{invariant factors} of $G$. The description of $G$ in Theorem 1.105(1) is called the \textit{invariant factor decomposition} of $G$.
\end{defn}
\begin{corollary}
    If $n$ is the product of distinct primes, then up to isomorphism the only abelian group of order $n$ is the cyclic group of order $n$, $\mathbb{Z}_n$.
\end{corollary}
\begin{theorem}
    Let $G$ be an abelian group of order $n>1$ and let the unique factorization of $n$ into distinct prime powers be $n=p_1^{\alpha_1}p_2^{\alpha_2}\cdots p_k^{\alpha_k}$. Then
    \begin{enumerate}
        \item $G\cong A_1\times A_2\times\cdots\times A_k$, where $|A_i|=p_i^{\alpha_i}$.
        \item For each $A\in\{A_1,A_2,\ldots,A_k\}$ with $|A|=p^{\alpha}$, $A\cong\mathbb{Z}_{p^{\beta_1}}\times\mathbb{Z}_{p^{\beta_2}}\times\cdots\times\mathbb{Z}_{p^{\beta_t}}$ with $\beta_1\geq\beta_2\geq\cdots\geq\beta_t\geq1$ and $\beta_1+\beta_2+\cdots+\beta_t=\alpha$ (where $t$ and $\beta_1,\ldots,\beta_t$ depend on $i$).
        \item The decomposition in (1) and (2) is unique, i.e., if $G\cong B_1\times B_2\times\cdots\times B_m$, with $|B_i|=p_i^{\alpha_i}$ for all $i$, then $B_i\cong A_i$ and $B_i$ and $A_i$ have the same invariant factors.
    \end{enumerate}
\end{theorem}
\begin{defn}(Elementary divisors, elementary divisor decomposition).
    The integers $p^{\beta_j}$ described in Theorem 1.108 are called the \textit{elementary divisors} of $G$. The description of $G$ in Theorem 1.108(1) and 1.108(2) is called the \textit{elementary divisor decomposition} of $G$.
\end{defn}
\begin{prop}
    Let $m,n\in\mathbb{Z}^+$.
    \begin{enumerate}
        \item $\mathbb{Z}_m\times\mathbb{Z}_n\cong\mathbb{Z}_{mn}$ if and only if $(m,n)=1$.
        \item If $n=p_1^{\alpha_1}p_2^{\alpha_2}\cdots p_k^{\alpha_k}$ then $\mathbb{Z}_n\cong\mathbb{Z}_{p_1^{\alpha_1}}\times\mathbb{Z}_{p_2^{\alpha_2}}\times\cdots\times\mathbb{Z}_{p_k^{\alpha_k}}$.
    \end{enumerate}
\end{prop}
\begin{defn}(Rank of a group, exponent).
    \begin{enumerate}
        \item If $G$ is a finite abelian group of type $(n_1,n_2,\ldots,n_t)$, the integer $t$ is called the \textit{rank} of $G$ (the free rank of $G$ is 0 so there will be no confusion).
        \item If $G$ is any group, the \textit{exponent} of $G$ is the smallest positive integer $n$ such that $x^n=1$ for all $x\in G$ (if no such integer exists the exponent of $G$ is $\infty$).
    \end{enumerate}
\end{defn}
\begin{defn}(Commutator, commutator subgroup).
    Let $G$ be a group, let $x,y\in G$ and let $A,B$ be nonempty subsets of $G$.
    \begin{enumerate}
        \item Define $[x,y]=x^{-1}y^{-1}xy$, called the \textit{commutator} of $x$ and $y$.
        \item Define $[A,B]=\langle[a,b]:a\in A,b\in B\rangle$, the group generated by commutators of elements from $A$ and from $B$.
        \item Define $G'=\langle[x,y]:x,y\in G\rangle$, the subgroup of $G$ generated by commutators of elements from $G$, called the \textit{commutator subgroup} of $G$.
    \end{enumerate}
\end{defn}
\begin{prop}
    Let $G$ be a group, let $x,y\in G$ and let $H\leq G$. Then
    \begin{enumerate}
        \item $xy=yx[x,y]$ (in particular, $xy=yx$ if and only if $[x,y]=1$.
        \item $H\trianglelefteq G$ if and only if $[H,G]\leq H$.
        \item $\sigma[x,y]=[\sigma(x),\sigma(y)]$ for any automorphism $\sigma$ of $G,G'\characteristic G$ and $G/G'$ is abelian.
        \item $G/G'$ is the largest abelian quotient of $G$ in the sense that if $H\trianglelefteq G$ and $G/H$ is abelian, then $G'\leq H$. Conversely, if $G'\leq H$, then $H\trianglelefteq G$ and $G/H$ is abelian.
        \item If $\varphi:G\rightarrow A$ is any homomorphism of $G$ into an abelian group $A$, then $\varphi$ factors through $G'$ i.e., $G'\leq\ker{\varphi}$.
    \end{enumerate}
\end{prop}
\begin{prop}
    Let $H$ and $K$ be subgroups of the group $G$. The number of distinct ways of writing each element of the set $HK$ in the form $hk$, for some $h\in H$ and $k\in K$ is $|H\cap K|$. In particular, if $H\cap K=1$, then each element of $HK$ can be written uniquely as a product $hk$, for some $h\in H$ and $k\in K$.
\end{prop}
\begin{theorem}
    Suppose $G$ is a group with subgroups $H$ and $K$ such that
    \begin{enumerate}
        \item $H$ and $K$ are normal in $G$.
        \item $H\cap K=1$.
    \end{enumerate}
\end{theorem}
\begin{defn}(Internal/external direct product).
    If $G$ is a group and $H$ and $K$ are normal subgroups of $G$ with $H\cap K=1$, we call $HK$ the \textit{internal direct product} of $H$ and $K$. We shall (when emphasis is called for) call $H\times K$ the \textit{external direct product} of $H$ and $K$.
\end{defn}
\begin{theorem}
    Let $H$ and $K$ be groups and let $\varphi$ be a homomorphism from $K$ into $\Aut(H)$. Let $\cdot$ denote the (left) action of $K$ on $H$ determined by $\varphi$. Let $G$ be the set of ordered pairs $(h,k)$ with $h\in H$ and $k\in K$ and define the following multiplication on $G$: $(h_1,k_1)(h_2,k_2)=(h_1k_1\cdot h_2,k_1k_2)$.
    \begin{enumerate}
        \item This multiplication makes $G$ into a group of order $|G|=|H||K|$.
        \item The sets $\{(h,1):h\in H\}$ and $\{(1,k):k\in K\}$ are subgroups of $G$ and the maps $h\mapsto(h,1)$ for $h\in H$ and $k\mapsto(1,k)$ for $k\in K$ are isomorphisms of these subgroups with the groups $H$ and $K$ respectively: $H\cong\{(h,1):h\in H\}$ and $K\cong\{(1,k):k\in K\}$.
        \item $H\trianglelefteq G$.
        \item $H\cap K=1$.
        \item For all $h\in H$ and $k\in K$, $khk^{-1}=k\cdot h=\varphi(k)(h)$.
    \end{enumerate}
\end{theorem}
\begin{defn}(Semidirect product).
    Let $H$ and $K$ be groups and let $\varphi$ be a homomorphism from $K$ into $\Aut(H)$. The group described in Theorem 1.117 is called the \textit{semidirect product} of $H$ and $K$ with respect to $\varphi$ and will be denoted by $H\rtimes_{\varphi}K$ (when there is no danger of confusion we shall simply write $H\rtimes K$).
\end{defn}
\begin{prop}
    Let $H$ and $K$ be groups and let $\varphi:K\rightarrow\Aut(H)$ be a homomorphism. Then the following are equivalent:
    \begin{enumerate}
        \item The identity (set) map between $H\rtimes K$ and $H\times K$ is a group homomorphism (hence an isomorphism).
        \item $\varphi$ is the trivial homomorphism from $K$ into $\Aut(H)$.
        \item $K\trianglelefteq H\rtimes K$.
    \end{enumerate}
\end{prop}
\begin{theorem}
    Suppose $G$ is a group with subgroups $H$ and $K$ such that $H\trianglelefteq G$ and $H\cap K=1$. Let $\varphi:K\rightarrow\Aut(H)$ be the homomorphism defined by mapping $k\in K$ to the automorphism of left conjugation by $k$ on $H$. Then $HK\cong H\rtimes K$. In particular, if $G=HK$ with $H$ and $K$ satisfying the above conditions, then $G$ is the semidirect product of $H$ and $K$.
\end{theorem}
\begin{defn}(Complement).
    Let $H$ be a subgroup of the group $G$. A subgroup $K$ of $G$ is called a \textit{complement} for $H$ in $G$ if $G=HK$ and $H\cap K=1$.
\end{defn}
\subsubsection{Further Topics in Group Theory}
\begin{defn}(Maximal subgroup).
    A \textit{maximal subgroup} of a group $G$ is a proper subgroup $M$ of $G$ such that there are no subgroups $H$ of $G$ with $M<H<G$.
\end{defn}
\begin{theorem}
    Let $p$ be a prime and let $P$ be a group of order $p^a,a\geq1$. Then
    \begin{enumerate}
        \item The center of $P$ is nontrivial: $Z(P)\neq1$.
        \item If $H$ is a nontrivial normal subgroup of $P$ then $H$ intersects the center non-trivially: $H\cap Z(P)\neq1$. In particular, every normal subgroup of order $p$ is contained in the center.
        \item If $H$ is a normal subgroup of $P$ then $H$ contains a subgroup of order $p^b$ that is normal in $P$ for each divisor $p^b$ of $|H|$. In particular, $P$ has a normal subgroup of order $p^b$ for every $b\in\{0,1,\ldots,a\}$.
        \item If $H<P$ then $H<N_P(H)$ (i.e., every proper subgroup of $P$ is a proper subgroup of its normalizer in $P$).
        \item Every maximal subgroup of $P$ is of index $p$ and is normal in $P$.
    \end{enumerate}
\end{theorem}
\begin{defn}(Upper central series, nilpotent, nilpotence class).
    \begin{enumerate}
        \item For any (finite or infinite) group $G$ define the following subgroups inductively: $Z_0(G)=1$, $Z_1(G)=Z(G)$ and $Z_{i+1}(G)$ is the subgroup of $G$ containing $Z_i(G)$ such that $Z_{i+1}(G)/Z_i(G)=Z(G/Z_i(G))$ (i.e., $Z_{i+1}(G)$ is the complete preimage in $G$ of the center of $G/Z_i(G)$ under the natural projection). The chain of subgroups $Z_0(G)\leq Z_1(G)\leq Z_2(G)\leq\cdots$ is called the \textit{upper central series of $G$} (the use of the term 'upper' indicates that $Z_i(G)\leq Z_{i+1}(G)$).
        \item A group $G$ is called \textit{nilpotent} if $Z_c(G)=G$ for some $c\in\mathbb{Z}$. The smallest such $c$ is called the nilpotence class of $G$.
    \end{enumerate}
\end{defn}
\begin{prop}
    Let $p$ be a prime and let $P$ be a group of order $p^a$. Then $P$ is nilpotent of nilpotence class at most $a-1$.
\end{prop}
\begin{theorem}
    Let $G$ be a finite group, let $p_1,p_2,\ldots,p_s$ be the distinct primes dividing its order and let $P_i\in\Syl_{p_i}(G),1\leq i\leq s$. Then the following are equivalent:
    \begin{enumerate}
        \item $G$ is nilpotent.
        \item If $H<G$ then $H<N_G(H)$, i.e., every proper subgroup of $G$ is a proper subgroup of its normalizer in $G$.
        \item $P_i\trianglelefteq G$ for $1\leq i\leq s$, i.e., every Sylow subgroup is normal in $G$.
        \item $G\cong P_1\times P_2\times\cdots\times P_s$.
    \end{enumerate}
\end{theorem}
\begin{corollary}
    A finite abelian group is the direct product of its Sylow subgroups.
\end{corollary}
\begin{prop}
    If $G$ is a finite group such that for all positive integers $n$ dividing its order, $G$ contains at most $n$ elements $x$ satisfying $x^n=1$, then $G$ is cyclic.
\end{prop}
\begin{prop}(Frattini's Argument).
    Let $G$ be a finite group, let $H$ be a normal subgroup of $G$ and let $P$ be a Sylow $p$-subgroup of $H$. Then $G=HN_G(P)$ and $|G:H|$ divides $|N_G(P)|$.
\end{prop}
\begin{prop}
    A finite group is nilpotent if and only if every maximal subgroup is normal.
\end{prop}
\begin{defn}(Lower central series).
    For any (finite or infinite) group $G$ define the following subgroups inductively: $G^0=G$, $G^1=[G,G]$ and $G^{i+1}=[G,G^i]$. The chain of groups $G^0\geq G^1\geq G^2\geq\cdots$ is called the \textit{lower central series of $G$} (the term 'lower' indicates that $G^i\geq G^{i+1}$).
\end{defn}
\begin{theorem}
    A group $G$ is nilpotent if and only if $G^n=1$ for some $n\geq0$. More precisely, $G$ is nilpotent of class $c$ if and only if $c$ is the smallest nonnegative integer such that $G^c=1$. If $G$ is nilpotent of class $c$ then $Z_i(G)\leq G^{c-i-1}\leq Z_{i+1}(G)$ for all $i\in\{0,1,\ldots,c-1\}$.
\end{theorem}
\begin{defn}(Derived/commutator series).
    For any group $G$ define the following sequence of subgroups inductively: $G^{(0)}=G$, $G^{(1)}=[G,G]$ and $G^{(i+1)}=[G^{(i)},G^{(i)}]$ for all $i\geq1$. This series of subgroups is called the \textit{derived} or \textit{commutator} series of $G$.
\end{defn}
\begin{theorem}
    A group $G$ is solvable if and only if $G^{(n)}=1$ for some $n\geq0$.
\end{theorem}
\begin{prop}
    Let $G$ and $K$ be groups, let $H$ be a subgroup of $G$ and let $\varphi:G\rightarrow K$ be a surjective homomorphism.
    \begin{enumerate}
        \item $H^{(i)}\leq G^{(i)}$ for all $i\geq0$. In particular, if $G$ is solvable, then so is $H$, i.e., subgroups of solvable groups are solvable (and the solvable length of $H$ is less than or equal to the solvable length of $G$).
        \item $\varphi(G^{(i)})=K^{(i)}$. In particular, homomorphic images and quotient groups of solvable groups are solvable (of solvable length less than or equal to that of the domain group).
        \item If $N$ is normal in $G$ and both $N$ and $G/N$ are solvable then so is $G$.
    \end{enumerate}
\end{prop}
\begin{theorem}
    Let $G$ be a finite group.
    \item (Burnside). If $|G|=p^aq^b$ for some primes $p$ and $q$, then $G$ is solvable.
    \item (Hall). If for every prime $p$ dividing $|G|$ we factor the order of $G$ as $|G|=p^am$ where $(p,m)=1$, and $G$ has a subgroup of order $m$, then $G$ is solvable (i.e., if for all primes $p$, $G$ has a subgroup whose index equals the order of a Sylow $p$-subgroup, then $G$ is solvable - such subgroups are called Sylow $p$-complements).
    \item (Feit-Thompson). If $|G|$ is odd then $G$ is solvable.
    \item (Thompson). If for every pair of elements $x,y\in G$, $\langle x,y\rangle$ is a solvable group, then $G$ is solvable.
\end{theorem}
\begin{prop}
    \begin{enumerate}
        \item If $G$ has no subgroup of index 2 and $G\leq S_k$, then $G\leq A_k$.
        \item If $P\in\Syl_p(S_k)$ for some odd prime $p$, then $P\in\Syl_p(A_k)$ and $|N_{A_k}(P)|=\frac{1}{2}|N_{S_k}(P)|$.
    \end{enumerate}
\end{prop}
\begin{lemma}
    In a finite group $G$ if $n_p\not\equiv1\mod p^2$, then there are distinct Sylow $p$-subgroups $P$ and $R$ of $G$ such that $P\cap R$ is of index $p$ in both $P$ and $R$ (hence is normal in each).
\end{lemma}
\begin{prop}
    If $G$ is a simple group of order 168, then the following hold:
    \begin{enumerate}
        \item $n_2=21$, $n_3=7$ and $n_7=8$.
        \item Sylow 2-subgroups of $G$ are dihedral, Sylow 3- and 7-subgroups are cyclic.
        \item $G$ is isomorphic to a subgroup of $A_7$ and $G$ has no subgroup of index less than 7.
        \item The conjugacy classes of $G$ are the following: the identity; two classes of elements of order 7 each of which contains 24 elements (represented by any element of order 7 and its inverse); one class of elements of order 3 containing 56 elements; one class of elements of order 4 containing 42 elements; one class of elements of order 2 containing 21 elements (in particular, every element of $G$ has order a power of a prime).
        \item If $T\in\Syl_2(G)$ and $U,W$ are the two Klein 4-groups in $T$, then $U$ and $W$ are not conjugate in $G$ and $N_G(U)\cong N_G(W)\cong S_4$.
        \item $G$ has precisely three conjugacy classes of maximal subgroups, two of which are isomorphic to $S_4$ and one of which is isomorphic to the non-abelian group of order 21.
    \end{enumerate}
\end{prop}
\begin{theorem}
    Up to isomorphism there is a unique simple group of order 168, $GL_3(\mathbb{F}_2)$, which is also the automorphism group of the projective plane $\mathcal{F}$.
\end{theorem}
\begin{defn}(Free group, free generators/basis, rank).
    The group $F(S)$ is called the \textit{free group} on the set $S$. A group $F$ is a \textit{free group} if there is some set $S$ such that $F=F(S)$ - in this case we call $S$ a set of \textit{free generators} (or a \textit{free basis}) of $F$. The cardinality of $S$ is called the \textit{rank} of the free group.
\end{defn}
\begin{theorem}(Schreier).
    Subgroups of a free group are free.
\end{theorem}
\begin{defn}(Presentation, generators, relations, finitely generated group, finitely presented).
    Let $S$ be a subset of a group $G$ such that $G=\langle S\rangle$.
    \begin{enumerate}
        \item A \textit{presentation} for $G$ is a pair $(S,R)$, where $R$ is a set of words in $F(S)$ such that the normal closure of $\langle R\rangle$ in $F(S)$ (the smallest normal subgroup containing $\langle R\rangle$) equals the kernel of the homomorphism $\pi:F(S)\rightarrow G$ (where $\pi$ extends the identity map from $S$ to $S$). The elements of $S$ are called \textit{generators} and those of $R$ are called \textit{relations} of $G$.
        \item We say $G$ is \textit{finitely generated} if there is a presentation $(S,R)$ such that $S$ if a finite set and we say $G$ is \textit{finitely presented} if there is a presentation $(S,R)$ with both $S$ and $R$ finite sets.
    \end{enumerate}
\end{defn}
\subsection{Ring Theory}
\subsubsection{Introduction to Rings}
\begin{defn}(Ring).
    \begin{enumerate}
        \item A \textit{ring} $R$ is a set together with two binary operations $+$ and $\times$ (called addition and multiplication) satisfying the following axioms:
        \begin{itemize}
            \item $(R,+)$ is an \textit{abelian} group.
            \item $\times$ is associative: $(a\times b)\times c=a\times(b\times c)$ for all $a,b,c\in R$.
            \item The \textit{distributive laws} hold in $R$: for all $a,b,c\in R$, $(a+b)\times c=(a\times c)+(b\times c)$ and $a\times(b+c)=(a\times b)+(a\times c)$.
        \end{itemize}
        \item The ring $R$ is \textit{commutative} if multiplication is commutative.
        \item The ring $R$ is said to have an \textit{identity} (or \textit{contain a 1}) if there is an element $1\in R$ with $1\times a=a\times 1=a$ for all $a\in R$.
    \end{enumerate}
\end{defn}
\begin{defn}(Division ring/skew field, field).
    A ring $R$ with identity 1, where $1\neq0$, is called a \textit{division ring} (or \textit{skew field}) if every nonzero element $a\in R$ has a multiplicative inverse, i.e., there exists $b\in R$ such that $ab=ba=1$. A commutative division ring is called a \textit{field}.
\end{defn}
\begin{prop}
    Let $R$ be a ring. Then
    \begin{enumerate}
        \item $0a=a0=0$ for all $a\in R$.
        \item $(-a)b=a(-b)=-(ab)$ for all $a,b\in R$ (recall $-a$ is the additive inverse of $a$).
        \item $(-a)(-b)=ab$ for all $a,b\in R$.
        \item If $R$ has an identity 1, then the identity is unique and $-a=(-1)a$.
    \end{enumerate}
\end{prop}
\begin{defn}(Zero divisor, unit).
    Let $R$ be a ring.
    \begin{enumerate}
        \item A nonzero element $a$ of $R$ is called a \textit{zero divisor} if there is a nonzero element $b$ in $R$ such that either $ab=0$ or $ba=0$.
        \item Assume $R$ has an identity $1\neq0$. An element $u$ of $R$ is called a \textit{unit} in $R$ if there is some $v$ in $R$ such that $uv=vu=1$. The set of units in $R$ is denoted $R^\times$.
    \end{enumerate}
\end{defn}
\begin{defn}(Integral domain).
    A commutative ring with identity $1\neq0$ is called an \textit{integral domain} if it has no zero divisors.
\end{defn}
\begin{prop}
    Assume $a$, $b$ and $c$ are elements of any ring with $a$ not a zero divisor. If $ab=ac$, then either $a=0$ or $b=c$ (i.e., if $a\neq0$ we can cancel the $a$'s). In particular, if $a,b,c$ are any elements in an integral domain and $ab=ac$, then either $a=0$ or $b=c$.
\end{prop}
\begin{corollary}
    Any finite integral domain is a field.
\end{corollary}
\begin{defn}(Subring).
    A \textit{subring} of the ring $R$ is a subgroup of $R$ that is closed under multiplication.
\end{defn}
\begin{prop}
    Let $R$ be an integral domain and let $p(x),q(x)$ be nonzero elements of $R[x]$. Then
    \begin{enumerate}
        \item $\deg(p(x)q(x))=\deg(p(x))+\deg(q(x))$.
        \item The units of $R[x]$ are just the units of $R$.
        \item $R[x]$ is an integral domain.
    \end{enumerate}
\end{prop}
\begin{defn}(Ring homomorphism, kernel, isomorphism).
    Let $R$ and $S$ be rings.
    \begin{enumerate}
        \item A \textit{ring homomorphism} is a map $\varphi:R\rightarrow S$ satisfying
        \begin{itemize}
            \item $\varphi(a+b)=\varphi(a)+\varphi(b)$ for all $a,b\in R$ (so $\varphi$ is a group homomorphism on the additive groups).
            \item $\varphi(ab)=\varphi(a)\varphi(b)$ for all $a,b\in R$.
        \end{itemize}
        \item The \textit{kernel} of the ring homomorphism $\varphi$, denoted $\ker{\varphi}$, is the set of elements of $R$ that map to 0 in $S$ (i.e., the kernel of $\varphi$ viewed as a homomorphism of additive groups).
        \item A bijective ring homomorphism is called an \textit{isomorphism}.
    \end{enumerate}
\end{defn}
\begin{prop}
    Let $R$ and $S$ be rings and let $\varphi:R\rightarrow S$ be a homomorphism.
    \begin{enumerate}
        \item The image of $\varphi$ is a subring of $S$.
        \item The kernel of $\varphi$ is a subring of $R$. Furthermore, if $\alpha\in\ker{\varphi}$ then $r\alpha$ and $\alpha r\in\ker{\varphi}$ for every $r\in R$, i.e., $\ker{\varphi}$ is closed under multiplication by elements from $R$.
    \end{enumerate}
\end{prop}
\begin{defn}(Left/right ideal, ideal).
    Let $R$ be a ring, let $I$ be a subset of $R$ and let $r\in R$.
    \begin{enumerate}
        \item $rI=\{ra:a\in I\}$ and $Ir=\{ar:a\in I\}$.
        \item A subset $I$ of $R$ is a \textit{left/right ideal} of $R$ if
        \begin{itemize}
            \item $I$ is a subring of $R$.
            \item $I$ is closed under left/right multiplication by elements from $R$.
        \end{itemize}
        \item A subset $I$ that is both a left ideal and a right ideal is called an \textit{ideal} (or, for added emphasis, a \textit{two-sided ideal}) of $R$.
    \end{enumerate}
\end{defn}
\begin{prop}
    Let $R$ be a ring and let $I$ be an ideal of $R$. Then the (additive) quotient group $R/I$ is a ring under the binary operations: $(r+I)+(s+I)=(r+s)+I$ and $(r+I)\times(s+I)=(rs)+I$ for all $r,s\in R$. Conversely, if $I$ is any subgroup such that the above operations are well defined, then $I$ is an ideal of $R$.
\end{prop}
\begin{defn}(Quotient ring).
    When $I$ is an ideal of $R$ the ring $R/I$ with the operations in Proposition 1.156 is called the \textit{quotient ring} of $R$ by $I$.
\end{defn}
\begin{theorem}
    \begin{enumerate}
        \item (The First Isomorphism Theorem for Rings). If $\varphi:R\rightarrow S$ is a homomorphism of rings, then the kernel of $\varphi$ is an ideal of $R$, the image of $\varphi$ is a subring of $S$ and $R/\ker{\varphi}$ is isomorphic as a ring to $\varphi(R)$.
        \item If $I$ is any ideal of $R$, then the map $R\rightarrow R/I$ defined by $r\mapsto r+I$ is a surjective ring homomorphism with kernel $I$ (this homomorphism is called the \textit{natural projection} of $R$ onto $R/I$). Thus every ideal is the kernel of a ring homomorphism and vice versa.
    \end{enumerate}
\end{theorem}
\begin{theorem}
    \begin{enumerate}
        \item (The Second Isomorphism Theorem for Rings). Let $A$ be a subring and let $B$ be an ideal of $R$. Then $A+B=\{a+b:a\in A,b\in B\}$ is a subring of $R$, $A\cap B$ is an ideal of $A$ and $(A+B)/B\cong A/(A\cap B)$.
        \item (The Third Isomorphism Theorem for Rings). Let $I$ and $J$ be ideals of $R$ with $I\subseteq J$. Then $J/I$ is an ideal of $R/I$ and $(R/I)/(J/I)\cong R/J$.
        \item (The Fourth or Lattice Isomorphism Theorem for Rings). Let $I$ be an ideal of $R$. The correspondence $A\leftrightarrow A/I$ is an inclusion preserving bijection between the set of subrings $A$ of $R$ that contain $I$ and the set of subrings of $R/I$. Furthermore, $A$ (a subring containing $I$) is an ideal of $R$ is and only if $A/I$ is an ideal of $R/I$.
    \end{enumerate}
\end{theorem}
\begin{defn}(Sum, product, $n$th power).
    Let $I$ and $J$ be ideals of $R$.
    \begin{enumerate}
        \item Define the \textit{sum} of $I$ and $J$ by $I+J=\{a+b:a\in I,b\in J\}$.
        \item Define the \textit{product} of $I$ and $J$, denoted by $IJ$, to be the set of all finite sums of elements of the form $ab$ with $a\in I$ and $b\in J$.
        \item For any $n\geq1$, define the \textit{$n$th power} of $I$, denoted by $I^n$, to be the set consisting of all finite sums of elements of the form $a_1a_2\cdots a_n$ with $a_i\in I$ for all $i$. Equivalently, $I^n$ is defined inductively by defining $I^1=I$ and $I^n=II^{n-1}$ for $n=2,3,\ldots$.
    \end{enumerate}
\end{defn}
\begin{defn}(Ideal generated by a subset, principal ideal, finitely generated ideal).
    Let $A$ be any subset of the ring $R$.
    \begin{enumerate}
        \item Let $(A)$ denote the smallest ideal of $R$ containing $A$, called \textit{the ideal generated by $A$}.
        \item Let $RA$ denote the set of all finite sums of elements of the form $ra$ with $r\in R$ and $a\in A$ i.e., $RA=\{r_1a_1+r_2a_2+\cdots+r_na_n:r_i\in R,a_i\in A,n\in\mathbb{Z}^+\}$ and $RAR=\{r_1a_1r_1'+r_2a_2r_2'+\cdots+r_na_nr_n':r_i,r_i'\in R,a_i\in A,n\in\mathbb{Z}^+\}$.
        \item An ideal generated by a single element is called a \textit{principal ideal}.
        \item An ideal generated by a finite set is called a \textit{finitely generated ideal}.
    \end{enumerate}
\end{defn}
\begin{prop}
    Let $I$ be an ideal of $R$.
    \begin{enumerate}
        \item $I=R$ if and only if $I$ contains a unit.
        \item Assume $R$ is commutative. Then $R$ is a field if and only if its only ideals are 0 and $R$.
    \end{enumerate}
\end{prop}
\begin{corollary}
    If $R$ is a field then any nonzero ring homomorphism from $R$ into another ring is an injection.
\end{corollary}
\begin{defn}(Maximal ideal).
    An ideal $M$ in an arbitrary ring $S$ is called a \textit{maximal ideal} if $M\neq S$ and the only ideals containing $M$ are $M$ and $S$.
\end{defn}
\begin{prop}
    In a ring with identity every proper ideal is contained in a maximal ideal.
\end{prop}
\begin{prop}
    Assume $R$ is commutative. The ideal $M$ is a maximal ideal if and only if the quotient ring $R/M$ is a field.
\end{prop}
\begin{defn}(Prime ideal).
    Assume $R$ is commutative. An ideal $P$ is called a \textit{prime ideal} if $P\neq R$ and whenever the product $ab$ of two elements $a,b \in R$ is an element of $P$, then at least one of $a$ and $b$ is an element of $P$.
\end{defn}
\begin{prop}
    Assume $R$ is commutative. Then the ideal $P$ is a prime ideal in $R$ if and only if the quotient ring $R/P$ is an integral domain.
\end{prop}
\begin{corollary}
    Assume $R$ is commutative. Every maximal ideal of $R$ is a prime ideal.
\end{corollary}
\begin{theorem}
    Let $R$ be a commutative ring. Let $D$ be any nonempty subset of $R$ that does not contain 0, does not contain any zero divisors and is closed under multiplication (i.e., $ab\in D$ for all $a,b\in D$). Then there is a commutative ring $Q$ with 1 such that $Q$ contains $R$ as a subring and every element of $D$ is a unit in $Q$. The ring $Q$ has the following additional properties.
    \begin{enumerate}
        \item Every element of $Q$ is of the form $rd^{-1}$ for some $r\in R$ and $d\in D$. In particular, if $D=R-\{0\}$ then $Q$ is a field.
        \item (Uniqueness of $Q$). The ring $Q$ is the 'smallest' ring containing $R$ in which all elements of $D$ become units, in the following sense. Let $S$ be any commutative ring with identity and let $\varphi:R\rightarrow S$ be any injective ring homomorphism such that $\varphi(d)$ is a unit in $S$ for every $d\in D$. Then there is an injective homomorphism $\Phi:Q\rightarrow S$ such that $\Phi|_R=\varphi$. In other words, any ring containing an isomorphic copy of $R$ in which all the elements of $D$ become units must also contain an isomorphic copy of $Q$.
    \end{enumerate}
\end{theorem}
\begin{defn}(Ring of fractions, field of fractions/quotient field).
    Let $R$, $D$ and $Q$ be as in Theorem 1.170.
    \begin{enumerate}
        \item The ring $Q$ is called the \textit{ring of fractions} of $D$ with respect to $R$ and is denoted $D^{-1}R$.
        \item If $R$ is an integral domain and $D=R-\{0\}$, $Q$ is called the \textit{field of fractions} or \textit{quotient field} of $R$.
    \end{enumerate}
\end{defn}
\begin{corollary}
    Let $R$ be an integral domain and let $Q$ be the field of fractions of $R$. If a field $F$ contains a subring $R'$ isomorphic to $R$ then the subfield of $F$ generated by $R'$ is isomorphic to $Q$.
\end{corollary}
\begin{defn}(Comaximal).
    The ideals $A$ and $B$ of the ring $R$ are said to be \textit{comaximal} if $A+B=R$.
\end{defn}
\begin{theorem}(Chinese Remainder Theorem).
    Let $A_1,A_2,\ldots,A_k$ be ideals in $R$. The map $R\rightarrow R/A_1\times R/A_2\times\cdots\times R/A_k$ defined by $r\mapsto(r+A_1,r+A_2,\ldots,r+A_k)$ is a ring homomorphism with kernel $A_1\cap A_2\cap\cdots\cap A_k$. If for each $i$, $j\in\{1,2,\ldots,k\}$ with $i\neq j$ the ideals $A_i$ and $A_j$ are comaximal, then this map is surjective and $A_1\cap A_2\cap\cdots\cap A_k=A_1A_2\cdots A_k$, so $R/(A_1A_2\cdots A_k)=R/(A_1\cap A_2\cap\cdots\cap A_k)\cong R/A_1\times R/A_2\times\cdots\times R/A_k$.
\end{theorem}
\begin{corollary}
    Let $n$ be a positive integer and let $p_1^{\alpha_1}p_2^{\alpha_2}\cdots p_k^{\alpha_k}$ be its factorization into powers of distinct primes. Then $\mathbb{Z}/n\mathbb{Z}\cong(\mathbb{Z}/p_1^{\alpha_1}\mathbb{Z})\times(\mathbb{Z}/p_2^{\alpha_2}\mathbb{Z})\times\cdots\times(\mathbb{Z}/p_k^{\alpha_k}\mathbb{Z})$ as rings, so in particular we have the following isomorphism of multiplicative groups: $(\mathbb{Z}/n\mathbb{Z})^\times\cong(\mathbb{Z}/p_1^{\alpha_1}\mathbb{Z})^\times\times(\mathbb{Z}/p_2^{\alpha_2}\mathbb{Z})^\times\times\cdots\times(\mathbb{Z}/p_k^{\alpha_k}\mathbb{Z})^\times$.
\end{corollary}
\subsubsection{Euclidean Domains, Principal Ideal Domains and Unique Factorization Domains}
\begin{defn}(Norm).
    Any function $N:R\rightarrow\mathbb{Z}^+\cup\{0\}$ with $N(0)=0$ is called a \textit{norm} on the integral domain $R$. If $N(a)>0$ for $a\neq0$ define $N$ to be a \textit{positive norm}.
\end{defn}
\begin{defn}(Euclidean domain, division algorithm, quotient, remainder).
    The integral domain $R$ is said to be a \textit{Euclidean domain} (or possess a \textit{division algorithm}) if there is a norm $N$ on $R$ such that for any two elements $a$ and $b$ of $R$ with $b\neq0$ there exist elements $q$ and $r$ in $R$ with $a=qb+r$ with $r=0$ or $N(r)<N(b)$. The element $q$ is called the \textit{quotient} and the element $r$ the \textit{remainder} of the division.
\end{defn}
\begin{prop}
    Every ideal in a Euclidean domain is principal. More precisely, if $I$ is any nonzero ideal in the Euclidean domain $R$ then $I=(d)$, where $d$ is any nonzero element of $I$ of minimum norm.
\end{prop}
\begin{defn}(Multiple, divisor, greatest common divisor).
    Let $R$ be a commutative ring and let $a,b\in R$ with $b\neq0$.
    \begin{enumerate}
        \item $a$ is said to be a \textit{multiple} of $b$ if there exists an element $x\in R$ with $a=bx$. In this case $b$ is said to \textit{divide} $a$ or be a \textit{divisor} of $a$, written $b\mid a$.
        \item A \textit{greatest common divisor} of $a$ and $b$ is a nonzero element $d$ such that
        \begin{itemize}
            \item $d\mid a$ and $d\mid b$.
            \item If $d'\mid a$ and $d'\mid b$ then $d'\mid d$.
        \end{itemize}
    \end{enumerate}
\end{defn}
\begin{prop}
    If $a$ and $b$ are nonzero elements in the commutative ring $R$ such that the ideal generated by $a$ and $b$ is a principal ideal $(d)$, then $d$ is a greatest common divisor of $a$ and $b$.
\end{prop}
\begin{prop}
    Let $R$ be an integral domain. If two elements $d$ and $d'$ of $R$ generate the same principal ideal, i.e., $(d)=(d')$, then $d'=ud$ for some unit $u$ in $R$. In particular, if $d$ and $d'$ are both greatest common divisors of $a$ and $b$, then $d'=ud$ for some unit $u$.
\end{prop}
\begin{theorem}
    Let $R$ be a Euclidean domain and let $a$ and $b$ be nonzero elements of $R$. Let $d=r_n$ be the last nonzero remainder in the Euclidean algorithm for $a$ and $b$. Then
    \begin{enumerate}
        \item $d$ is a greatest common divisor of $a$ and $b$.
        \item The principal ideal $(d)$ is the ideal generated by $a$ and $b$. In particular, $d$ can be written as an \textit{$R$-linear combination} of $a$ and $b$, i.e., there are elements $x$ and $y$ in $R$ such that $d=ax+by$.
    \end{enumerate}
\end{theorem}
\begin{prop}
    Let $R$ be an integral domain that is not a field. If $R$ is a Euclidean domain then there are universal side divisors in $R$.
\end{prop}
\begin{defn}(Principal ideal domain).
    A \textit{principal ideal domain} (P.I.D.) is an integral domain in which every ideal is principal.
\end{defn}
\begin{prop}
    Let $R$ be a P.I.D. and let $a$ and $b$ be nonzero elements of $R$. Let $d$ be a generator for the principal ideal generated by $a$ and $b$. Then
    \begin{enumerate}
        \item $d$ is a greatest common divisor of $a$ and $b$.
        \item $d$ can be written as an \textit{$R$-linear combination} of $a$ and $b$, i.e., there are elements $x$ and $y$ in $R$ with $d=ax+by$.
        \item $d$ is unique up to multiplication by a unit of $R$.
    \end{enumerate}
\end{prop}
\begin{prop}
    Every nonzero prime ideal in a P.I.D. is a maximal ideal.
\end{prop}
\begin{corollary}
    If $R$ is any commutative ring such that the polynomial ring $R[x]$ is a P.I.D. (or a Euclidean domain), then $R$ is necessarily a field.
\end{corollary}
\begin{defn}(Dedekind-Hasse norm).
    Define $N$ to be a \textit{Dedekind-Hasse norm} if $N$ is a positive norm and for every nonzero $a,b\in R$ either $a$ is an element of the ideal $(b)$ or there is a nonzero element in the ideal $(a,b)$ of norm strictly smaller than the norm of $b$ (i.e., either $b$ divides $a$ in $R$ or there exist $s,t\in R$ with $0<N(sa-tb)<N(b)$).
\end{defn}
\begin{prop}
    The integral domain $R$ is a P.I.D. if and only if $R$ has a Dedekind-Hasse norm.
\end{prop}
\begin{defn}(Irreducible, reducible, prime, associate).
    Let $R$ be an integral domain.
    \begin{enumerate}
        \item Suppose $r\in R$ is nonzero and is not a unit. Then $r$ is called \textit{irreducible} in $R$ if whenever $r=ab$ with $a,b\in R$, at least one of $a$ or $b$ must be a unit in $R$. Otherwise $r$ is said to be \textit{reducible}.
        \item The nonzero element $p\in R$ is called \textit{prime} in $R$ if the ideal $(p)$ generated by $p$ is a prime ideal. In other words, a nonzero element $p$ is a prime if it is not a unit and whenever $p\mid ab$ for any $a,b\in R$, then either $p\mid a$ or $p\mid b$.
        \item Two elements $a$ and $b$ of $R$ differing by a unit are said to be \textit{associate} in $R$ (i.e., $a=ub$ for some unit $u$ in $R$).
    \end{enumerate}
\end{defn}
\begin{prop}
    In an integral domain a prime element is always irreducible.
\end{prop}
\begin{prop}
    In a P.I.D. a nonzero element is a prime if and only if it is irreducible.
\end{prop}
\begin{defn}(Unique factorization domain).
    A \textit{unique factorization domain} (U.F.D.) is an integral domain $R$ in which every nonzero element $r\in R$ which is not a unit has the following two properties:
    \begin{enumerate}
        \item $r$ can be written as a finite product of irreducibles $p_i$ of $R$ (not necessarily distinct): $r=p_1p_2\cdots p_n$.
        \item The decomposition in (1) is \textit{unique up to associates}: namely, if $r=q_1q_2\cdots q_m$ is another factorization of $r$ into irreducibles, then $m=n$ and there is some renumbering of the factors so that $p_i$ is associate to $q_i$ for $i=1,2,\ldots,n$.
    \end{enumerate}
\end{defn}
\begin{prop}
    In a U.F.D. a nonzero element is a prime if and only if it is irreducible.
\end{prop}
\begin{prop}
    Let $a$ and $b$ be two nonzero elements of the U.F.D. $R$ and suppose $a=up_1^{e_1}p_2^{e_2}\cdots p_n^{e_n}$ and $b=vp_1^{f_1}p_2^{f_2}\cdots p_n^{f_n}$ are prime factorizations for $a$ and $b$, where $u$ and $v$ are units, the primes $p_1,p_2,\ldots,p_n$ are \textit{distinct} and the exponents $e_i$ and $f_i$ are nonnegative. Then the element $d=p_1^{\min(e_1,f_1)}p_2^{\min(e_2,f_2)}\cdots p_n^{\min(e_n,f_n)}$ (where $d=1$ if all the exponents are 0) is a greatest common divisor of $a$ and $b$.
\end{prop}
\begin{theorem}
    Every P.I.D. is a U.F.D. In particular, every Euclidean domain is a U.F.D.
\end{theorem}
\begin{corollary}(Fundamental Theorem of Arithmetic).
    The integers $\mathbb{Z}$ are a U.F.D.
\end{corollary}
\begin{corollary}
    Let $R$ be a P.I.D. Then there exists a multiplicative Dedekind-Hasse norm on $R$.
\end{corollary}
\begin{lemma}
    The prime number $p\in\mathbb{Z}$ divides an integer of the form $n^2+1$ if and only if $p$ is either 2 or is an odd prime congruent to 1 modulo 4.
\end{lemma}
\begin{prop}
    \begin{enumerate}
        \item (Fermat's Theorem on sums of squares). The prime $p$ is the sum of two integer squares, $p=a^2+b^2,a,b\in\mathbb{Z}$, if and only if $p=2$ or $p\equiv1\mod4$. Except for interchanging $a$ and $b$ or changing the signs of $a$ and $b$, the representation of $p$ as a sum of two squares is unique.
        \item The irreducible elements in the Gaussian integers $\mathbb{Z}[i]$ are as follows:
        \begin{itemize}
            \item $1+i$ (which has norm 2).
            \item The primes $p\in\mathbb{Z}$ with $p\equiv3\mod4$ (which have norm $p^2$).
            \item $a+bi,a-bi$, the distinct irreducible factors of $p=a^2+b^2=(a+bi)(a-bi)$ for the primes $p\in\mathbb{Z}$ with $p\equiv1\mod4$ (both of which have norm $p$).
        \end{itemize}
    \end{enumerate}
\end{prop}
\begin{corollary}
    Let $n$ be a positive integer and write $n=2^kp_1^{a_1}\cdots p_r^{a_r}q_1^{b_1}\cdots q_s^{b_s}$ where $p_1,\ldots,p_r$ are distinct primes congruent to 1 modulo 4 and $q_1,\ldots,q_s$ are distinct primes congruent to 3 modulo 4. Then $n$ can be written as a sum of two squares in $\mathbb{Z}$, i.e., $n=A^2+B^2$ with $A,B\in\mathbb{Z}$, if and only if each $b_i$ is even. Further, if this condition on $n$ is satisfied, then the number of representations of $n$ as a sum of two squares is $4(a_1+1)(a_2+1)\cdots(a_r+1)$.
\end{corollary}
\subsubsection{Polynomial Rings}
\begin{prop}
    Let $I$ be an ideal of the ring $R$ and let $(I)=I[x]$ denote the ideal of $R[x]$ generated by $I$ (the set of polynomials with coefficients in $I$). Then $R[x]/(I)\cong(R/I)[x]$. In particular, if $I$ is a prime ideal of $R$ then $(I)$ is a prime ideal of $R[x]$.
\end{prop}
\begin{defn}(Polynomial ring).
    The \textit{polynomial ring in the variables $x_1,x_2,\ldots,x_n$ with coefficients in $R$}, denoted $R[x_1,x_2,\ldots,x_n]$, is defined inductively by $R[x_1,x_2,\ldots,x_n]=R[x_1,x_2,\ldots,x_{n-1}][x_n]$.
\end{defn}
\begin{theorem}
    Let $F$ be a field. The polynomial ring $F[x]$ is a Euclidean domain. Specifically, if $a(x)$ and $b(x)$ are two polynomials in $F[x]$ with $b(x)$ nonzero, then there are \textit{unique} $q(x)$ and $r(x)$ in $F[x]$ such that $a(x)=q(x)b(x)+r(x)$ with $r(x)=0$ or $\deg(r(x))<\deg(b(x))$.
\end{theorem}
\begin{corollary}
    If $F$ is a field, then $F[x]$ is a P.I.D. and a U.F.D.
\end{corollary}
\begin{prop}(Gauss' Lemma).
    Let $R$ be a U.F.D. with field of fractions $F$ and let $p(x)\in R[x]$. If $p(x)$ is reducible in $F[x]$ then $p(x)$ is reducible in $R[x]$. More precisely, if $p(x)=A(x)B(x)$ for some nonconstant polynomials $A(x),B(x)\in F[x]$, then there are nonzero elements $r,s\in F$ such that $rA(x)=a(x)$ and $sB(x)=b(x)$ both lie in $R[x]$ and $p(x)=a(x)b(x)$ is a factorization in $R[x]$.
\end{prop}
\begin{corollary}
    Let $R$ be a U.F.D., let $F$ be its field of fractions and let $p(x)\in R[x]$. Suppose the greatest common divisor of the coefficients of $p(x)$ is 1. Then $p(x)$ is irreducible in $R[x]$ if and only if it is irreducible in $F[x]$. In particular, if $p(x)$ is a monic polynomial that is irreducible in $R[x]$, then $p(x)$ is irreducible in $F[x]$.
\end{corollary}
\begin{theorem}
    $R$ is a U.F.D. if and only if $R[x]$ is a U.F.D.
\end{theorem}
\begin{corollary}
    If $R$ is a U.F.D., then a polynomial ring in an arbitrary number of variables with coefficients in $R$ is also a U.F.D.
\end{corollary}
\begin{prop}
    Let $F$ be a field and let $p(x)\in F[x]$. Then $p(x)$ has a factor of degree one if and only if $p(x)$ has a root in $F$, i.e., there is an $\alpha\in F$ with $p(\alpha)=0$.
\end{prop}
\begin{prop}
    A polynomial of degree two or three over a field $F$ is reducible if and only if it has a root in $F$.
\end{prop}
\begin{prop}
    Let $p(x)=a_nx^n+a_{n-1}x^{n-1}+\cdots+a_0$ be a polynomial of degree $n$ with integer coefficients. If $r/s\in\mathbb{Q}$ is in lowest terms (i.e., $r$ and $s$ are relatively prime integers) and $r/s$ is a root of $p(x)$, then $r$ divides the constant term and $s$ divides the leading coefficient of $p(x)$: $r\mid a_0$ and $s\mid a_n$. In particular, if $p(x)$ is a \textit{monic} polynomial with integer coefficients and $p(d)\neq0$ for all integers $d$ dividing the constant term of $p(x)$, then $p(x)$ has no roots in $\mathbb{Q}$.
\end{prop}
\begin{prop}
    Let $I$ be a proper ideal in the integral domain $R$ and let $p(x)$ be a nonconstant monic polynomial in $R[x]$. If the image of $p(x)$ in $(R/I)[x]$ cannot be factored in $(R/I)[x]$ into two polynomials of smaller degree, then $p(x)$ is irreducible in $R[x]$.
\end{prop}
\begin{prop}(Eisenstein's Criterion).
    Let $P$ be a prime ideal of the integral domain $R$ and let $f(x)=x^n+a_{n-1}x^{n-1}+\cdots+a_1x+a_0$ be a polynomial in $R[x]$ (here $n\geq1$). Suppose $a_{n-1},\ldots,a_1,a_0$ are all elements of $P$ and suppose $a_0$ is not an element of $P^2$. Then $f(x)$ is irreducible in $R[x]$.
\end{prop}
\begin{corollary}(Eisenstein's Criterion for $\mathbb{Z}[x]$).
    Let $p$ be a prime in $\mathbb{Z}$ and let $f(x)=x^n+a_{n-1}x^{n-1}+\cdots+a_1x+a_0\in\mathbb{Z}[x],n\geq1$. Suppose $p$ divides $a_i$ for all $i\in\{0,1,\ldots,n-1\}$ but that $p^2$ does not divide $a_0$. Then $f(x)$ is irreducible in both $\mathbb{Z}[x]$ and $\mathbb{Q}[x]$.
\end{corollary}
\begin{prop}
    The maximal ideals in $F[x]$ are the ideals $(f(x))$ generated by irreducible polynomials $f(x)$. In particular, $F[x]/(f(x))$ is a field if and only if $f(x)$ is irreducible.
\end{prop}
\begin{prop}
    Let $g(x)$ be a nonconstant element of $F[x]$ and let $g(x)=f_1(x)^{n_1}f_2(x)^{n_2}\cdots f_k(x)^{n_k}$ be its factorization into irreducibles, where the $f_i(x)$ are distinct. Then we have the following isomorphism of rings: $F[x]/(g(x))\cong F[x]/(f_1(x)^{n_1})\times F[x]/(f_2(x)^{n_2})\times\cdots\times F[x]/(f_k(x)^{n_k})$.
\end{prop}
\begin{prop}
    If the polynomial $f(x)$ has roots $\alpha_1,\alpha_2,\ldots,\alpha_k$ in $F$ (not necessarily distinct), then $f(x)$ has $(x-\alpha_1)\cdots(x-\alpha_k)$ as a factor. In particular, a polynomial of degree $n$ in one variable over a field $F$ has at most $n$ roots in $F$, even counted with multiplicity.
\end{prop}
\begin{prop}
    A finite subgroup of the multiplicative group of a field is cyclic. In particular, if $F$ is a finite field, then the multiplicative group $F^\times$ of nonzero elements of $F$ is a cyclic group.
\end{prop}
\begin{corollary}
    Let $p$ be a prime. The multiplicative group $(\mathbb{Z}/p\mathbb{Z})^\times$ of nonzero residue classes modulo $p$ is cyclic.
\end{corollary}
\begin{corollary}
    Let $n\geq2$ be an integer with factorization $n=p_1^{\alpha_1}p_2^{\alpha_2}\cdots p_r^{\alpha_r}$ in $\mathbb{Z}$, where $p_1,\ldots,p_r$ are distinct primes. We have the following isomorphisms of (multiplicative) groups:
    \begin{enumerate}
        \item $(\mathbb{Z}/n\mathbb{Z})^\times\cong(\mathbb{Z}/p_1^{\alpha_1}\mathbb{Z})^\times\times(\mathbb{Z}/p_2^{\alpha_2}\mathbb{Z})^\times\times\cdots\times(\mathbb{Z}/p_r^{\alpha_r}\mathbb{Z})^\times$.
        \item $(\mathbb{Z}/2^{\alpha}\mathbb{Z})^\times$ is the direct product of a cyclic group of order 2 and a cyclic group of order $2^{\alpha-2}$ for all $\alpha\geq2$.
        \item $(\mathbb{Z}/p^{\alpha}\mathbb{Z})^\times$ is a cyclic group of order $p^{\alpha-1}(p-1)$ for all odd primes $p$.
    \end{enumerate}
\end{corollary}
\begin{defn}(Noetherian).
    A commutative ring $R$ with 1 is called \textit{Noetherian} if every ideal of $R$ is finitely generated.
\end{defn}
\begin{theorem}(Hilbert's Basis Theorem).
    If $R$ is a Noetherian ring then so is the polynomial ring $R[x]$.
\end{theorem}
\begin{corollary}
    Every ideal in the polynomial ring $F[x_1,x_2,\ldots,x_n]$ with coefficients from a field $F$ is finitely generated.
\end{corollary}
\begin{defn}(Monomial ordering).
    A \textit{monomial ordering} is a well ordering '$\geq$' on the set of monomials that satisfies $mm_1\geq mm_2$ whenever $m_1\geq m_2$ for monomials $m,m_1,m_2$. Equivalently, a monomial ordering may be specified by defining a well ordering on the $n$-tuples $\alpha=(a_1,\ldots,a_n)\in\mathbb{Z}^n$ of multidegrees of monomials $Ax_1^{a_1}\cdots x_n^{a_n}$ that satisfies $\alpha+\gamma\geq\beta+\gamma$ if $\alpha\geq\beta$.
\end{defn}
\begin{defn}(Leading term, multidegree, ideal of leading terms).
    Fix a monomial ordering on the polynomial ring $F[x_1,x_2,\ldots,x_n]$.
    \begin{enumerate}
        \item The \textit{leading term} of a nonzero polynomial $f$ in $F[x_1,x_2,\ldots,x_n]$, denoted $LT(f)$, is the monomial term of maximal order in $f$ and the leading term of $f=0$ is 0. Define the \textit{multidegree of $f$}, denoted $\partial(f)$, to be the multidegree of the leading term of $f$.
        \item If $I$ is an ideal in $F[x_1,x_2,\ldots,x_n]$, the \textit{ideal of leading terms}, denoted $LT(I)$, is the ideal generated by the leading terms of all the elements in the ideal, i.e., $LT(I)=(LT(f):f\in I)$.
    \end{enumerate}
\end{defn}
\begin{defn}(Gröbner basis).
    A \textit{Gröbner basis} for an ideal $I$ in the polynomial ring $F[x_1,\ldots,x_n]$ is a finite set of generators $\{g_1,\ldots,g_m\}$ for $I$ whose leading terms generate the ideal of all leading terms in $I$, i.e., $I=(g_1,\ldots,g_m)$ and $LT(I)=(LT(g_1),\ldots,LT(g_m))$.
\end{defn}
\begin{theorem}
    Fix a monomial ordering on $R=F[x_1,\ldots,x_n]$ and suppose $\{g_1,\ldots,g_m\}$ is a Gröbner basis for the nonzero ideal $I$ in $R$. Then
    \begin{enumerate}
        \item Every polynomial $f\in R$ can be written uniquely in the form $f=f_I+r$ where $f_I\in I$ and no nonzero monomial term of the 'remainder' $r$ is divisible by any of the leading terms $LT(g_1),\ldots,LT(g_m)$.
        \item Both $f_I$ and $r$ can be computed by general polynomial division by $g_1,\ldots,g_m$ and are independent of the order in which these polynomials are used in the division.
        \item The remainder $r$ provides a unique representative for the coset of $f$ in the quotient ring $F[x_1,\ldots,x_n]/I$. In particular, $f\in I$ if and only if $r=0$.
    \end{enumerate}
\end{theorem}
\begin{prop}
    Fix a monomial ordering on $R=F[x_1,\ldots,x_n]$ and let $I$ be a nonzero ideal in $R$.
    \begin{enumerate}
        \item If $g_1,\ldots,g_m$ are any elements of $I$ such that $LT(I)=(LT(g_1),\ldots,LT(g_m))$, then $\{g_1,\ldots,g_m\}$ is a Gröbner basis for $I$.
        \item The ideal $I$ has a Gröbner basis.
    \end{enumerate}
\end{prop}
\begin{lemma}
    Suppose $f_1,\ldots,f_m\in F[x_1,\ldots,x_n]$ are polynomials with the same multidegree $\alpha$ and that the linear combination $h=a_1f_1+\cdots+a_mf_m$ with constants $a_i\in F$ has strictly smaller multidegree. Then \[h=\sum_{i=2}^nb_iS(f_{i-1},f_i)\] for some constants $b_i\in F$.
\end{lemma}
\begin{prop}(Buchberger's Criterion).
    Let $R=F[x_1,\ldots,x_n]$ and fix a monomial ordering on $R$. If $I=(g_1,\ldots,g_m)$ is a nonzero ideal in $R$, then $G=\{g_1,\ldots,g_m\}$ is a Gröbner basis for $I$ if and only if $S(g_i,g_j)\equiv0\mod G$ for $1\leq i<j\leq m$.
\end{prop}
\begin{defn}(Reduced Gröbner basis).
    Fix a monomial ordering on $R=F[x_1,\ldots,x_n]$. A Gröbner basis $\{g_1,\ldots,g_m\}$ for the nonzero ideal $I$ in $R$ is called a \textit{reduced Gröbner basis} if
    \begin{enumerate}
        \item Each $g_i$ has monic leading term, i.e., $LT(g_i)$ is monic, $i=1,\ldots,m$.
        \item No term in $g_j$ is divisible by $LT(g_i)$ for $j\neq i$.
    \end{enumerate}
\end{defn}
\begin{theorem}
    Fix a monomial ordering on $R=F[x_1,\ldots,x_n]$. Then there is a unique reduced Gröbner basis for every nonzero ideal $I$ in $R$.
\end{theorem}
\begin{corollary}
    Let $I$ and $J$ be two ideals in $F[x_1,\ldots,x_n]$. Then $I=J$ if and only if $I$ and $J$ have the same reduced Gröbner basis with respect to any fixed monomial ordering on $F[x_1,\ldots,x_n]$.
\end{corollary}
\begin{defn}(Elimination ideal).
    If $I$ is an ideal in $F[x_1,\ldots,x_n]$ then $I_i=I\cap F[x_{i+1},\ldots,x_n]$ is called the $i$th \textit{elimination ideal} of $I$ with respect to the ordering $x_1>\cdots>x_n$.
\end{defn}
\begin{prop}(Elimination).
    Suppose $G=\{g_1,\ldots,g_m\}$ is a Gröbner basis for the nonzero ideal $I$ in $F[x_1,\ldots,x_n]$ with respect to the lexicographic monomial ordering $x_1>\cdots>x_n$. Then $G\cap F[x_{i+1},\ldots,x_n]$ is a Gröbner basis of the $i$th elimination ideal $I_i=I\cap F[x_{i+1},\ldots,x_n]$ of $I$. In particular, $I\cap F[x_{i+1},\ldots,x_n]=0$ if and only if $G\cap F[x_{i+1},\ldots,x_n]=\emptyset$.
\end{prop}
\begin{prop}
    If $I$ and $J$ are any two ideals in $F[x_1,\ldots,x_n]$ then $tI+(1-t)J$ is an ideal in $F[t,x_1,\ldots,x_n]$ and $I\cap J=(tI+(1-t)J)\cap F[x_1,\ldots,x_n]$. In particular, $I\cap J$ is the first elimination ideal of $tI+(1-t)J$ with respect to the ordering $t>x_1>\cdots>x_n$.
\end{prop}
\subsection{Modules and Vector Spaces}
\subsubsection{Introduction to Module Theory}
\begin{defn}(Left/right module).
    Let $R$ be a ring (not necessarily commutative nor with 1). A \textit{left $R$-module} or a \textit{left module over $R$} is a set $M$ together with
    \begin{enumerate}
        \item A binary operation $+$ on $M$ under which $M$ is an abelian group.
        \item An action of $R$ on $M$ (that is, a map $R\times M\rightarrow M$) denoted by $rm$, for all $r\in R$ and for all $m\in M$ which satisfies
        \begin{itemize}
            \item $(r+s)m=rm+sm$ for all $r,s\in R,m\in M$.
            \item $(rs)m=r(sm)$ for all $r,s\in R,m\in M$.
            \item $r(m+n)=rm+rn$ for all $r\in R,m,n\in M$.
            \item Additionally, if the ring $R$ has a 1, we impose $1m=m$ for all $m\in M$.
        \end{itemize}
    \end{enumerate}
\end{defn}
\begin{defn}(Submodule).
    Let $R$ be a ring and let $M$ be an $R$-module. An \textit{$R$-submodule} of $M$ is a subgroup $N$ of $M$ which is closed under the action of ring elements, i.e., $rn\in N$ for all $r\in R,n\in N$.
\end{defn}
\begin{prop}(The Submodule Criterion).
    Let $R$ be a ring and let $M$ be an $R$-module. A subset $N$ of $M$ is a submodule of $M$ if and only if
    \begin{enumerate}
        \item $N\neq\emptyset$.
        \item $x+ry\in N$ for all $r\in R$ and for all $x,y\in N$.
    \end{enumerate}
\end{prop}
\begin{defn}(Algebra).
    Let $R$ be a commutative ring with identity. An \textit{$R$-algebra} is a ring $A$ with identity together with a ring homomorphism $f:R\rightarrow A$ mapping $1_R$ to $1_A$ such that the subring $f(R)$ of $A$ is contained in the center of $A$.
\end{defn}
\begin{defn}(Algebra homomorphism).
    If $A$ and $B$ are two $R$-algebras, an \textit{$R$-algebra homomorphism} (or isomorphism) is a ring homomorphism (isomorphism, respectively) $\varphi:A\rightarrow B$ mapping $1_A$ to $1_B$ such that $\varphi(r\cdot a)=r\cdot\varphi(a)$ for all $r\in R$ and $a\in A$.
\end{defn}
\begin{defn}(Module homomorphism).
    Let $R$ be a ring and let $M$ and $N$ be $R$-modules.
    \begin{enumerate}
        \item A map $\varphi:M\rightarrow N$ is an \textit{$R$-module homomorphism} if it respects the $R$-module structures of $M$ and $N$, i.e.,
        \begin{itemize}
            \item $\varphi(x+y)=\varphi(x)+\varphi(y)$ for all $x,y\in M$.
            \item $\varphi(rx)=r\varphi(x)$ for all $r\in R,x\in M$.
        \end{itemize}
        \item An $R$-module homomorphism is an \textit{isomorphism (of $R$-modules)} if it is both injective and surjective. The modules $M$ and $N$ are said to be \textit{isomorphic}, denoted $M\cong N$, if there is some $R$-module isomorphism $\varphi:M\rightarrow N$.
        \item If $\varphi:M\rightarrow N$ is an $R$-module homomorphism, let $\ker{\varphi}=\{m\in M:\varphi(m)=0\}$ (the \textit{kernel} of $\varphi$) and let $\varphi(M)=\{n\in N:n=\varphi(m)\}$ for some $m\in M$ (the \textit{image} of $\varphi$, as usual).
        \item Let $M$ and $N$ be $R$-modules and define $\Hom_R(M,N)$ to be the set of all $R$-module homomorphisms from $M$ into $N$.
    \end{enumerate}
\end{defn}
\begin{prop}
    Let $M$, $N$ and $L$ be $R$-modules.
    \begin{enumerate}
        \item A map $\varphi:M\rightarrow N$ is an $R$-module homomorphism if and only if $\varphi(rx+y)=r\varphi(x)+\varphi(y)$ for all $x,y\in M$ and all $r\in R$.
        \item Let $\varphi,\psi$ be elements of $\Hom_R(M,N)$. Define $\varphi+\psi$ by $(\varphi+\psi)(m)=\varphi(m)+\psi(m)$ for all $m\in M$. Then $\varphi+\psi\in\Hom_R(M,N)$ and with this operation $\Hom_R(M,N)$ is an abelian group. If $R$ is a commutative ring then for $r\in R$ define $r\varphi$ by $(r\varphi)(m)=r(\varphi(m))$ for all $m\in M$. Then $r\varphi\in\Hom_R(M,N)$ and with this action of the commutative ring $R$ the abelian group $\Hom_R(M,N)$ is an $R$-module.
        \item If $\varphi\in\Hom_R(L,M)$ and $\psi\in\Hom_R(M,N)$ then $\psi\circ\varphi\in\Hom_R(L,N)$.
        \item With addition as above and multiplication defined as function composition, $\Hom_R(M,M)$ is a ring with 1. When $R$ is commutative $\Hom_R(M,M)$ is an $R$-algebra.
    \end{enumerate}
\end{prop}
\begin{defn}(Endomorphism ring, endomorphism).
    The ring $\Hom_R(M,M)$ is called the \textit{endomorphism ring of $M$} and will often be denoted by $\End_R(M)$, or just $\End(M)$ when the ring $R$ is clear from the context. Elements of $\End(M)$ are called \textit{endomorphisms}.
\end{defn}
\begin{prop}
    Let $R$ be a ring, let $M$ be an $R$-module and let $N$ be a submodule of $M$. The (additive, abelian) quotient group $M/N$ can be made into an $R$-module by defining an action of elements of $R$ by $r(x+N)=(rx)+N$ for all $r\in R,x+N\in M/N$. The natural projection map $\pi:M\rightarrow M/N$ defined by $\pi(x)=x+N$ is an $R$-module homomorphism with kernel $N$.
\end{prop}
\begin{defn}(Sum of submodules).
    Let $A,B$ be submodules of the $R$-module $M$. The \textit{sum} of $A$ and $B$ is the set $A+B=\{a+b:a\in A,b\in B\}$.
\end{defn}
\begin{theorem}(Isomorphism Theorems for Modules).
    \begin{enumerate}
        \item (The First Isomorphism Theorem for Modules). Let $M,N$ be $R$-modules and let $\varphi:M\rightarrow N$ be an $R$-module homomorphism. Then $\ker{\varphi}$ is a submodule of $M$ and $M/\ker{\varphi}\cong\varphi(M)$.
        \item (The Second Isomorphism Theorem for Modules). Let $A,B$ be submodules of the $R$-module $M$. Then $(A+B)/B\cong A/(A\cap B)$.
        \item (The Third Isomorphism Theorem for Modules). Let $M$ be an $R$-module, and let $A$ and $B$ be submodules of $M$ with $A\subseteq B$. Then $(M/A)/(B/A)\cong M/B$.
        \item (The Fourth or Lattice Isomorphism Theorem for Modules). Let $N$ be a submodule of the $R$-module $M$. There is a bijection between the submodules of $M$ which contain $N$ and the submodules of $M/N$. The correspondence is given by $A\leftrightarrow A/N$, for all $A\supseteq N$. This correspondence commutes with the processes of taking sums and intersections (i.e., is a lattice isomorphism between the lattice of submodules of $M/N$ and the lattice of submodules of $M$ which contain $N$).
    \end{enumerate}
\end{theorem}
\begin{defn}(Sum of submodules, submodule generated by a subset, set of generators/generating set, finitely generated submodule, cyclic submodule).
    Let $M$ be an $R$-module and let $N_1,\ldots,N_n$ be submodules of $M$.
    \begin{enumerate}
        \item The \textit{sum} of $N_1,\ldots,N_n$ is the set of all finite sums of elements from the sets $N_i:\{a_1+a_2+\cdots+a_n:a_i\in N_i\;\forall i\}$. Denote this sum by $N_1+\cdots+N_n$.
        \item For any subset $A$ of $M$ let $RA=\{r_1a_1+r_2a_2+\cdots+r_ma_m:r_1,\ldots,r_m\in R,a_1,\ldots,a_m\in A,m\in\mathbb{Z}^+\}$ (where by convention $RA=\{0\}$ if $A=\emptyset$). If $A$ is the finite set $\{a_1,a_2,\ldots,a_n\}$ we shall write $Ra_1+Ra_2+\cdots+Ra_n$ for $RA$. Call $RA$ the \textit{submodule of $M$ generated by $A$}. If $N$ is a submodule of $M$ (possibly $N=M$) and $N=RA$, for some subset $A$ of $M$, we call $A$ a \textit{set of generators} or \textit{generating set} for $N$, and we say $N$ \textit{is generated by} $A$.
        \item A submodule $N$ of $M$ (possibly $N=M$) is \textit{finitely generated} if there is some finite subset $A$ of $M$ such that $N=RA$, that is, if $N$ is generated by some finite subset.
        \item A submodule $N$ of $M$ (possibly $N=M$) is \textit{cyclic} if there exists an element $a\in M$ such that $N=Ra$, that is, if $N$ is generated by one element: $N=Ra=\{ra:r\in R\}$.
    \end{enumerate}
\end{defn}
\begin{defn}(Direct product of modules).
    Let $M_1,\ldots,M_k$ be a collection of $R$-modules. The collection of $k$-tuples $(m_1,m_2,\ldots,m_k)$ where $m_i\in M_i$ with addition and action of $R$ defined componentwise is called the \textit{direct product} of $M_1,\ldots,M_k$, denoted $M_1\times\cdots\times M_k$.
\end{defn}
\begin{prop}
    Let $N_1,N_2,\ldots,N_k$ be submodules of the $R$-module $M$. Then the following are equivalent:
    \begin{enumerate}
        \item The map $\pi:N_1\times N_2\times\cdots\times N_k\rightarrow N_1+N_2+\cdots+N_k$ defined by $\pi(a_1,a_2,\ldots,a_k)=a_1+a_2+\cdots+a_k$ is an isomorphism (of $R$-modules): $N_1+N_2+\cdots+N_k\cong N_1\times N_2\times\cdots\times N_k$.
        \item $N_j\cap(N_1+N_2+\cdots+N_{j-1}+N_{j+1}+\cdots+N_k)=0$ for all $j\in\{1,2,\ldots,k\}$.
        \item Every $x\in N_1+\cdots+N_k$ can be written \textit{uniquely} in the form $a_1+a_2+\cdots+a_k$ with $a_i\in N_i$.
    \end{enumerate}
\end{prop}
\begin{defn}(Free module, basis/set of free generators).
    An $R$-module $F$ is said to be \textit{free} on the subset $A$ of $F$ if for every nonzero element $x$ of $F$, there exist unique nonzero elements $r_1,r_2,\ldots,r_n$ of $R$ and unique $a_1,a_2,\ldots,a_n$ in $A$ such that $x=r_1a_1+r_2a_2+\cdots+r_na_n$ for some $n\in\mathbb{Z}^+$. In this situation we say $A$ is a \textit{basis} or \textit{set of free generators} for $F$. If $R$ is a commutative ring the cardinality of $A$ is called the \textit{rank} of $F$.
\end{defn}
\begin{theorem}
    For any set $A$ there is a free $R$-module $F(A)$ on the set $A$ and $F(A)$ satisfies the following \textit{universal property}: if $M$ is any $R$-module and $\varphi:A\rightarrow M$ is any map of sets, then there is a unique $R$-module homomorphism $\Phi:F(A)\rightarrow M$ such that $\Phi(a)=\varphi(a)$ for all $a\in A$.
\end{theorem}
\begin{corollary}
    \begin{enumerate}
        \item If $F_1$ and $F_2$ are free modules on the same set $A$, there is a unique isomorphism between $F_1$ and $F_2$ which is the identity map on $A$.
        \item If $F$ is any free $R$-module with basis $A$, then $F\cong F(A)$. In particular, $F$ enjoys the same universal property with respect to $A$ as $F(A)$ does in Theorem 1.253.
    \end{enumerate}
\end{corollary}
\begin{theorem}
    Let $R$ be a subring of $S$, let $N$ be a left $R$-module and let $\iota:N\rightarrow S\otimes_R N$ be the $R$-module homomorphism defined by $\iota(n)=1\otimes n$. Suppose that $L$ is any left $S$-module (hence also an $R$-module) and that $\varphi:N\rightarrow L$ is an $R$-module homomorphism from $N$ to $L$. Then there is a unique $S$-module homomorphism $\Phi:S\otimes_R N\rightarrow L$ such that $\varphi$ factors through $\Phi$, i.e., $\varphi=\Phi\circ\iota$. Conversely, if $\Phi:S\otimes_R N\rightarrow L$ is an $S$-module homomorphism then $\varphi=\Phi\circ\iota$ is an $R$-module homomorphism from $N$ to $L$.
\end{theorem}
\begin{corollary}
    Let $\iota:N\rightarrow S\otimes_R N$ be the $R$-module homomorphism in Theorem 1.255. Then $N/\ker{\iota}$ is the unique largest quotient of $N$ that can be embedded in any $S$-module. In particular, $N$ can be embedded as an $R$-submodule of some left $S$-module if and only if $\iota$ is injective (in which case $N$ is isomorphic to the $R$-submodule $\iota(N)$ of the $S$-module $S\otimes_R N$).
\end{corollary}
\begin{defn}(Balanced/middle linear).
    Let $M$ be a right $R$-module, let $N$ be a left $R$-module and let $L$ be an abelian group (written additively). A map $\varphi:M\times N\rightarrow L$ is called \textit{$R$-balanced} or \textit{middle linear with respect to $R$} if $\varphi(m_1+m_2,n)=\varphi(m_1,n)+\varphi(m_2,n),\varphi(m,n_1+n_2)=\varphi(m,n_1)+\varphi(m,n_2),\varphi(m,rn)=\varphi(mr,n)$ for all $m,m_1,m_2\in M$, $n,n_1,n_2\in N$ and $r\in R$.
\end{defn}
\begin{theorem}
    Suppose $R$ is a ring with 1, $M$ is a right $R$-module, and $N$ is a left $R$-module. Let $M\otimes_R N$ be the tensor product of $M$ and $N$ over $R$ and let $\iota:M\times N\rightarrow M\otimes_R N$ be the $R$-balanced map.
    \begin{enumerate}
        \item If $\Phi:M\otimes_R N\rightarrow L$ is any group homomorphism from $M\otimes_R N$ to an abelian group $L$ then the composite map $\varphi=\Phi\circ\iota$ is an $R$-balanced map from $M\times N$ to $L$.
        \item Conversely, suppose $L$ is an abelian group and $\varphi:M\times N\rightarrow L$ is any $R$-balanced map. Then there is a unique group homomorphism $\Phi:M\otimes_R N\rightarrow L$ such that $\varphi$ factors through $\iota$, i.e., $\varphi=\Phi\circ\iota$ as in (1).
    \end{enumerate}
\end{theorem}
\begin{corollary}
    Suppose $D$ is an abelian group and $\iota':M\times N\rightarrow D$ is an $R$-balanced map such that the image of $\iota'$ generates $D$ as an abelian group, and every $R$-balanced map defined on $M\times N$ factors through $\iota'$ as in Theorem 1.258. Then there is an isomorphism $f:M\otimes_R N\cong D$ of abelian groups with $\iota'=f\circ\iota$.
\end{corollary}
\begin{defn}(Bimodule).
    Let $R$ and $S$ be any rings with 1. An abelian group $M$ is called an \textit{$(S,R)$-bimodule} if $M$ is a left $S$-module, a right $R$-module, and $s(mr)=(sm)r$ for all $s\in S$, $r\in R$ and $m\in M$.
\end{defn}
\begin{defn}(Standard).
    Suppose $M$ is a left (or right) $R$-module over the commutative ring $R$. Then the $(R,R)$-bimodule structure on $M$ defined by letting the left and right $R$-actions coincide, i.e., $mr=rm$ for all $m\in M$ and $r\in R$, will be called the \textit{standard} $R$-module structure on $M$.
\end{defn}
\begin{defn}(Bilinear).
    Let $R$ be a commutative ring with 1 and let $M$, $N$ and $L$ be left $R$-modules. The map $\varphi:M\times N\rightarrow L$ is called \textit{$R$-bilinear} if it is $R$-linear in each factor, i.e., if $\varphi(r_1m_1+r_2m_2,n)=r_1\varphi(m_1,n)+r_2\varphi(m_2,n)$ and $\varphi(m,r_1n_1+r_2n_2)=r_1\varphi(m,n_1)+r_2\varphi(m,n_2)$ for all $m,m_1,m_2\in M$, $n,n_1,n_2\in N$ and $r_1,r_2\in R$.
\end{defn}
\begin{corollary}
    Suppose $R$ is a commutative ring. Let $M$ and $N$ be two left $R$-modules and let $M\otimes_R N$ be the tensor product of $M$ and $N$ over $R$, where $M$ is given the standard $R$-module structure. Then $M\otimes_R N$ is a left $R$-module with $r(m\otimes n)=(rm)\otimes n=(mr)\otimes n=m\otimes(rn)$, and the map $\iota:M\times N\rightarrow M\otimes_R N$ with $\iota(m,n)=m\otimes n$ is an $R$-bilinear map. If $L$ is any left $R$-module then there is a bijection between the $R$-bilinear maps with $\varphi:M\times N\rightarrow L$ and the $R$-module homomorphisms with $\Phi:M\otimes_R N\rightarrow L$.
\end{corollary}
\begin{theorem}(The 'Tensor Product' of Two Homomorphisms).
    Let $M,M'$ be right $R$-modules, let $N,N'$ be left $R$-modules, and suppose $\varphi:M\rightarrow M'$ and $\psi:N\rightarrow N'$ are $R$-module homomorphisms.
    \begin{enumerate}
        \item There is a unique group homomorphism, denoted by $\varphi\otimes\psi$, mapping $M\otimes_R N$ into $M'\otimes_R N'$ such that $(\varphi\otimes\psi)(m\otimes n)=\varphi(m)\otimes\psi(n)$ for all $m\in M$ and $n\in N$.
        \item If $M,M'$ are also $(S,R)$-bimodules for some ring $S$ and $\varphi$ is also an $S$-module homomorphism, then $\varphi\otimes\psi$ is a homomorphism of left $S$-modules. In particular, if $R$ is commutative then $\varphi\otimes\psi$ is always an $R$-module homomorphism for the standard $R$-module structures.
        \item If $\lambda:M'\rightarrow M''$ and $\mu:N'\rightarrow N''$ are $R$-module homomorphisms then $(\lambda\otimes\mu)\circ(\varphi\otimes\psi)=(\lambda\circ\varphi)\otimes(\mu\circ\psi)$.
    \end{enumerate}
\end{theorem}
\begin{theorem}(Associativity of the Tensor Product).
    Suppose $M$ is a right $R$-module, $N$ is an $(R,T)$-bimodule, and $L$ is a left $T$-module. Then there is a unique isomorphism $(M\otimes_R N)\otimes_T L\cong M\otimes_R(N\otimes_T L)$ of abelian groups such that $(m\otimes n)\otimes l\mapsto m\otimes(n\otimes l)$. If $M$ is an $(S,R)$-bimodule, then this is an isomorphism of $S$-modules.
\end{theorem}
\begin{corollary}
    Suppose $R$ is commutative and $M$, $N$ and $L$ are left $R$-modules. Then $(M\otimes N)\otimes L\cong M\otimes(N\otimes L)$ as $R$-modules for the standard $R$-module structures on $M$, $N$ and $L$.
\end{corollary}
\begin{defn}(Multilinear).
    Let $R$ be a commutative ring with 1 and let $M_1,M_2,\ldots,M_n$ and $L$ be $R$-modules with the standard $R$-module structures. A map $\varphi:M_1\times\cdots\times M_n\rightarrow L$ is called \textit{$n$-multilinear over $R$} (or simply \textit{multilinear} if $n$ and $R$ are clear from the context) if it is an $R$-module homomorphism in each component when the other component entries are kept constant, i.e., for each $i$, $\varphi(m_1,\ldots,m_{i-1},rm_i+r'm_i',m_{i+1},\ldots,m_n)=r\varphi(m_1,\ldots,m_i,\ldots,m_n)+r'\varphi(m_1,\ldots,m_i',\ldots,m_n)$ for all $m_i,m_i'\in M_i$ and $r,r'\in R$. When $n=2$ (respectively, 3) one says $\varphi$ is \textit{bilinear} (respectively \textit{trilinear}) rather than 2-multilinear (or 3-multilinear).
\end{defn}
\begin{corollary}
    Let $R$ be a commutative ring and let $M_1,\ldots,M_n,L$ be $R$-modules. Let $M_1\otimes M_2\otimes\cdots\otimes M_n$ denote any bracketing of the tensor product of these modules and let $\iota:M_1\times\cdots\times M_n\rightarrow M_1\otimes\cdots\otimes M_n$ be the map defined by $\iota(m_1,\ldots,m_n)=m_1\otimes\cdots\otimes m_n$. Then
    \begin{enumerate}
        \item For every $R$-module homomorphism $\Phi:M_1\otimes\cdots\otimes M_n\rightarrow L$ the map $\varphi=\Phi\circ\iota$ is $n$-multilinear from $M_1\times\cdots\times M_n$ to $L$.
        \item If $\varphi:M_1\times\cdots\times M_n\rightarrow L$ is an $n$-multilinear map then there is a unique $R$-module homomorphism $\Phi:M_1\otimes\cdots\otimes M_n\rightarrow L$ such that $\varphi=\Phi\circ\iota$. Hence there is a bijection between the $n$-multilinear maps with $\varphi:M_1\times\cdots\times M_n\rightarrow L$ and the $R$-module homomorphisms with $\Phi:M_1\otimes\cdots\otimes M_n\rightarrow L$.
    \end{enumerate}
\end{corollary}
\begin{theorem}(Tensor Products of Direct Sums).
    Let $M,M'$ be right $R$-modules and let $N,N'$ be left $R$-modules. Then there are unique group isomorphisms $(M\otimes M')\otimes_R N\cong(M\otimes_R N)\otimes(M'\otimes_R N),M\otimes_R(N\otimes N')\cong(M\otimes_R N)\otimes(M\otimes_R N')$ such that $(m,m')\otimes n\mapsto(m\otimes n,m'\otimes n)$ and $m\otimes(n,n')\mapsto(m\otimes n,m\otimes n')$ respectively. If $M,M'$ are also $(S,R)$-bimodules, then these are isomorphisms of left $S$-modules. In particular, if $R$ is commutative, these are isomorphisms of $R$-modules.
\end{theorem}
\begin{corollary}(Extension of Scalars for Free Modules).
    The module obtained from the free $R$-module $N\cong R^n$ by extension of scalars from $R$ to $S$ is the free $S$-module $S^n$, i.e., $S\otimes_R R^n\cong S^n$ as left $S$-modules.
\end{corollary}
\begin{corollary}
    Let $R$ be a commutative ring and let $M\cong R^s$ and $N\cong R^t$ be free $R$-modules with bases $m_1,\ldots,m_s$ and $n_1,\ldots,n_t$ respectively. Then $M\otimes_R N$ is a free $R$-module of rank $st$, with basis $m_i\otimes n_j$ for $1\leq i\leq s$ and $1\leq j\leq t$, i.e., $R^s\otimes_R R^t\cong R^{st}$.
\end{corollary}
\begin{prop}
    Suppose $R$ is a commutative ring and $M,N$ are left $R$-modules, considered with the standard $R$-module structures. Then there is a unique $R$-module isomorphism $M\otimes_R N\cong N\otimes_R M$ mapping $m\otimes n$ to $n\otimes m$.
\end{prop}
\begin{prop}
    Let $R$ be a commutative ring and let $A$ and $B$ be $R$-algebras. Then the multiplication $(a\otimes b)(a'\otimes b')=aa'\otimes bb'$ is well defined and makes $A\otimes_R B$ into an $R$-algebra.
\end{prop}
\begin{defn}(Exact, exact sequence).
    \begin{enumerate}
        \item The pair of homomorphisms $X\overset{\alpha}{\rightarrow}Y\overset{\beta}{\rightarrow}Z$ is said to be \textit{exact} (at $Y$) if $\im{\alpha}=\ker{\beta}$.
        \item A sequence $\cdots\rightarrow X_{n-1}\rightarrow X_n\rightarrow X_{n+1}\rightarrow\cdots$ of homomorphisms is said to be an \textit{exact sequence} if it is exact at every $X_n$ between a pair of homomorphisms.
    \end{enumerate}
\end{defn}
\begin{prop}
    Let $A$, $B$ and $C$ be $R$-modules over some ring $R$. Then
    \begin{enumerate}
        \item The sequence $0\rightarrow A\overset{\psi}{\rightarrow}B$ is exact (at $A$) if and only if $\psi$ is injective.
        \item The sequence $B\overset{\varphi}{\rightarrow}C\rightarrow0$ is exact (at $C$) if and only if $\varphi$ is surjective.
    \end{enumerate}
\end{prop}
\begin{corollary}
    The sequence $0\rightarrow A\overset{\psi}{\rightarrow}B\overset{\varphi}{\rightarrow}C\rightarrow0$ is exact if and only if $\psi$ is injective, $\varphi$ is surjective, and $\im{\psi}=\ker{\varphi}$, i.e., $B$ is an extension of $C$ by $A$.
\end{corollary}
\begin{defn}(Short exact sequence).
    The exact sequence $0\rightarrow A\overset{\psi}{\rightarrow}B\overset{\varphi}{\rightarrow}C\rightarrow0$ is called a \textit{short exact sequence}.
\end{defn}
\begin{defn}(Split).
    \begin{enumerate}
        \item Let $R$ be a ring and let $0\rightarrow A\overset{\psi}{\rightarrow}B\overset{\varphi}{\rightarrow}C\rightarrow0$ be a short exact sequence of $R$-modules. The sequence is said to be \textit{split} if there is an $R$-module complement to $\psi(A)$ in $B$. In this case, up to isomorphism, $B=A\otimes C$ (more precisely, $B=\psi(A)\otimes C'$ for some submodule $C'$, and $C'$ is mapped isomorphically onto $C$ by $\varphi$: $\varphi(C')\cong C$).
        \item If $1\rightarrow A\overset{\psi}{\rightarrow}B\overset{\varphi}{\rightarrow}C\rightarrow1$ is a short exact sequence of groups, then the sequence is said to be \textit{split} if there is a subgroup complement to $\psi(A)$ in $B$. In this case, up to isomorphism, $B=A\rtimes C$ (more precisely, $B=\psi(A)\rtimes C'$ for some subgroup $C'$, and $C'$ is mapped isomorphically onto $C$ by $\varphi$: $\varphi(C')\cong C$).
    \end{enumerate}
\end{defn}
\begin{prop}
    The short exact sequence $0\rightarrow A\overset{\psi}{\rightarrow}B\overset{\varphi}{\rightarrow}C\rightarrow0$ of $R$-modules is split if and only if there is an $R$-module homomorphism $\mu:C\rightarrow B$ such that $\varphi\circ\mu$ is the identity map on $C$. Similarly, the short exact sequence $1\rightarrow A\overset{\psi}{\rightarrow}B\overset{\varphi}{\rightarrow}C\rightarrow1$ of groups is split if and only if there is a group homomorphism $\mu:C\rightarrow B$ such that $\varphi\circ\mu$ is the identity map on $C$.
\end{prop}
\begin{defn}(Section, splitting homomorphism).
    With notation as in Proposition 1.279, any set map $\mu:C\rightarrow B$ such that $\varphi\circ\mu$ is the identity is called a \textit{section} of $\varphi$. If $\mu$ is a \textit{homomorphism} as in Proposition 1.279 then $\mu$ is called a \textit{splitting homomorphism} for the sequence.
\end{defn}
\begin{prop}
    Let $0\rightarrow A\overset{\psi}{\rightarrow} B\overset{\varphi}{\rightarrow} C\rightarrow0$ be a short exact sequence of modules (respectively, $1\rightarrow A\overset{\psi}{\rightarrow}B\overset{\varphi}{\rightarrow}C\rightarrow1$ a short exact sequence of groups). Then $B=\psi(A)\otimes C'$ for some submodule $C'$ of $B$ with $\varphi(C')\cong C$ (respectively, $B=\psi(A)\times C'$ for some subgroup $C'$ of $B$ with $\varphi(C')\cong C$) if and only if there is a homomorphism $\lambda:B\rightarrow A$ such that $\lambda\circ\psi$ is the identity map on $A$.
\end{prop}
\begin{prop}
    Let $D$, $L$ and $M$ be $R$-modules and let $\psi:L\rightarrow M$ be an $R$-module homomorphism. Then the map $\psi':\Hom_R(D,L)\rightarrow\Hom_R(D,M),f\mapsto f'=\psi\circ f$ is a homomorphism of abelian groups. If $\psi$ is injective, then $\psi'$ is also injective, i.e., if $0\rightarrow L\overset{\psi}{\rightarrow}M$ is exact, then $0\rightarrow\Hom_R(D,L)\overset{\psi'}{\rightarrow}\Hom_R(D,M)$ is also exact.
\end{prop}
\begin{theorem}
    Let $D,L,M$ and $N$ be $R$-modules. If $0\rightarrow L\overset{\psi}{\rightarrow}M\overset{\varphi}{\rightarrow}N\rightarrow0$ is exact, then the associated sequence $0\rightarrow\Hom_R(D,L)\overset{\psi}{\rightarrow}\Hom_R(D,M)\overset{\varphi}{\rightarrow}\Hom_R(D,N)$ is exact.
\end{theorem}
\begin{prop}
    Let $D$, $L$ and $N$ be $R$-modules. Then
    \begin{enumerate}
        \item $\Hom_R(D,L\otimes N)\cong\Hom_R(D,L)\otimes\Hom_R(D,N)$.
        \item $\Hom_R(L\otimes N,D)\cong\Hom_R(L,D)\otimes\Hom_R(N,D)$.
    \end{enumerate}
\end{prop}
\begin{prop}
    Let $P$ be an $R$-module. Then the following are equivalent:
    \begin{enumerate}
        \item For any $R$-modules $L$, $M$ and $N$, if $0\rightarrow L\overset{\psi}{\rightarrow}M\overset{\varphi}{\rightarrow}N\rightarrow0$ is a short exact sequence, then $0\rightarrow\Hom_R(P,L)\overset{\psi'}{\rightarrow}\Hom_R(P,M)\overset{\varphi'}{\rightarrow}\Hom_R(P,N)\rightarrow0$ is also a short exact sequence.
        \item If $P$ is a quotient of the $R$-module $M$ then $P$ is isomorphic to a direct summand of $M$, i.e., every short exact sequence $0\rightarrow L\rightarrow M\rightarrow P\rightarrow0$ splits.
        \item $P$ is a direct summand of a free $R$-module.
    \end{enumerate}
\end{prop}
\begin{defn}(Projective).
    An $R$-module $P$ is called \textit{projective} if it satisfies any of the equivalent conditions of Proposition 1.285.
\end{defn}
\begin{corollary}
    Free modules are projective. A finitely generated module is projective if and only if it is a direct summand of a finitely generated free module. Every module is a quotient of a projective module.
\end{corollary}
\begin{corollary}
    If $D$ is an $R$-module, then the functor $\Hom_R(D,\_)$ from the category of $R$-modules to the category of abelian groups is left exact. It is exact if and only if $D$ is a projective $R$-module.
\end{corollary}
\begin{theorem}
    Let $D,L,M$ and $N$ be $R$-modules. If $0\rightarrow L\overset{\psi}{\rightarrow}M\overset{\varphi}{\rightarrow}N\rightarrow0$ is exact, then the associated sequence $0\rightarrow\Hom_R(N,D)\overset{\varphi'}{\rightarrow}\Hom_R(M,D)\overset{\psi'}{\rightarrow}\Hom_R(L,D)$
\end{theorem}
\begin{prop}
    Let $Q$ be an $R$-module. Then the following are equivalent:
    \begin{enumerate}
        \item For any $R$-modules $L$, $M$ and $N$, if $0\rightarrow L\overset{\psi}{\rightarrow}M\overset{\varphi}{\rightarrow}N\rightarrow0$ is a short exact sequence, then $0\rightarrow\Hom_R(N,Q)\overset{\varphi'}{\rightarrow}\Hom_R(M,Q)\overset{\psi'}{\rightarrow}\Hom_R(L,Q)\rightarrow0$
        \item If $Q$ is a submodule of the $R$-module $M$ then $Q$ is a direct summand of $M$, i.e., every short exact sequence $0\rightarrow Q\rightarrow M\rightarrow N\rightarrow0$ splits.
    \end{enumerate}
\end{prop}
\begin{defn}(Injective).
    An $R$-module $Q$ is called \textit{injective} if it satisfies any of the equivalent conditions of Proposition 1.290.
\end{defn}
\begin{corollary}
    If $D$ is an $R$-module, then the functor $\Hom_R(\_,D)$ from the category of $R$-modules to the category of abelian groups is left exact. It is exact if and only if $D$ is an injective $R$-module.
\end{corollary}
\begin{prop}
    Let $Q$ be an $R$-module.
    \begin{enumerate}
        \item (Baer's Criterion). The module $Q$ is injective if and only if for every left ideal $I$ of $R$ any $R$-module homomorphism $g:I\rightarrow Q$ can be extended to an $R$-module homomorphism $G:R\rightarrow Q$.
        \item If $R$ is a P.I.D. then $Q$ is injective if and only if $rQ=Q$ for every nonzero $r\in R$. In particular, a $\mathbb{Z}$-module is injective if and only if it is divisible. When $R$ is a P.I.D., quotient modules of injective $R$-modules are again injective.
    \end{enumerate}
\end{prop}
\begin{corollary}
    Every $\mathbb{Z}$-module is a submodule of an injective $\mathbb{Z}$-module.
\end{corollary}
\begin{theorem}
    Let $R$ be a ring with 1 and let $M$ be an $R$-module. Then $M$ is contained in an injective $R$-module.
\end{theorem}
\begin{theorem}
    Suppose that $D$ is a right $R$-module and that $L$, $M$ and $N$ are left $R$-modules. If $0\rightarrow L\overset{\psi}{\rightarrow}M\overset{\varphi}{\rightarrow}N\rightarrow0$ is exact, then the associated sequence of abelian groups $D\otimes_R L\overset{1\otimes\psi}{\rightarrow}D\otimes_R M\overset{1\otimes\varphi}{\rightarrow}D\otimes_R N\rightarrow0$ is exact.
\end{theorem}
\begin{prop}
    Let $A$ be a right $R$-module. Then the following are equivalent:
    \begin{enumerate}
        \item For any left $R$-modules $L$, $M$ and $N$, if $0\rightarrow L\overset{\psi}{\rightarrow}M\overset{\varphi}{\rightarrow}N\rightarrow0$ is a short exact sequence, then $0\rightarrow A\otimes_R L\overset{1\otimes\psi}{\rightarrow}A\otimes_R M\overset{1\otimes\varphi}{\rightarrow}A\otimes_R N\rightarrow0$ is also a short exact sequence.
        \item For any left $R$-modules $L$ and $M$, if $0\rightarrow L\overset{\psi}{\rightarrow}M$ is an exact sequence of left $R$-modules (i.e., $\psi:L\rightarrow M$ is injective) then $0\rightarrow A\otimes_R L\overset{1\otimes\psi}{\rightarrow}A\otimes_R M$ is an exact sequence of abelian groups (i.e., $1\otimes\psi:A\otimes_R L\rightarrow A\otimes_R M$ is injective).
    \end{enumerate}
\end{prop}
\begin{defn}(Flat).
    A right $R$-module $A$ is called \textit{flat} if it satisfies either of the two equivalent conditions of Proposition 1.297.
\end{defn}
\begin{corollary}
    If $D$ is a right $R$-module, then the functor $D\otimes_R\_$ from the category of left $R$-modules to the category of abelian groups is right exact. If $D$ is an $(S,R)$-bimodule (for example when $S=R$ is commutative and $D$ is given the standard $R$-module structure), then $D\otimes_R\_$ is a right exact functor from the category of left $R$-modules to the category of left $S$-modules. The functor is exact if and only if $D$ is a flat $R$-module.
\end{corollary}
\begin{corollary}
    Free modules are flat; more generally, projective modules are flat.
\end{corollary}
\begin{theorem}(Adjoint Associativity).
    Let $R$ and $S$ be rings, let $A$ be a right $R$-module, let $B$ be an $(R,S)$-bimodule and let $C$ be a right $S$-module. Then there is an isomorphism of abelian groups: $\Hom_S(A\otimes_R B,C)\cong\Hom_R(A,\Hom_S(B,C))$ (the homomorphism groups are right module homomorphisms - note that $\Hom_S(B,C)$ has the structure of a right $R$-module). If $R=S$ is commutative this is an isomorphism of $R$-modules with the standard $R$-module structures.
\end{theorem}
\begin{corollary}
    If $R$ is commutative then the tensor product of two projective $R$-modules is projective.
\end{corollary}
\subsubsection{Vector Spaces}
\begin{defn}(Linearly independent, basis).
    \begin{enumerate}
        \item A subset $S$ of $V$ is called a set of \textit{linearly independent} vectors if an equation $\alpha_1v_1+\alpha_2v_2+\cdots+\alpha_nv_n=0$ with $\alpha_1,\alpha_2,\ldots,\alpha_n\in F$ and $v_1,v_2,\ldots,v_n\in S$ implies $\alpha_1=\alpha_2=\cdots=\alpha_n=0$.
        \item A \textit{basis} of a vector space $V$ is an ordered set of linearly independent vectors which span $V$. In particular two bases will be considered different even if one is simply a rearrangement of the other. This is sometimes referred to as an \textit{ordered basis}.
    \end{enumerate}
\end{defn}
\begin{prop}
    Assume the set $\mathcal{A}=\{v_1,v_2,\ldots,v_n\}$ spans the vector space $V$ but no proper subset of $\mathcal{A}$ spans $V$. Then $\mathcal{A}$ is a basis of $V$. In particular, any finitely generated (i.e., finitely spanned) vector space over $F$ is a free $F$-module.
\end{prop}
\begin{corollary}
    Assume the finite set $\mathcal{A}$ spans the vector space $V$. Then $\mathcal{A}$ contains a basis of $V$.
\end{corollary}
\begin{theorem}(A Replacement Theorem).
    Assume $\mathcal{A}=\{a_1,a_2,\ldots,a_n\}$ is a basis for $V$ containing $n$ elements and $\{b_1,b_2,\ldots,b_m\}$ is a set of linearly independent vectors in $V$. Then there is an ordering $a_1,a_2,\ldots,a_n$ such that for each $k\in\{1,2,\ldots,m\}$ the set $\{b_1,b_2,\ldots,b_k,a_{k+1},a_{k+2},\ldots,a_n\}$ is a basis of $V$. In other words, the elements $b_1,b_2,\ldots,b_m$ can be used to successively replace the elements of the basis $\mathcal{A}$, still retaining a basis. In particular, $n\geq m$.
\end{theorem}
\begin{corollary}
    \begin{enumerate}
        \item Suppose $V$ has a finite basis with $n$ elements. Any set of linearly independent vectors has at most $n$ elements. Any spanning set has at least $n$ elements.
        \item If $V$ has some finite basis then any two bases of $V$ have the same cardinality.
    \end{enumerate}
\end{corollary}
\begin{defn}(Dimension, finite dimensional).
    If $V$ is a finitely generated $F$-module (i.e., has a finite basis) the cardinality of any basis is called the \textit{dimension} of $V$ and is denoted by $\dim_FV$, or just $\dim{V}$ when $F$ is clear from the context, and $V$ is said to be \textit{finite dimensional} over $F$. If $V$ is not finitely generated, $V$ is said to be infinite dimensional (written $\dim{V}=\infty$).
\end{defn}
\begin{corollary}(Building-Up Lemma).
    If $A$ is a set of linearly independent vectors in the finite dimensional space $V$ then there exists a basis of $V$ containing $A$.
\end{corollary}
\begin{theorem}
    If $V$ is an $n$-dimensional vector space over $F$, then $V\cong F^n$. In particular, any two finite dimensional vector spaces over $F$ of the same dimension are isomorphic.
\end{theorem}
\begin{theorem}
    Let $V$ be a vector space over $F$ and let $W$ be a subspace of $V$. Then $V/W$ is a vector space with $\dim{V}=\dim{W}+\dim{V/W}$ (where if one side is infinite then both are).
\end{theorem}
\begin{corollary}
    Let $\varphi:V\rightarrow U$ be a linear transformation of vector spaces over $F$. Then $\ker{\varphi}$ is a subspace of $V$, $\varphi(V)$ is a subspace of $U$ and $\dim{V}=\dim{\ker{\varphi}}+\dim{\varphi(V)}$.
\end{corollary}
\begin{corollary}
    Let $\varphi:V\rightarrow W$ be a linear transformation of vector spaces of the same finite dimension. Then the following are equivalent:
    \begin{enumerate}
        \item $\varphi$ is an isomorphism.
        \item $\varphi$ is injective, i.e., $\ker{\varphi}=0$.
        \item $\varphi$ is surjective, i.e., $\varphi(V)=W$.
        \item $\varphi$ sends a basis of $V$ to a basis of $W$.
    \end{enumerate}
\end{corollary}
\begin{defn}(Null space, nullity, rank, nonsingular transformation).
    If $\varphi:V\rightarrow U$ is a linear transformation of vector spaces over $F$, $\ker{\varphi}$ is sometimes called the \textit{null space} of $\varphi$ and the dimension of $\ker{\varphi}$ is called the \textit{nullity} of $\varphi$. The dimension of $\varphi(V)$ is called the \textit{rank} of $\varphi$. If $\ker{\varphi}=0$, the transformation is said to be \textit{nonsingular}.
\end{defn}
\begin{defn}(Represent).
    The $m\times n$ matrix $A=(a_{ij})$ associated to the linear transformation $\varphi$ is said to \textit{represent} the linear transformation $\varphi$ with respect to the bases $\mathcal{B},\mathcal{E}$. Similarly, $\varphi$ is the linear transformation represented by $A$ with respect to the bases $\mathcal{B},\mathcal{E}$.
\end{defn}
\begin{theorem}
    Let $V$ be a vector space over $F$ of dimension $n$ and let $W$ be a vector space over $F$ of dimension $m$, with bases $\mathcal{B},\mathcal{E}$ respectively. Then the map $\Hom_F(V,W)\rightarrow M_{m\times n}(F)$ from the space of linear transformations from $V$ to $W$ to the space of $m\times n$ matrices with coefficients in $F$ defined by $\varphi\mapsto M_{\mathcal{B}}^{\mathcal{E}}(\varphi)$ is a vector space isomorphism. In particular, there is a bijective correspondence between linear transformations and their associated matrices with respect to a fixed choice of bases.
\end{theorem}
\begin{corollary}
    The dimension of $\Hom_F(V,W)$ is $(\dim{V})(\dim{W})$.
\end{corollary}
\begin{defn}(Nonsingular matrix).
    An $m\times n$ matrix $A$ is called \textit{nonsingular} if $Ax=0$ with $x\in F^n$ implies $x=0$.
\end{defn}
\begin{theorem}
    $M_{\mathcal{D}}^{\mathcal{E}}(\varphi\circ\psi)=M_{\mathcal{B}}^{\mathcal{E}}(\varphi)M_{\mathcal{D}}^{\mathcal{B}}(\psi)$, i.e., with respect to a compatible choice of bases, the product of the matrices representing the linear transformations $\varphi$ and $\psi$ is the matrix representing the composite linear transformation $\varphi\circ\psi$.
\end{theorem}
\begin{corollary}
    Matrix multiplication is associative and distributive (whenever the dimensions are such as to make products defined). An $n\times n$ matrix $A$ is nonsingular if and only if it is invertible.
\end{corollary}
\begin{corollary}
    \begin{enumerate}
        \item If $\mathcal{B}$ is a basis of the $n$-dimensional space $V$, the map $\varphi\mapsto M_{\mathcal{B}}^{\mathcal{B}}(\varphi)$ is a ring and a vector space isomorphism of $\Hom_F(V,V)$ onto the space $M_n(F)$ of $n\times n$ matrices with coefficients in $F$.
        \item $GL(V)\cong GL_n(F)$ where $\dim{V}=n$.
    \end{enumerate}
\end{corollary}
\begin{defn}(Row/column rank).
    If $A$ is any $n\times n$ matrix with entries from $F$, the \textit{row rank} (respectively, \textit{column rank}) of $A$ is the maximal number of linearly independent rows (respectively, columns) of $A$ (where the rows or columns of $A$ are considered as vectors in affine $n$-space, $m$-space, respectively).
\end{defn}
\begin{defn}(Similar matrices).
    Two $n\times n$ matrices $A$ and $B$ are said to be \textit{similar} if there is an invertible (i.e., nonsingular) $n\times n$ matrix $P$ such that $P^{-1}AP=B$. Two linear transformations $\varphi$ and $\psi$ from a vector space $V$ to itself are said to be \textit{similar} if there is a nonsingular linear transformation $\xi$ from $V$ to $V$ such that $\xi^{-1}\varphi\xi=\psi$.
\end{defn}
\begin{prop}
    Let $F$ be a subfield of the field $K$. If $W$ is an $m$-dimensional vector space over $F$ with basis $w_1,\ldots,w_m$, then $K\otimes_F W$ is an $m$-dimensional vector space over $K$ with basis $1\otimes w_1,\ldots,1\otimes w_m$.
\end{prop}
\begin{prop}
    Let $V$ and $W$ be finite dimensional vector spaces over the field $F$ with bases $v_1,\ldots,v_n$ and $w_1,\ldots,w_m$ respectively. Then $V\otimes_F W$ is a vector space over $F$ of dimension $nm$ with basis $v_i\otimes w_j$, $1\leq i\leq n$ and $1\leq j\leq m$.
\end{prop}
\begin{defn}(Kronecker/tensor product).
    Let $A=(\alpha_{ij})$ and $B$ be $r\times n$ and $s\times m$ matrices, respectively, with coefficients from any commutative ring. The \textit{Kronecker product} or \textit{tensor product} of $A$ and $B$, denoted by $A\otimes B$, is the $rs\times nm$ matrix consisting of an $r\times n$ block matrix whose $i,j$ block is the $s\times m$ matrix $\alpha_{ij}B$.
\end{defn}
\begin{prop}
    Let $\varphi:V\rightarrow X$ and $\psi:W\rightarrow Y$ be linear transformations of finite dimensional vector spaces. Then the Kronecker product of matrices representing $\varphi$ and $\psi$ is a matrix representation of $\varphi\otimes\psi$.
\end{prop}
\begin{defn}(Dual space, linear functional).
    \begin{enumerate}
        \item For $V$ any vector space over $F$ let $V^*=\Hom_F(V,F)$ be the space of linear transformations from $V$ to $F$, called the \textit{dual space} of $V$. Elements of $V^*$ are called \textit{linear functionals}.
        \item If $\mathcal{B}=\{v_1,v_2,\ldots,v_n\}$ is a basis of the finite dimensional space $V$, define $v_i^*\in V^*$ for each $i\in\{1,2,\ldots,n\}$ by its action on the basis $\mathcal{B}$ for $1\leq j\leq n$: \[v_i^*(v_j)=\begin{cases}1&\textrm{if}\;i=j\\0&\textrm{if}\;i\neq j\end{cases}\]
    \end{enumerate}
\end{defn}
\begin{prop}
    $\{v_1^*,v_2^*,\ldots,v_n^*\}$ is a basis of $V^*$. In particular, if $V$ is finite dimensional then $V^*$ has the same dimension as $V$.
\end{prop}
\begin{defn}(Dual basis).
    The basis $\{v_1^*,v_2^*,\ldots,v_n^*\}$ of $V^*$ is called the \textit{dual basis} to $\{v_1,v_2,\ldots,v_n\}$.
\end{defn}
\begin{defn}(Double/second dual).
    The dual of $V^*$, namely $V^{**}$, is called the \textit{double dual} or \textit{second dual} of $V$.
\end{defn}
\begin{theorem}
    There is a natural injective linear transformation from $V$ to $V^{**}$. If $V$ is finite dimensional then this linear transformation is an isomorphism.
\end{theorem}
\begin{theorem}
    $\varphi^*$ is a linear transformation from $W^*$ to $V^*$ and $M_{\mathcal{E}^*}^{\mathcal{B}^*}(\varphi^*)$ is the transpose of the matrix $M_{\mathcal{B}}^{\mathcal{E}}(\varphi)$ (recall that the transpose of the matrix $(a_{ij})$ is the matrix $(a_{ji})$.
\end{theorem}
\begin{corollary}
    For any matrix $A$, the row rank of $A$ equals the column rank of $A$.
\end{corollary}
\begin{prop}
    Let $\varphi$ be an $n$-multilinear alternating function on $V$. Then
    \begin{enumerate}
        \item $\varphi(v_1,\ldots,v_{i-1},v_{i+1},v_i,v_{i+2},\ldots,v_n)=-\varphi(v_1,v_2,\ldots,v_n)$ for any $i\in\{1,2,\ldots,n-1\}$, i.e., the value of $\varphi$ on an $n$-tuple is negated if two adjacent components are interchanged.
        \item For each $\sigma\in S_n$, $\varphi(v_{\sigma(1)},v_{\sigma(2)},\ldots,v_{\sigma(n)})=\epsilon(\sigma)\varphi(v_1,v_2,\ldots,v_n)$, where $\epsilon(\sigma)$ is the sign of the permutation $\sigma$.
        \item If $v_i=v_j$ for any pair of distinct $i,j\in\{1,2,\ldots,n\}$ then $\varphi(v_1,v_2,\ldots,v_n)=0$.
        \item If $v_i$ is replaced by $v_i+\alpha v_j$ in $(v_1,\ldots,v_n)$ for any $j\neq i$ and any $\alpha\in R$, the value of $\varphi$ on this $n$-tuple is not changed.
    \end{enumerate}
\end{prop}
\begin{prop}
    Assume $\varphi$ is an $n$-multilinear alternating function on $V$ and that for some $v_1,v_2,\ldots,v_n$ and $w_1,w_2,\ldots,w_n\in V$ and some $\alpha_{ij}\in R$ we have $w_1=\alpha_{11}v_1+\alpha_{21}v_2+\cdots+\alpha_{n1}v_n$ up to $w_n=\alpha_{1n}v_1+\alpha_{2n}v_2+\cdots+\alpha_{nn}v_n$ (we have purposely written the indices of the $\alpha_{ij}$ in 'column format'). Then \[\varphi(w_1,w_2,\ldots,w_n)=\sum_{\sigma\in S_n}\epsilon(\sigma)\alpha_{\sigma(1)1}\alpha_{\sigma(2)2}\cdots\alpha_{\sigma(n)n}\varphi(v_1,v_2,\ldots,v_n)\]
\end{prop}
\begin{defn}(Determinant function).
    An $n\times n$ \textit{determinant function} on $R$ is any function $\det:M_{n\times n}(R)\rightarrow R$ that satisfies the following two axioms:
    \begin{enumerate}
        \item $\det$ is an $n$-multilinear alternating form on $R^n(=V)$, where the $n$-tuples are the $n$ columns of the matrices in $M_{n\times n}(R)$.
        \item $\det(I)=1$, where $I$ is the $n\times n$ identity matrix.
    \end{enumerate}
\end{defn}
\begin{theorem}
    There is a unique $n\times n$ determinant function on $R$ and it can be computed for any $n\times n$ matrix $(\alpha_{ij})$ by the formula: \[\det(\alpha_{ij})=\sum_{\sigma\in S_n}\epsilon(\sigma)\alpha_{\sigma(1)1}\alpha_{\sigma(2)2}\cdots\alpha_{\sigma(n)n}\]
\end{theorem}
\begin{corollary}
    The determinant is an $n$-multilinear function of the rows of $M_{n\times n}(R)$ and for any $n\times n$ matrix $A$, $\det{A}=\det(A^t)$, where $A^t$ is the transpose of $A$.
\end{corollary}
\begin{theorem}(Cramer's Rule).
    If $A_1,A_2,\ldots,A_n$ are the columns of an $n\times n$ matrix $A$ and $B=\beta_1A_1+\beta_2A_2+\cdots+\beta_nA_n$ for some $\beta_1,\ldots,\beta_n\in R$, then $\beta_i\det{A}=\det(A_1,\ldots,A_{i-1},B,A_{i+1},\ldots,A_n)$.
\end{theorem}
\begin{corollary}
    If $R$ is an integral domain, then $\det{A}=0$ for $A\in M_n(R)$ if and only if the columns of $A$ are $R$-linearly dependent as elements of the free $R$-module of rank $n$. Also, $\det{A}=0$ if and only if the rows of $A$ are $R$-linearly dependent.
\end{corollary}
\begin{theorem}
    For matrices $A,B\in M_{n\times n}(R)$, $\det{AB}=(\det{A})(\det{B})$.
\end{theorem}
\begin{defn}(Minor, cofactor).
    Let $A=(\alpha_{ij})$ be an $n\times n$ matrix. For each $i,j$, let $A_{ij}$ be the $n-1\times n-1$ matrix obtained from $A$ by deleting its $i$th row and $j$th column (an $n-1\times n-1$ \textit{minor} of $A$). Then $(-1)^{i+j}\det(A_{ij})$ is called the \textit{$ij$ cofactor of $A$}.
\end{defn}
\begin{theorem}(The Cofactor Expansion Formula along the $i$th row).
    If $A=(\alpha_{ij})$ is an $n\times n$ matrix, then for each fixed $i\in\{1,2,\ldots,n\}$ the determinant of $A$ can be computed from the formula $\det{A}=(-1)^{i+1}\alpha_{i1}\det{A_{i1}}+(-1)^{i+2}\alpha_{i2}\det{A_{i2}}+\cdots+(-1)^{i+n}\alpha_{in}\det{A_{in}}$.
\end{theorem}
\begin{theorem}(Cofactor Formula for the Inverse of a Matrix).
    Let $A=(\alpha_{ij})$ be an $n\times n$ matrix and let $B$ be the transpose of its matrix of cofactors, i.e., $B=(\beta_{ij})$, where $\beta_{ij}=(-1)^{i+j}\det{A_{ji}},1\leq i,j\leq n$. Then $AB=BA=(\det{A})I$. Moreover, $\det{A}$ is a unit in $R$ if and only if $A$ is a unit in $M_{n\times n}(R)$; in this case the matrix $\frac{1}{\det{A}}B$ is the inverse of $A$.
\end{theorem}
\begin{theorem}
    If $M$ is any $R$-module over the commutative ring $R$ then
    \begin{enumerate}
        \item $\mathcal{T}(M)$ is an $R$-algebra containing $M$ with multiplication defined by mapping $(m_1\otimes\cdots\otimes m_i)(m_1'\otimes\cdots\otimes m_j')=m_1\otimes\cdots\otimes m_i\otimes m_1'\otimes\cdots\otimes m_j'$ and extended to sums via the distributive laws. With respect to this multiplication $\mathcal{T}^i(M)\mathcal{T}^j(M)\subseteq\mathcal{T}^{i+j}(M)$.
        \item (Universal Property). If $A$ is any $R$-algebra and $\varphi:M\rightarrow A$ is an $R$-module homomorphism, then there is a unique $R$-algebra homomorphism $\Phi:\mathcal{T}(M)\rightarrow A$ such that $\Phi|_M=\varphi$.
    \end{enumerate}
\end{theorem}
\begin{defn}(Tensor algebra).
    The ring $\mathcal{T}(M)$ is called the \textit{tensor algebra} of $M$.
\end{defn}
\begin{prop}
    Let $V$ be a finite dimensional vector space over the field $F$ with basis $\mathcal{B}=\{v_1,\ldots,v_n\}$. Then the $k$-tensors $v_{i_1}\otimes v_{i_2}\otimes\cdots\otimes v_{i_k}$ with $v_{i_j}\in\mathcal{B}$ are a vector space basis of $\mathcal{T}^k(V)$ over $F$ (with the understanding that the basis vector is the element $1\in F$ when $k=0$). In particular, $\dim_F(\mathcal{T}^k(V))=n^k$.
\end{prop}
\begin{defn}(Graded ring, homogeneous, homogeneous).
    \begin{enumerate}
        \item A ring $S$ is called a \textit{graded ring} if it is the direct sum of additive subgroups: $S=S_0\otimes S_1\otimes S_2\otimes\cdots$ such that $S_iS_j\subseteq S_{i+j}$ for all $i,j\geq0$. The elements of $S_k$ are said to be \textit{homogeneous of degree $k$}, and $S_k$ is called the \textit{homogeneous component of $S$ of degree $k$}.
        \item An ideal $I$ of the graded ring $S$ is called a \textit{graded ideal} if $I=\otimes_{k=0}^\infty(I\cap S_k)$.
        \item A ring homomorphism $\varphi:S\rightarrow T$ between two graded rings is called a \textit{homomorphism of graded rings} if it respects the grading structures on $S$ and $T$, i.e., if $\varphi(S_k)\subseteq T_k$ for $k\in\mathbb{N}_0$.
    \end{enumerate}
\end{defn}
\begin{prop}
    Let $S$ be a graded ring, let $I$ be a graded ideal in $S$ and let $I_k=I\cap S_k$ for all $k\geq0$. Then $S/I$ is naturally a graded ring whose homogeneous component of degree $k$ is isomorphic to $S_k/I_k$.
\end{prop}
\begin{defn}(Symmetric algebra).
    The \textit{symmetric algebra} of an $R$-module $M$ is the $R$-algebra obtained by taking the quotient of the tensor algebra $\mathcal{T}(M)$ by the ideal $\mathcal{C}(M)$ generated by all elements of the form $m_1\otimes m_2-m_2\otimes m_1$ for all $m_1,m_2\in M$. The symmetric algebra $\mathcal{T}(M)/\mathcal{C}(M)$ is denoted by $\mathcal{S}(M)$.
\end{defn}
\begin{theorem}
    Let $M$ be an $R$-module over the commutative ring $R$ and let $\mathcal{S}(M)$ be its symmetric algebra.
    \begin{enumerate}
        \item The $k$th symmetric power, $\mathcal{S}^k(M)$ of $M$ is equal to $M\otimes\cdots\otimes M$ ($k$ factors) modulo the submodule generated by all elements of the form $(m_1\otimes m_2\otimes\cdots\otimes m_k)-(m_{\sigma(1)}\otimes m_{\sigma(2)}\otimes\cdots\otimes m_{\sigma(k)})$ for all $m_i\in M$ and all permutations $\sigma$ in the symmetric group $S_k$.
        \item (Universal Property for Symmetric Multilinear Maps). If $\varphi:M\times\cdots\times M\rightarrow N$ is a symmetric $k$-multilinear map over $R$ then there is a unique $R$-module homomorphism $\Phi:\mathcal{S}^k(M)\rightarrow N$ such that $\varphi=\Phi\circ\iota$, where $\iota:M\times\cdots\times M\rightarrow\mathcal{S}^k(M)$ is the map defined by $\iota(m_1,\ldots,m_k)=m_1\otimes\cdots\otimes m_n\mod\mathcal{C}(M)$.
        \item (Universal Property for maps to commutative $R$-algebras). If $A$ is any commutative $R$-algebra and $\varphi:M\rightarrow A$ is an $R$-module homomorphism, then there is a unique $R$-algebra homomorphism $\Phi:\mathcal{S}(M)\rightarrow A$ such that $\Phi|_M=\varphi$.
    \end{enumerate}
\end{theorem}
\begin{corollary}
    Let $V$ be an $n$-dimensional vector space over the field $F$. Then $\mathcal{S}(V)$ is isomorphic as a graded $F$-algebra to the ring of polynomials in $n$ variables over $F$ (i.e., the isomorphism is also a vector space isomorphism from $\mathcal{S}^k(V)$ onto the space of all homogeneous polynomials of degree $k$). In particular, $\dim_F(\mathcal{S}^k(V))=\begin{pmatrix}k+n-1\\n-1\end{pmatrix}$.
\end{corollary}
\begin{defn}(Exterior algebra).
    The \textit{exterior algebra} of an $R$-module $M$ is the $R$-algebra obtained by taking the quotient of the tensor algebra $\mathcal{T}(M)$ by the ideal $\mathcal{A}(M)$ generated by all elements of the form $m\otimes m$ for $m\in M$. The exterior algebra $\mathcal{T}(M)/\mathcal{A}(M)$ is denoted by $\bigwedge(M)$ and the image of $m_1\otimes m_2\otimes\cdots\otimes m_k$ in $\bigwedge(M)$ is denoted by $m_1\wedge m_2\wedge\cdots\wedge m_k$.
\end{defn}
\begin{theorem}
    Let $M$ be an $R$-module over the commutative ring $R$ and let $\bigwedge(M)$ be its exterior algebra.
    \begin{enumerate}
        \item The $k$th exterior power, $\bigwedge^k(M)$ of $M$ is equal to $M\otimes\cdots\otimes M$ ($k$ factors) modulo the submodule generated by all elements of the form $m_1\otimes m_2\otimes\cdots\otimes m_k$ where $m_i=m_j$ for some $i\neq j$. In particular, $m_1\wedge m_2\wedge\cdots\wedge m_k=0$ if $m_i=m_j$ for some $i\neq j$.
        \item (Universal Property for Alternating Multilinear Maps). If $\varphi:M\times\cdots\times M\rightarrow N$ is an alternating $k$-multilinear map then there is a unique $R$-module homomorphism $\Phi:\bigwedge^k(M)\rightarrow N$ such that $\varphi=\Phi\circ\iota$, where $\iota:M\times\cdots\times M\rightarrow\bigwedge^k(M)$ is the map defined by $\iota(m_1,\ldots,m_k)=m_1\wedge\cdots\wedge m_k$.
    \end{enumerate}
\end{theorem}
\begin{corollary}
    Let $V$ be a finite dimensional vector space over the field $F$ with basis $\mathcal{B}=\{v_1,\ldots,v_n\}$. Then the vectors $v_{i_1}\wedge v_{i_2}\wedge\cdots\wedge v_{i_k}$ for $1\leq i_1<i_2<\cdots<i_k\leq n$ are a basis of $\bigwedge^k(V)$, and $\bigwedge^k(V)=0$ when $k>n$ (when $k=0$ the basis vector is the element $1\in F$). In particular, $\dim_F(\bigwedge^k(V))=\begin{pmatrix}n\\k\end{pmatrix}$.
\end{corollary}
\begin{prop}
    If $\varphi$ is an endomorphism on an $n$-dimensional vector space $V$, then $\bigwedge^n(\varphi)(w)=\det(\varphi)w$ for all $w\in\bigwedge^n(V)$.
\end{prop}
\begin{defn}(Symmetric, alternating).
    \begin{enumerate}
        \item An element $z\in\mathcal{T}^k(M)$ is called a \textit{symmetric} $k$-tensor if $\sigma z=z$ for all $\sigma$ in the symmetric group $S_k$.
        \item An element $z\in\mathcal{T}^k(M)$ is called an \textit{alternating} $k$-tensor if $\sigma z=\epsilon(\sigma)z$ for all $\sigma$ in the symmetric group $S_k$, where $\epsilon(\sigma)$ is the sign, $\pm1$, of the permutation $\sigma$.
    \end{enumerate}
\end{defn}
\begin{prop}
    Let $\sigma$ be an element in the symmetric group $S_k$ and let $\epsilon(\sigma)$ be the sign of the permutation $\sigma$. Then
    \begin{enumerate}
        \item For every $w\in\mathcal{S}^k(M)$ we have $\sigma w=w$.
        \item For every $w\in\bigwedge^k(M)$ we have $\sigma w=\epsilon(\sigma)w$.
    \end{enumerate}
\end{prop}
\begin{prop}
    Suppose $k!$ is a unit in the ring $R$ and $M$ is an $R$-module. Then
    \begin{enumerate}
        \item The map $(1/k!)\Sym$ induces an $R$-module isomorphism between the $k$th symmetric power of $M$ and the $R$-submodule of symmetric $k$-tensors: $(1/k!)\Sym:\mathcal{S}^k(M)\cong\{\textrm{symmetric}\;k\textrm{-tensors}\}$.
        \item The map $(1/k!)\Alt$ induces an $R$-module isomorphism between the $k$th exterior power of $M$ and the $R$-submodule of alternating $k$-tensors: $(1/k!)\Alt:\bigwedge^k(M)\cong\{\textrm{alternating}\;k\textrm{-tensors}\}$.
    \end{enumerate}
\end{prop}
\subsubsection{Modules over Principal Ideal Domains}
\begin{defn}(Noetherian module/ascending chain condition on submodules).
    \begin{enumerate}
        \item The left $R$-module $M$ is said to be a \textit{Noetherian $R$-module} or to satisfy the \textit{ascending chain condition on submodules} (or \textit{A.C.C. on submodules}) if there are no infinite increasing chains of submodules, i.e., whenever $M_1\subseteq M_2\subseteq M_3\subseteq\cdots$ is an increasing chain of submodules of $M$, then there is a positive integer $m$ such that for all $k\geq m$, $M_k=M_m$ (so the chain becomes stationary at stage $m$: $M_m=M_{m+1}=M_{m+2}=\ldots$).
        \item The ring $R$ is said to be \textit{Noetherian} if it is Noetherian as a left module over itself, i.e., if there are no infinite increasing chains of left ideals in $R$.
    \end{enumerate}
\end{defn}
\begin{theorem}
    Let $R$ be a ring and let $M$ be a left $R$-module. Then the following are equivalent:
    \begin{enumerate}
        \item $M$ is a Noetherian $R$-module.
        \item Every nonempty set of submodules of $M$ contains a maximal element under inclusion.
        \item Every submodule of $M$ is finitely generated.
    \end{enumerate}
\end{theorem}
\begin{corollary}
    If $R$ is a P.I.D. then every nonempty set of ideals of $R$ has a maximal element and $R$ is a Noetherian ring.
\end{corollary}
\begin{prop}
    Let $R$ be an integral domain and let $M$ be a free $R$-module of rank $n<\infty$. Then any $n+1$ elements of $M$ are $R$-linearly dependent, i.e., for any $y_1,y_2,\ldots,y_{n+1}\in M$ there are elements $r_1,r_2,\ldots,r_{n+1}\in R$, not all zero, such that $r_1y_1+r_2y_2+\ldots+r_{n+1}y_{n+1}=0$.
\end{prop}
\begin{defn}(Rank of a module).
    For any integral domain $R$ the \textit{rank} of an $R$-module $M$ is the maximum number of $R$-linearly independent elements of $M$.
\end{defn}
\begin{theorem}
    Let $R$ be a P.I.D., let $M$ be a free $R$-module of finite rank $n$ and let $N$ be a submodule of $M$. Then
    \begin{enumerate}
        \item $N$ is free of rank $m$, $m\leq n$.
        \item There exists a basis $y_1,y_2,\ldots,y_n$ of $M$ so that $a_1y_1,a_2y_2,\ldots,a_my_m$ is a basis of $N$ where $a_1,a_2,\ldots,a_m$ are nonzero elements of $R$ with the divisibility relations $a_1\mid a_2\mid\cdots\mid a_m$.
    \end{enumerate}
\end{theorem}
\begin{theorem}(Fundamental Theorem, Existence: Invariant Factor Form).
    Let $R$ be a P.I.D. and let $M$ be a finitely generated $R$-module.
    \begin{enumerate}
        \item Then $M$ is isomorphic to the direct sum of finitely many cyclic modules. More precisely, $M\cong R^r\otimes R/(a_1)\otimes R/(a_2)\otimes\cdots\otimes R/(a_m)$ for some integer $r\geq0$ and nonzero elements $a_1,a_2\ldots,a_m$ of $R$ which are not units in $R$ and which satisfy the divisibility relations $a_1\mid a_2\mid\cdots\mid a_m$.
        \item $M$ is torsion free if and only if $M$ is free.
        \item In the decomposition in (1), $\Tor(M)\cong R/(a_1)\otimes R/(a_2)\otimes\cdots\otimes R/(a_m)$. In particular $M$ is a torsion module if and only if $r=0$ and in this case the annihilator of $M$ is the ideal $(a_m)$.
    \end{enumerate}
\end{theorem}
\begin{defn}(Free rank/Betti number of a module, invariant factors).
    The integer $r$ in Theorem 1.367 is called the \textit{free rank} or the \textit{Betti number} of $M$ and the elements $a_1,a_2,\ldots,a_m\in R$ (defined up to multiplication by units in $R$) are called the \textit{invariant factors} of $M$.
\end{defn}
\begin{theorem}(Fundamental Theorem, Existence: Elementary Divisor Form).
    Let $R$ be a P.I.D. and let $M$ be a finitely generated $R$-module. Then $M$ is the direct sum of a finite number of cyclic modules whose annihilators are either $(0)$ or generated by powers of primes in $R$, i.e., $M\cong R^r\otimes R/(p_1^{\alpha_1})\otimes R/(p_2^{\alpha_2})\otimes\cdots\otimes R/(p_t^{\alpha_t})$ where $r\geq0$ is an integer and $p_1^{\alpha_1},\ldots,p_t^{\alpha_t}$ are positive powers of (not necessarily distinct) primes in $R$.
\end{theorem}
\begin{defn}(Elementary divisors).
    Let $R$ be a P.I.D. and let $M$ be a finitely generated $R$-module as in Theorem 1.369. The prime powers $p_1^{\alpha_1},\ldots,p_t^{\alpha_t}$ (defined up to multiplication by units in $R$) are called the \textit{elementary divisors} of $M$.
\end{defn}
\begin{theorem}(The Primary Decomposition Theorem).
    Let $R$ be a P.I.D. and let $M$ be a nonzero torsion $R$-module (not necessarily finitely generated) with nonzero annihilator $a$. Suppose the factorization of $a$ into distinct prime powers in $R$ is $a=up_1^{\alpha_1}p_2^{\alpha_2}\cdots p_n^{\alpha_n}$ and let $N_i=\{x\in M:p_i^{\alpha_i}x=0\},1\leq i\leq n$. Then $N_i$ is a submodule of $M$ with annihilator $p_i^{\alpha_i}$ and is the submodule of $M$ of all elements annihilated by some power of $p_i$. We have $M=N_1\otimes N_2\otimes\cdots\otimes N_n$. If $M$ is finitely generated then each $N_i$ is the direct sum of finitely many cyclic modules whose annihilators are divisors of $p_i^{\alpha_i}$.
\end{theorem}
\begin{defn}(Primary component).
    The submodule $N_i$ in the Primary Decomposition Theorem is called the \textit{$p_i$-primary component} of $M$.
\end{defn}
\begin{lemma}
    Let $R$ be a P.I.D. and let $p$ be a prime in $R$. Let $F$ denote the field $R/(p)$.
    \begin{enumerate}
        \item Let $M=R^r$. Then $M/pM\cong F^r$.
        \item Let $M=R/(a)$ where $a$ is a nonzero element of $R$. Then $M/pM\cong\begin{cases}F&\textrm{if}\;p\;\textrm{divides}\;a\;\textrm{in}\;R\\0&\textrm{if}\;p\;\textrm{does not divide}\;a\;\textrm{in}\;R\end{cases}$.
        \item Let $M=R/(a_1)\otimes R/(a_2)\otimes\cdots\otimes R/(a_k)$ where each $a_i$ is divisible by $p$. Then $M/pM\cong F^k$.
    \end{enumerate}
\end{lemma}
\begin{theorem}(Fundamental Theorem, Uniqueness).
    Let $R$ be a P.I.D.
    \begin{enumerate}
        \item Two finitely generated $R$-modules $M_1$ and $M_2$ are isomorphic if and only if they have the same free rank and the same list of invariant factors.
        \item Two finitely generated $R$-modules $M_1$ and $M_2$ are isomorphic if and only if they have the same free rank and the same list of elementary divisors.
    \end{enumerate}
\end{theorem}
\begin{corollary}
    Let $R$ be a P.I.D. and let $M$ be a finitely generated $R$-module.
    \begin{enumerate}
        \item The elementary divisors of $M$ are the prime power factors of the invariant factors of $M$.
        \item The largest invariant factor of $M$ is the product of the largest of the distinct prime powers among the elementary divisors of $M$, the next largest invariant factor is the product of the largest of the distinct prime powers among the remaining elementary divisors of $M$, and so on.
    \end{enumerate}
\end{corollary}
\begin{defn}(Eigenvalue, eigenvector, eigenspace).
    \begin{enumerate}
        \item An element $\lambda$ of $F$ is called an \textit{eigenvalue} of the linear transformation $T$ if there is a nonzero vector $v\in V$ such that $T(v)=\lambda v$. In this situation $v$ is called an \textit{eigenvector} of $T$ with corresponding eigenvalue $\lambda$.
        \item If $A$ is an $n\times n$ matrix with coefficients in $F$, an element $\lambda$ is called an \textit{eigenvalue} of $A$ with corresponding eigenvector $v$ if $v$ is a nonzero $n\times1$ column vector such that $Av=\lambda v$.
        \item If $\lambda$ is an eigenvalue of the linear transformation $T$, the set $\{v\in V:T(v)=\lambda v\}$ is called the \textit{eigenspace} of $T$ corresponding to the eigenvalue $\lambda$. Similarly, if $\lambda$ is an eigenvalue of the $n\times n$ matrix $A$, the set of $n\times1$ matrices $v$ with $Av=\lambda v$ is called the \textit{eigenspace} of $A$ corresponding to the eigenvalue $\lambda$.
    \end{enumerate}
\end{defn}
\begin{defn}(Determinant of a linear transformation).
    The determinant of a linear transformation from $V$ to $V$ is the determinant of any matrix representing the linear transformation (note that this does not depend on the choice of the basis used).
\end{defn}
\begin{prop}
    The following are equivalent:
    \begin{enumerate}
        \item $\lambda$ is an eigenvalue of $T$.
        \item $\lambda I-T$ is a singular linear transformation of $V$.
        \item $\det(\lambda I-T)=0$.
    \end{enumerate}
\end{prop}
\begin{defn}(Characteristic polynomial).
    Let $x$ be an indeterminate over $F$. The polynomial $\det(xI-T)$ is called the \textit{characteristic polynomial} of $T$ and will be denoted $c_T(x)$. If $A$ is an $n\times n$ matrix with coefficients in $F$, $\det(xI-A)$ is called the \textit{characteristic polynomial} of $A$ and will be denoted $c_A(x)$.
\end{defn}
\begin{defn}(Minimal polynomial).
    The unique monic polynomial which generates the ideal $\Ann(V)$ in $F[x]$ is called the \textit{minimal polynomial} of $T$ and will be denoted $m_T(x)$. The unique monic polynomial of smallest degree which when evaluated at the matrix $A$ is the zero matrix is called the \textit{minimal polynomial} of $A$ and will be denoted $m_A(x)$.
\end{defn}
\begin{prop}
    The minimal polynomial $m_T(x)$ is the largest invariant factor of $V$. All the invariant factors of $V$ divide $m_T(x)$.
\end{prop}
\begin{defn}(Companion matrix).
    Let $a(x)=x^k+b_{k-1}x^{k-1}+\cdots+b_1x+b_0$ be any monic polynomial in $F[x]$. The \textit{companion matrix} of $a(x)$ is the $k\times k$ matrix with 1's down the first subdiagonal, $-b_0,-b_1,\ldots,-b_{k-1}$ down the last column and zeros elsewhere. The companion matrix of $a(x)$ will be denoted by $\mathcal{C}_{a(x)}$.
\end{defn}
\begin{defn}(Rational canonical form, invariant factors, block diagonal).
    \begin{enumerate}
        \item A matrix is said to be in \textit{rational canonical form} if it is the direct sum of companion matrices for monic polynomials $a_1(x),\ldots,a_m(x)$ of degree at least one with $a_1(x)\mid a_2(x)\mid\cdots\mid a_m(x)$. The polynomials $a_i(x)$ are called the \textit{invariant factors} of the matrix. Such a matrix is also said to be a \textit{block diagonal} matrix with blocks the companion matrices for the $a_i(x)$.
        \item A \textit{rational canonical form} for a linear transformation $T$ is a matrix representing $T$ which is in rational canonical form.
    \end{enumerate}
\end{defn}
\begin{theorem}(Rational Canonical Form for Linear Transformations).
    Let $V$ be a finite dimensional vector space over the field $F$ and let $T$ be a linear transformation of $V$.
    \begin{enumerate}
        \item There is a basis for $V$ with respect to which the matrix for $T$ is in rational canonical form, i.e., is a block diagonal matrix whose diagonal blocks are the companion matrices for monic polynomials $a_1(x),a_2(x),\ldots,a_m(x)$ of degree at least one with $a_1(x)\mid a_2(x)\mid\cdots\mid a_m(x)$.
        \item The rational canonical form for $T$ is unique.
    \end{enumerate}
\end{theorem}
\begin{theorem}
    Let $S$ and $T$ be linear transformations of $V$. Then the following are equivalent:
    \begin{enumerate}
        \item $S$ and $T$ are similar linear transformations.
        \item The $F[x]$-modules obtained from $V$ via $S$ and via $T$ are isomorphic $F[x]$-modules.
        \item $S$ and $T$ have the same rational canonical form.
    \end{enumerate}
\end{theorem}
\begin{theorem}(Reduced Canonical Form for Matrices).
    Let $A$ be an $n\times n$ matrix over the field $F$.
    \begin{enumerate}
        \item The matrix $A$ is similar to a matrix in rational canonical form, i.e., there is an invertible $n\times n$ matrix $P$ over $F$ such that $P^{-1}AP$ is a block diagonal matrix whose diagonal blocks are the companion matrices for monic polynomials $a_1(x),a_2(x),\ldots,a_m(x)$ of degree at least one with $a_1(x)\mid a_2(x)\mid\cdots\mid a_m(x)$.
        \item The rational canonical form for $A$ is unique.
    \end{enumerate}
\end{theorem}
\begin{defn}(Invariant factors).
    The \textit{invariant factors} of an $n\times n$ matrix over a field $F$ are the invariant factors of its rational canonical form.
\end{defn}
\begin{theorem}
    Let $A$ and $B$ be $n\times n$ matrices over the field $F$. Then $A$ and $B$ are similar if and only if $A$ and $B$ have the same rational canonical form.
\end{theorem}
\begin{corollary}
    Let $A$ and $B$ be two $n\times n$ matrices over a field $F$ and suppose $F$ is a subfield of the field $K$.
    \begin{enumerate}
        \item The rational canonical form of $A$ is the same whether it is computed over $K$ or over $F$. The minimal and characteristic polynomials and the invariant factors of $A$ are the same whether $A$ is considered as a matrix over $F$ or as a matrix over $K$.
        \item The matrices $A$ and $B$ are similar over $K$ if and only if they are similar over $F$, i.e., there exists an invertible $n\times n$ matrix $P$ with entries from $K$ such that $B=P^{-1}AP$ if and only if there exists an (in general different) invertible $n\times n$ matrix $Q$ with entries from $F$ such that $B=Q^{-1}AQ$.
    \end{enumerate}
\end{corollary}
\begin{lemma}
    Let $a(x)\in F[x]$ be any monic polynomial.
    \begin{enumerate}
        \item The characteristic polynomial of the companion matrix of $a(x)$ is $a(x)$.
        \item If $M$ is the block diagonal matrix \[M=\begin{pmatrix}A_1&0&\ldots&0\\0&A_2&\ldots&0\\\vdots&\vdots&\ddots&\vdots\\0&0&\ldots&A_k\end{pmatrix}\] given by the direct sum of matrices $A_1,A_2,\ldots,A_k$ then the characteristic polynomial of $M$ is the product of the characteristic polynomials of $A_1,A_2,\ldots,A_k$.
    \end{enumerate}
\end{lemma}
\begin{prop}
    Let $A$ be an $n\times n$ matrix over the field $F$.
    \begin{enumerate}
        \item The characteristic polynomial of $A$ is the product of all the invariant factors of $A$.
        \item (The Cayley-Hamilton Theorem). The minimal polynomial of $A$ divides the characteristic polynomial of $A$.
        \item The characteristic polynomial of $A$ divides some power of the minimal polynomial of $A$. In particular these polynomials have the same roots, not counting multiplicities.
    \end{enumerate}
\end{prop}
\begin{theorem}
    Let $A$ be an $n\times n$ matrix over the field $F$. Using the three elementary row and columns operations (interchanging, adding multiples, multiplying - all with respect to any row or column), the $n\times n$ matrix $xI-A$ with entries from $F[x]$ can be put into the diagonal form (called the \textit{Smith Normal Form} for $A$) \[\begin{pmatrix}1&&&&&&\\&\ddots&&&&&\\&&1&&&&\\&&&a_1(x)&&&\\&&&&a_2(x)&&\\&&&&&\ddots&\\&&&&&&a_m(x)\end{pmatrix}\] with monic nonzero elements $a_1(x),a_2(x),\ldots,a_m(x)$ of $F[x]$ with degrees at least one and satisfying $a_1(x)\mid a_2(x)\mid\cdots\mid a_m(x)$. The elements $a_1(x),\ldots,a_m(x)$ are the invariant factors of $A$.
\end{theorem}
\begin{defn}(Elementary Jordan matrix/Jordan block).
    The $k\times k$ matrix with $\lambda$ along the main diagonal and 1 along the first superdiagonal is called the $k\times k$ \textit{elementary Jordon matrix with eigenvalue $\lambda$} or the \textit{Jordan block of size $k$ with eigenvalue $\lambda$}.
\end{defn}
\begin{defn}(Jordan canonical form).
    \begin{enumerate}
        \item A matrix is said to be in \textit{Jordan canonical form} if it is a block diagonal matrix with Jordan blocks along the diagonal.
        \item A \textit{Jordan canonical form} for a linear transformation $T$ is a matrix representing $T$ which is in Jordan canonical form.
    \end{enumerate}
\end{defn}
\begin{theorem}(Jordan Canonical Form for Linear Transformations).
    Let $V$ be a finite dimensional vector space over the field $F$ and let $T$ be a linear transformation of $V$. Assume $F$ contains all the eigenvalues of $T$.
    \begin{enumerate}
        \item There is a basis for $V$ with respect to which the matrix for $T$ is in Jordan canonical form, i.e., is a block diagonal matrix whose diagonal blocks are the Jordan blocks for the elementary divisors of $V$.
        \item The Jordan canonical form for $T$ is unique up to a permutation of the Jordan blocks along the diagonal.
    \end{enumerate}
\end{theorem}
\begin{theorem}(Jordan Canonical Form for Matrices).
    Let $A$ be an $n\times n$ matrix over the field $F$ and assume $F$ contains all the eigenvalues of $A$.
    \begin{enumerate}
        \item The matrix $A$ is similar to a matrix in Jordan canonical form, i.e., there is an invertible $n\times n$ matrix $P$ over $F$ such that $P^{-1}AP$ is a block diagonal matrix whose diagonal blocks are the Jordan blocks for the elementary divisors of $A$.
        \item The Jordan canonical form for $A$ is unique up to a permutation of the Jordan blocks along the diagonal.
    \end{enumerate}
\end{theorem}
\begin{corollary}
    \begin{enumerate}
        \item If a matrix $A$ is similar to a diagonal matrix $D$, then $D$ is the Jordan canonical form of $A$.
        \item Two diagonal matrices are similar if and only if their diagonal entries are the same up to a permutation.
    \end{enumerate}
\end{corollary}
\begin{corollary}
    If $A$ is an $n\times n$ matrix with entries from $F$ and $F$ contains all the eigenvalues of $A$, then $A$ is similar to a diagonal matrix over $F$ if and only if the minimal polynomial of $A$ has no repeated roots.
\end{corollary}
\subsubsection{Field Theory}
\begin{defn}(Characteristic of a field).
    The \textit{characteristic} of a field $F$, denoted $\ch(F)$, is defined to be the smallest positive integer $p$ such that $p\cdot1_F=0$ if such a $p$ exists and is defined to be 0 otherwise.
\end{defn}
\begin{prop}
    The characteristic of a field $F$, $\ch(F)$, is either 0 or a prime $p$. If $\ch(F)=p$ then for any $\alpha\in F$, $p\cdot\alpha=\alpha+\cdots+\alpha=0$ where the sum is over $p$ lots of $\alpha$.
\end{prop}
\begin{defn}(Prime subfield).
    The \textit{prime subfield} of a field $F$ is the subfield of $F$ generated by the multiplicative identity $1_F$ of $F$. It is (isomorphic to) either $\mathbb{Q}$ (if $\ch(F)=0$) or $\mathbb{F}_p$ (if $\ch(F)=p$).
\end{defn}
\begin{defn}(Extension field, base field).
    If $K$ is a field containing the subfield $F$, then $K$ is said to be an \textit{extension field} (or simply an \textit{extension}) of $F$, denoted $K/F$. In particular, every field $F$ is an extension of its prime subfield. The field $F$ is sometimes called the \textit{base field} of the extension.
\end{defn}
\begin{defn}(Degree).
    The \textit{degree} (or \textit{relative degree} or \textit{index}) of a field extension $K/F$, denoted $[K:F]$, is the dimension of $K$ as a vector space over $F$ (i.e., $[K:F]=\dim_FK$). The extension is said to be \textit{finite} if $[K:F]$ is finite and is said to be \textit{infinite} otherwise.
\end{defn}
\begin{prop}
    Let $\varphi:F\rightarrow F'$ be a homomorphism of fields. Then $\varphi$ is either identically 0 or is injective, so that the image of $\varphi$ is either 0 or isomorphic to $F$.
\end{prop}
\begin{theorem}
    Let $F$ be a field and let $p(x)\in F[x]$ be an irreducible polynomial. Then there exists a field $K$ containing an isomorphic copy of $F$ in which $p(x)$ has a root. Identifying $F$ with this isomorphic copy shows that there exists an extension of $F$ in which $p(x)$ has a root.
\end{theorem}
\begin{theorem}
    Let $p(x)\in F[x]$ be an irreducible polynomial of degree $n$ over the field $F$ and let $K$ be the field $F[x]/(p(x))$. Let $\theta\equiv x\mod(p(x))\in K$. Then the elements $1,\theta,\theta^2,\ldots,\theta^{n-1}$ are a basis for $K$ as a vector space over $F$, so the degree of the extension is $n$, i.e., $[K:F]=n$. Hence $K=\{a_0+a_1\theta+a_2\theta^2+\cdots+a_{n-1}\theta^{n-1}:a_0,a_1,\ldots,a_{n-1}\in F\}$ consists of all polynomials of degree less than $n$ in $\theta$.
\end{theorem}
\begin{corollary}
    Let $K$ be as in Theorem 1.405, and let $a(\theta),b(\theta)\in K$ be two polynomials of degree less than $n$ in $\theta$. Then addition in $K$ is defined simply by usual polynomial addition and multiplication in $K$ is defined by $a(\theta)b(\theta)=r(\theta)$ where $r(x)$ is the remainder (of degree less than $n$) obtained after dividing the polynomial $a(x)b(x)$ by $p(x)$ in $F[x]$.
\end{corollary}
\begin{defn}(Generated).
    Let $K$ be an extension of the field $F$ and let $\alpha,\beta,\cdots\in K$ be a collection of elements of $K$. Then the smallest subfield of $K$ containing both $F$ and the elements $\alpha,\beta,\ldots$ denoted $F(\alpha,\beta,\ldots)$ is called the field \textit{generated by $\alpha,\beta,\ldots$ over $F$}.
\end{defn}
\begin{defn}(Simple extension, primitive element).
    If the field $K$ is generated by a single element $\alpha$ over $F,K=F(\alpha)$, then $K$ is said to be a \textit{simple} extension of $F$ and the element $\alpha$ is called a \textit{primitive element} for the extension.
\end{defn}
\begin{theorem}
    Let $F$ be a field and let $p(x)\in F[x]$ be an irreducible polynomial. Suppose $K$ is an extension field of $F$ containing a root $\alpha$ of $p(x)$: $p(\alpha)=0$. Let $F(\alpha)$ denote the subfield of $K$ generated over $F$ by $\alpha$. Then $F(\alpha)\cong F[x]/(p(x))$.
\end{theorem}
\begin{corollary}
    Suppose in Theorem 1.410 that $p(x)$ is of degree $n$. Then $F(\alpha)=\{a_0+a_1\alpha+a_2\alpha^2+\cdots+a_{n-1}\alpha^{n-1}:a_0,a_1,\ldots,a_{n-1}\in F\}\subseteq K$.
\end{corollary}
\begin{theorem}
    Let $\varphi:F\overset{\sim}{\rightarrow}F'$ be an isomorphism of fields. Let $p(x)\in F[x]$ be an irreducible polynomial and let $p'(x)\in F'[x]$ be the irreducible polynomial obtained by applying the map $\varphi$ to the coefficients of $p(x)$. Let $\alpha$ be a root of $p(x)$ (in some extension of $F$) and let $\beta$ be a root of $p'(x)$ (in some extension of $F'$). Then there is an isomorphism $\sigma:F(\alpha)\overset{\sim}{\rightarrow}F'(\beta),\alpha\mapsto\beta$ mapping $\alpha$ to $\beta$ and extending $\varphi$, i.e., such that $\sigma$ restricted to $F$ is the isomorphism $\varphi$.
\end{theorem}
\begin{defn}(Algebraic, transcendental).
    The element $\alpha\in K$ is said to be \textit{algebraic} over $F$ if $\alpha$ is a root of some nonzero polynomial $f(x)\in F[x]$. If $\alpha$ is not algebraic over $F$ (i.e., is not the root of any nonzero polynomial with coefficients in $F$) then $\alpha$ is said to be \textit{transcendental} over $F$. The extension $K/F$ is said to be \textit{algebraic} if every element of $K$ is algebraic over $F$.
\end{defn}
\begin{prop}
    Let $\alpha$ be algebraic over $F$. Then there is a unique monic irreducible polynomial $m_{\alpha,F}(x)\in F[x]$ which has $\alpha$ as a root. A polynomial $f(x)\in F[x]$ has $\alpha$ as a root if and only if $m_{\alpha,F}(x)$ divides $f(x)$ in $F[x]$.
\end{prop}
\begin{corollary}
    If $L/F$ is an extension of fields and $\alpha$ is algebraic over both $F$ and $L$, then $m_{\alpha,L}(x)$ divides $m_{\alpha,F}(x)$ in $L[x]$.
\end{corollary}
\begin{defn}(Minimal polynomial).
    The polynomial $m_{\alpha,F}(x)$ (or just $m_{\alpha}(x)$ if the field $F$ is understood) in Proposition 1.414 is called the \textit{minimal polynomial} for $\alpha$ over $F$. The \textit{degree} of $m_{\alpha}(x)$ is called the \textit{degree} of $\alpha$.
\end{defn}
\begin{prop}
    Let $\alpha$ be algebraic over the field $F$ and let $F(\alpha)$ be the field generated by $\alpha$ over $F$. Then $F(\alpha)\cong F[x]/(m_{\alpha}(x))$ so that in particular $F[(\alpha):F]=\deg{m_{\alpha}(x)}=\deg{\alpha}$, i.e., the degree of $\alpha$ over $F$ is the degree of the extension it generates over $F$.
\end{prop}
\begin{prop}
    The element $\alpha$ is algebraic over $F$ if and only if the simple extension $F(\alpha)/F$ is finite. More precisely, if $\alpha$ is an element of an extension of degree $n$ over $F$ then $\alpha$ satisfies a polynomial of degree at most $n$ over $F$ and if $\alpha$ satisfies a polynomial of degree $n$ over $F$ then the degree of $F(\alpha)$ over $F$ is at most $n$.
\end{prop}
\begin{corollary}
    If the extension $K/F$ is finite, then it is algebraic.
\end{corollary}
\begin{theorem}
    Let $F\subseteq K\subseteq L$ be fields. Then $[L:F]=[L:K][K:F]$.
\end{theorem}
\begin{corollary}
    Suppose $L/F$ is a finite extension and let $K$ be any subfield of $L$ containing $F$, $F\subseteq K\subseteq L$. Then $[K:F]$ divides $[L:F]$.
\end{corollary}
\begin{defn}(Finitely generated extension).
    An extension $K/F$ is \textit{finitely generated} if there are elements $\alpha_1,\alpha_2,\ldots,\alpha_k$ in $K$ such that $K=F(\alpha_1,\alpha_2,\ldots,\alpha_k)$.
\end{defn}
\begin{lemma}
    $F(\alpha,\beta)=(F(\alpha))(\beta)$, i.e., the field generated over $F$ by $\alpha$ and $\beta$ is the field generated by $\beta$ over the field $F(\alpha)$ generated by $\alpha$.
\end{lemma}
\begin{theorem}
    The extension $K/F$ is finite if and only if $K$ is generated by a finite number of algebraic elements over $F$. More precisely, a field generated over $F$ by a finite number of algebraic elements of degrees $n_1,n_2,\ldots,n_k$ is algebraic of degree at most $n_1n_2\cdots n_k$.
\end{theorem}
\begin{corollary}
    Suppose $\alpha$ and $\beta$ are algebraic over $F$. Then $\alpha\pm\beta,\alpha\beta,\alpha/\beta$ (for $\beta\neq0$) are all algebraic (in particular $\alpha^{-1}$ for $\alpha\neq0$).
\end{corollary}
\begin{corollary}
    Let $L/F$ be an arbitrary extension. Then the collection of elements of $L$ that are algebraic over $F$ form a subfield $K$ of $L$.
\end{corollary}
\begin{theorem}
    If $K$ is algebraic over $F$ and $L$ is algebraic over $K$, then $L$ is algebraic over $F$.
\end{theorem}
\begin{defn}(Composite field).
    Let $K_1$ and $K_2$ be two subfields of a field $K$. Then the \textit{composite field} of $K_1$ and $K_2$, denoted $K_1K_2$, is the smallest subfield of $K$ containing both $K_1$ and $K_2$. Similarly, the composite of any collection of subfields of $K$ is the smallest subfield containing all the subfields.
\end{defn}
\begin{prop}
    Let $K_1$ and $K_2$ be two finite extensions of a field $F$ contained in $K$. Then $[K_1K_2:F]\leq[K_1:F][K_2:F]$ with equality if and only if an $F$-basis for one of the fields remains linearly independent over the other field. If $\alpha_1,\alpha_2,\ldots,\alpha_n$ and $\beta_1,\beta_2,\ldots,\beta_m$ are bases for $K_1$ and $K_2$ over $F$ respectively, then the elements $\alpha_i\beta_j$ for $i=1,2,\ldots,n$ and $j=1,2,\ldots,m$ span $K_1K_2$ over $F$.
\end{prop}
\begin{corollary}
    Suppose that $[K_1:F]=n,[K_2:F]=m$ in Proposition 1.429, where $n$ and $m$ are relatively prime: $(n,m)=1$. Then $[K_1K_2:F]=[K_1:F][K_2:F]=nm$.
\end{corollary}
\begin{prop}
    If the element $\alpha\in\mathbb{R}$ is obtained from a field $F\subset\mathbb{R}$ by a series of compass and straightedge constructions then $[F(\alpha):F]=2^k$ for some integer $k\geq0$.
\end{prop}
\begin{theorem}
    None of the classical Greek problems are possible:
    \begin{enumerate}
        \item Doubling the Cube.
        \item Trisecting an Angle.
        \item Squaring the Circle.
    \end{enumerate}
\end{theorem}
\begin{defn}(Splitting field, splits completely).
    The extension field $K$ of $F$ is called a \textit{splitting field} for the polynomial $f(x)\in F[x]$ if $f(x)$ factors completely into linear factors (or \textit{splits completely}) in $K[x]$ and $f(x)$ does not factor completely into linear factors over any proper subfield of $K$ containing $F$.
\end{defn}
\begin{theorem}
    For any field $F$, if $f(x)\in F[x]$ then there exists an extension $K$ of $F$ which is a splitting field for $f(x)$.
\end{theorem}
\begin{defn}(Normal extension).
    If $K$ is an algebraic extension of $F$ which is the splitting field over $F$ for a collection of polynomials $f(x)\in F[x]$ then $K$ is called a \textit{normal} extension of $F$.
\end{defn}
\begin{prop}
    A splitting field of a polynomial of degree $n$ over $F$ is of degree at most $n!$ over $F$.
\end{prop}
\begin{theorem}
    Let $\varphi:F\overset{\sim}{\rightarrow}F'$ be an isomorphism of fields. Let $f(x)\in F[x]$ be a polynomial and let $f'(x)\in F'[x]$ be the polynomial obtained by applying $\varphi$ to the coefficients of $f(x)$. Let $E$ be a splitting field for $f(x)$ over $F$ and let $E'$ be a splitting field for $f'(x)$ over $F'$. Then the isomorphism $\varphi$ extends to an isomorphism $\sigma:E\overset{\sim}{\rightarrow}E'$, i.e., $\sigma$ restricted to $F$ is the isomorphism $\varphi$.
\end{theorem}
\begin{corollary}(Uniqueness of Splitting Fields).
    Any two splitting fields for a polynomial $f(x)\in F[x]$ over a field $F$ are isomorphic.
\end{corollary}
\begin{defn}(Algebraic closure).
    The field $\overline{F}$ is called an \textit{algebraic closure} of $F$ if $\overline{F}$ is algebraic over $F$ and if every polynomial $f(x)\in F[x]$ splits completely over $\overline{F}$ (so that $\overline{F}$ can be said to contain all the elements algebraic over $F$).
\end{defn}
\begin{defn}(Algebraically closed).
    A field $K$ is said to be \textit{algebraically closed} if every polynomial with coefficients in $K$ has a root in $K$.
\end{defn}
\begin{prop}
    Let $\overline{F}$ be an algebraic closure of $F$. Then $\overline{F}$ is algebraically closed.
\end{prop}
\begin{prop}
    For any field $F$ there exists an algebraically closed field $K$ containing $F$.
\end{prop}
\begin{prop}
    Let $K$ be an algebraically closed field and let $F$ be a subfield of $K$. Then the collection of elements $\overline{F}$ of $K$ that are algebraic over $F$ is an algebraic closure of $F$. An algebraic closure of $F$ is unique up to isomorphism.
\end{prop}
\begin{theorem}(Fundamental Theorem of Algebra).
    The field $\mathbb{C}$ is algebraically closed.
\end{theorem}
\begin{corollary}
    The field $\mathbb{C}$ contains an algebraic closure for any of its subfields. In particular, $\overline{\mathbb{Q}}$, the collection of complex numbers algebraic over $\mathbb{Q}$, is an algebraic closure of $\mathbb{Q}$.
\end{corollary}
\begin{defn}(Separable polynomial, inseparable polynomial).
    A polynomial over $F$ is called \textit{separable} if it has no multiple roots (i.e., all its roots are distinct). A polynomial which is not separable is called \textit{inseparable}.
\end{defn}
\begin{defn}(Derivative).
    The \textit{derivative} of the polynomial $f(x)=a_nx^n+a_{n-1}x^{n-1}+\cdots+a_1x+a_0\in F[x]$ is defined to be the polynomial $D_xf(x)=na_nx^{n-1}+(n-1)a_{n-1}x^{n-2}+\cdots+2a_2x+a_1\in F[x]$.
\end{defn}
\begin{prop}
    A polynomial $f(x)$ has a multiple root $\alpha$ if and only if $\alpha$ is also a root of $D_xf(x)$, i.e., $f(x)$ and $D_xf(x)$ are both divisible by the minimal polynomial for $\alpha$. In particular, $f(x)$ is separable if and only if it is relatively prime to its derivative: $(f(x),D_xf(x))=1$.
\end{prop}
\begin{corollary}
    Every irreducible polynomial over a field of characteristic 0 (for example, $\mathbb{Q}$) is separable. A polynomial over such a field is separable if and only if it is the product of distinct irreducible polynomials.
\end{corollary}
\begin{prop}
    Let $F$ be a field of characteristic $p$. Then for any $a,b\in F$, $(a+b)^p=a^p+b^p$ and $(ab)^p=a^pb^p$.
\end{prop}
\begin{defn}(Frobenius endomorphism).
    The map in Proposition 1.450 is called the \textit{Frobenius endomorphism} of $F$.
\end{defn}
\begin{corollary}
    Suppose that $\mathbb{F}$ is a finite field of characteristic $p$. Then every element of $\mathbb{F}$ is a $p$th power in $\mathbb{F}$ (notationally, $\mathbb{F}=\mathbb{F}^p$).
\end{corollary}
\begin{prop}
    Every irreducible polynomial over a finite field $\mathbb{F}$ is separable. A polynomial in $\mathbb{F}[x]$ is separable if and only if it is the product of distinct irreducible polynomials in $\mathbb{F}[x]$.
\end{prop}
\begin{defn}(Perfect).
    A field $K$ of characteristic $p$ is called \textit{perfect} if every element of $K$ is a $p$th power in $K$, i.e., $K=K^p$. Any field of characteristic 0 is also called perfect.
\end{defn}
\begin{prop}
    Let $p(x)$ be an irreducible polynomial over a field $F$ of characteristic $p$. Then there is a unique integer $k\geq0$ and a unique irreducible separable polynomial $p_{\textrm{sep}}(x)\in F[x]$ such that $p(x)=p_{\textrm{sep}}(x^{p^k})$.
\end{prop}
\begin{defn}(Separable degree, inseparable degree).
    Let $p(x)$ be an irreducible polynomial over a field of characteristic $p$. The degree of $p_{\textrm{sep}}(x)$ in Proposition 1.455 is called the \textit{separable degree} of $p(x)$, denoted $\deg_sp(x)$. The integer $p^k$ in Proposition 1.455 is called the \textit{inseparable degree} of $p(x)$, denoted $\deg_ip(x)$.
\end{defn}
\begin{defn}(Separable/separably algebraic field, inseparable field).
    The field $K$ is said to be \textit{separable} (or \textit{separably algebraic}) over $F$ if every element of $K$ is the root of a separable polynomial over $F$ (equivalently, the minimal polynomial over $F$ of every element of $K$ is separable). A field which is not separable is \textit{inseparable}.
\end{defn}
\begin{corollary}
    Every finite extension of a perfect field is separable. In particular, every finite extension of either $\mathbb{Q}$ or a finite field is separable.
\end{corollary}
\begin{defn}(Roots of unity).
    Let $\mu_n$ denote the \textit{group of $n$th roots of unity over $\mathbb{Q}$}.
\end{defn}
\begin{defn}(Cyclotomic polynomial).
    Define the $n$th \textit{cyclotomic polynomial} $\Phi_n(x)$ to be the polynomial whose roots are the primitive $n$th roots of unity: \[\Phi_n(x)=\prod_{\zeta\;\textrm{primitive}\in\mu_n}(x-\zeta)=\prod_{\substack{1\leq a<n\\(a,n)=1}}(x-\zeta_n^a)\]
\end{defn}
\begin{lemma}
    The cyclotomic polynomial $\Phi_n(x)$ is a monic polynomial in $\mathbb{Z}[x]$ of degree $\varphi(n)$.
\end{lemma}
\begin{theorem}
    The cyclotomic polynomial $\Phi_n(x)$ is an irreducible monic polynomial in $\mathbb{Z}[x]$ of degree $\varphi(n)$.
\end{theorem}
\begin{corollary}
    The degree over $\mathbb{Q}$ of the cyclotomic field of $n$th roots of unity is $\varphi(n)$: $[\mathbb{Q}(\zeta_n):\mathbb{Q}]=\varphi(n)$.
\end{corollary}
\subsubsection{Galois Theory}
\begin{defn}(Automorphism of a field, fix).
    \begin{enumerate}
        \item An isomorphism $\sigma$ of $K$ with itself is called an \textit{automorphism} of $K$. The collection of automorphisms of $K$ is denoted $\Aut(K)$. If $\alpha\in K$ we shall write $\sigma\alpha$ for $\sigma(\alpha)$.
        \item An automorphism $\sigma\in\Aut(K)$ is said to \textit{fix} an element $\alpha\in K$ if $\sigma\alpha=\alpha$. If $F$ is a subset of $K$ (for example, a subfield), then an automorphism $\sigma$ is said to \textit{fix} $F$ if it fixes all the elements of $F$, i.e., $\sigma a=a$ for all $a\in F$.
    \end{enumerate}
\end{defn}
\begin{defn}($\Aut(K/F)$).
    Let $K/F$ be an extension of fields. Let $\Aut(K/F)$ be the collection of automorphisms of $K$ which fix $F$.
\end{defn}
\begin{prop}
    $\Aut(K)$ is a group under composition and $\Aut(K/F)$ is a subgroup.
\end{prop}
\begin{prop}
    Let $K/F$ be a field extension and let $\alpha\in K$ be algebraic over $F$. Then for any $\sigma\in\Aut(K/F)$, $\sigma\alpha$ is a root of the minimal polynomial for $\alpha$ over $F$ i.e., $\Aut(K/F)$ permutes the roots of irreducible polynomials. Equivalently, any polynomial with coefficients in $F$ having $\alpha$ as a root also has $\sigma\alpha$ as a root.
\end{prop}
\begin{prop}
    Let $H\leq\Aut(K)$ be a subgroup of the group of automorphisms of $K$. Then the collection $F$ of elements of $K$ fixed by all the elements of $H$ is a subfield of $K$.
\end{prop}
\begin{defn}(Fixed field).
    If $H$ is a subgroup of the group of automorphisms of $K$, the subfield of $K$ fixed by all the elements of $H$ is called the \textit{fixed field} of $H$.
\end{defn}
\begin{prop}
    The association of groups to fields and fields to groups is inclusion reversing, namely
    \begin{enumerate}
        \item If $F_1\subseteq F_2\subseteq K$ are two subfields of $K$ then $\Aut(K/F_2)\leq\Aut(K/F_1)$.
        \item If $H_1\leq H_2\leq\Aut(K)$ are two subgroups of automorphism with associated fixed fields $F_1$ and $F_2$ respectively, then $F_2\subseteq F_1$.
    \end{enumerate}
\end{prop}
\begin{prop}
    Let $E$ be the splitting field over $F$ of the polynomial $f(x)\in F[x]$. Then $|\Aut(E/F)|\leq[E:F]$ with equality if $f(x)$ is separable over $F$.
\end{prop}
\begin{defn}(Galois, Galois extension, Galois group).
    Let $K/F$ be a finite extension. Then $K$ is said to be \textit{Galois} over $F$ and $K/F$ is a \textit{Galois} extension if $|\Aut(K/F)|=[K:F]$. If $K/F$ is Galois the group of automorphisms $\Aut(K/F)$ is called the \textit{Galois group of $K/F$}, denoted $\Gal(K/F)$.
\end{defn}
\begin{corollary}
    If $K$ is the splitting field over $F$ of a separable polynomial $f(x)$ then $K/F$ is Galois.
\end{corollary}
\begin{defn}(Galois group of a separable polynomial over a field).
    If $f(x)$ is a separable polynomial over $F$, then the \textit{Galois group of $f(x)$ over $F$} is the Galois group of the splitting field of $f(x)$ over $F$.
\end{defn}
\begin{defn}(Character).
    A \textit{character} $\chi$ of a group $G$ with values in a field $L$ is a homomorphism from $G$ to the multiplicative group of $L$: $\chi:G\rightarrow L^\times$ i.e., $\chi(g_1g_2)=\chi(g_1)\chi(g_2)$ for all $g_1,g_2\in G$ and $\chi(g)$ is a nonzero element of $L$ for all $g\in G$.
\end{defn}
\begin{defn}(Linearly independent characters).
    The characters $\chi_1,\chi_2,\ldots,\chi_n$ of $G$ are said to be \textit{linearly independent} over $L$ if they are linearly independent as functions on $G$, i.e., if there is no nontrivial relation $a_1\chi_1+a_2\chi_2+\cdots+a_n\chi_n=0$ with $a_1,\ldots,a_n\in L$ not all 0, as a function on $G$ (that is, $a_1\chi_1(g)+a_2\chi_2(g)+\cdots+a_n\chi_n(g)=0$ for all $g\in G$).
\end{defn}
\begin{theorem}(Linear Independence of Characters).
    If $\chi_1,\chi_2,\ldots,\chi_n$ are distinct characters of $G$ with values in $L$ then they are linearly independent over $L$.
\end{theorem}
\begin{corollary}
    If $\sigma_1,\sigma_2,\ldots,\sigma_n$ are distinct embeddings of a field $K$ into a field $L$, then they are linearly independent as functions on $K$. In particular distinct automorphisms of a field $K$ are linearly independent as functions on $K$.
\end{corollary}
\begin{theorem}
    Let $G=\{\sigma_1=1,\sigma_2,\ldots,\sigma_n\}$ be a subgroup of automorphisms of a field $K$ and let $F$ be the fixed field. Then $[K:F]=n=|G|$.
\end{theorem}
\begin{corollary}
    Let $K/F$ be any finite extension. Then $|\Aut(K/F)|\leq[K:F]$ with equality if and only if $F$ is the fixed field of $\Aut(K/F)$. Put another way, $K/F$ is Galois if and only if $F$ is the fixed field of $\Aut(K/F)$.
\end{corollary}
\begin{corollary}
    Let $G$ be a finite subgroup of automorphisms of a field $K$ and let $F$ be the fixed field. Then every automorphism of $K$ fixing $F$ is contained in $G$, i.e., $\Aut(K/F)=G$, so that $K/F$ is Galois, with Galois group $G$.
\end{corollary}
\begin{corollary}
    If $G_1\neq G_2$ are distinct finite subgroups of automorphisms of a field $K$ then their fixed fields are also distinct.
\end{corollary}
\begin{theorem}
    The extension $K/F$ is Galois if and only if $K$ is the splitting field of some separable polynomial over $F$. Furthermore, if this is the case then every irreducible polynomial with coefficients in $F$ which has a root in $K$ is separable and has all its roots in $K$ (so in particular $K/F$ is a separable extension).
\end{theorem}
\begin{defn}(Galois conjugates, conjugate field).
    Let $K/F$ be a Galois extension. If $\alpha\in K$ the elements $\sigma\alpha$ for $\sigma$ in $\Gal(K/F)$ are called the \textit{conjugates} (or \textit{Galois conjugates}) of $\alpha$ over $F$. If $E$ is a subfield of $K$ containing $F$, the field $\sigma(E)$ is called the \textit{conjugate field} of $E$ over $F$.
\end{defn}
\begin{prop}
    Any finite field is isomorphic to $\mathbb{F}_{p^n}$ for some prime $p$ and some integer $n\geq1$. The field $\mathbb{F}_{p^n}$ is the splitting field over $\mathbb{F}_p$ of the polynomial $x^{p^n}-x$, with cyclic Galois group of order $n$ generated by the Frobenius automorphism $\sigma_p$. The subfields of $\mathbb{F}_{p^n}$ are all Galois over $\mathbb{F}_p$ and are in one to one correspondence with the divisors $d$ of $n$. They are the fields $\mathbb{F}_{p^d}$, the fixed fields of $\sigma_p^d$.
\end{prop}
\begin{corollary}
    The irreducible polynomial $x^4+1\in\mathbb{Z}[x]$ is reducible modulo every prime $p$.
\end{corollary}
\begin{prop}
    The finite field $\mathbb{F}_{p^n}$ is simple. In particular, there exists an irreducible polynomial of degree $n$ over $\mathbb{F}_p$ for every $n\geq1$.
\end{prop}
\begin{prop}
    The polynomial $x^{p^n}-x$ is precisely the product of all the distinct irreducible polynomials in $\mathbb{F}_p[x]$ of degree $d$ where $d$ runs through all divisors of $n$.
\end{prop}
\begin{prop}
    Suppose $K/F$ is a Galois extension and $F'/F$ is any extension. Then $KF'/F'$ is a Galois extension, with Galois group $\Gal(KF'/F')\cong\Gal(K/K\cap F')$ isomorphic to a subgroup of $\Gal(K/F)$.
\end{prop}
\begin{corollary}
    Suppose $K/F$ is a Galois extension and $F'/F$ is any finite extension. Then $[KF':F]=\frac{[K:F][F':F]}{[K\cap F':F]}$.
\end{corollary}
\begin{prop}
    Let $K_1$ and $K_2$ be Galois extensions of a field $F$. Then
    \begin{enumerate}
        \item The intersection $K_1\cap K_2$ is Galois over $F$.
        \item The composite $K_1K_2$ is Galois over $F$. The Galois group is isomorphic to the subgroup $H=\{(\sigma,\tau):\sigma|_{K_1\cap K_2}=\tau|_{K_1\cap K_2}\}$ of the direct product $\Gal(K_1/F)\times\Gal(K_2/F)$ consisting of elements whose restrictions to the intersection $K_1\cap K_2$ are equal.
    \end{enumerate}
\end{prop}
\begin{corollary}
    Let $K_1$ and $K_2$ be Galois extensions of a field $F$ with $K_1\cap K_2=F$. Then $\Gal(K_1K_2/F)\cong\Gal(K_1/F)\times\Gal(K_2/F)$. Conversely, if $K$ is Galois over $F$ and $G=\Gal(K/F)=G_1\times G_2$ is the direct product of two subgroups $G_1$ and $G_2$, then $K$ is the composite of two Galois extensions $K_1$ and $K_2$ of $F$ with $K_1\cap K_2=F$.
\end{corollary}
\begin{corollary}
    Let $E/F$ be any finite separable extension. Then $E$ is contained in an extension $K$ which is Galois over $F$ and is minimal in the sense that in a fixed algebraic closure of $K$ any other Galois extension of $F$ containing $E$ contains $K$.
\end{corollary}
\begin{defn}(Galois closure).
    The Galois extension $K$ of $F$ containing $E$ in Corollary 1.493 is called the \textit{Galois closure} of $E$ over $F$.
\end{defn}
\begin{prop}
    Let $K/F$ be a finite extension. Then $K=F(\theta)$ if and only if there exist only finitely many subfields of $K$ containing $F$.
\end{prop}
\begin{theorem}(The Primitive Element Theorem).
    If $K/F$ is finite and separable, then $K/F$ is simple. In particular, any finite extension of fields of characteristic 0 is simple.
\end{theorem}
\begin{theorem}
    The Galois group of the cyclotomic field $\mathbb{Q}(\zeta_n)$ of $n$th roots of unity is isomorphic to the multiplicative group $(\mathbb{Z}/n\mathbb{Z})^\times$. The isomorphism is given explicitly by the map $(\mathbb{Z}/n\mathbb{Z})^\times\overset{\sim}{\rightarrow}\Gal(\mathbb{Q}(\zeta_n)/\mathbb{Q}),a\mod n\mapsto\sigma_a$ where $\sigma_a$ is the automorphism defined by $\sigma_a(\zeta_n)=\zeta_n^a$.
\end{theorem}
\begin{corollary}
    Let $n=p_1^{a_1}p_2^{a_2}\cdots p_k^{a_k}$ be the decomposition of the positive integer $n$ into distinct prime powers. Then the cyclotomic fields $\mathbb{Q}(\zeta_{p_i^{a_i}}),i=1,2,\ldots,k$ intersect only in the field $\mathbb{Q}$ and their composite is the cyclotomic field $\mathbb{Q}(\zeta_n)$. We have $\Gal(\mathbb{Q}(\zeta_n)/\mathbb{Q})\cong\Gal(\mathbb{Q}(\zeta_{p_1^{a_1}})/\mathbb{Q})\times\Gal(\mathbb{Q}(\zeta_{p_2^{a_2}})/\mathbb{Q})\times\cdots\times\Gal(\mathbb{Q}(\zeta_{p_k^{a_k}})/\mathbb{Q})$ which under the isomorphism in Theorem 1.497 is the Chinese Remainder Theorem: $(\mathbb{Z}/n\mathbb{Z})^\times\cong(\mathbb{Z}/p_1^{a_1}\mathbb{Z})^\times\times(\mathbb{Z}/p_2^{a_2}\mathbb{Z})^\times\times\cdots\times(\mathbb{Z}/p_k^{a_k}\mathbb{Z})^\times$.
\end{corollary}
\begin{defn}(Abelian extension).
    The extension $K/F$ is called an \textit{abelian} extension if $K/F$ is Galois and $\Gal(K/F)$ is an abelian group.
\end{defn}
\begin{corollary}
    Let $G$ be any finite abelian group. Then there is a subfield $K$ of a cyclotomic field with $\Gal(K/\mathbb{Q})\cong G$.
\end{corollary}
\begin{theorem}(Kronecker-Weber).
    Let $K$ be a finite abelian extension of $\mathbb{Q}$. Then $K$ is contained in a cyclotomic extension of $\mathbb{Q}$.
\end{theorem}
\begin{prop}
    The regular $n$-gon can be constructed by straightedge and compass if and only if $n=2^kp_1\cdots p_r$ is the product of a power of 2 and distinct Fermat primes.
\end{prop}
\begin{defn}(Elementary symmetric functions).
    Let $x_1,x_2,\ldots,x_n$ be indeterminates. The \textit{elementary symmetric functions} $s_1,s_2,\ldots,s_n$ are defined by $s_1=x_1+x_2+\cdots+x_n,s_2=x_1x_2+x_1x_3+\cdots+x_2x_3+x_2x_4+\cdots+x_{n-1}x_n,\cdots,s_n=x_1x_2\cdots x_n$ i.e., the $i$th symmetric function $s_i$ of $x_1,x_2,\ldots,x_n$ is the sum of all products of the $x_j$'s taken $i$ at a time.
\end{defn}
\begin{defn}(General polynomial).
    The \textit{general polynomial of degree $n$} is the polynomial $(x-x_1)(x-x_2)\cdots(x-x_n)$ whose roots are the indeterminates $x_1,x_2,\ldots,x_n$.
\end{defn}
\begin{prop}
    The fixed field of the symmetric group $S_n$ acting on the field of rational functions in $n$ variables $F(x_1,x_2,\ldots,x_n)$ is the field of rational functions in the elementary symmetric functions $F(s_1,s_2,\ldots,s_n)$.
\end{prop}
\begin{defn}(Symmetric function).
    A rational function $f(x_1,x_2,\ldots,x_n)$ is called \textit{symmetric} if it is not changed by any permutation of the variables $x_1,x_2,\ldots,x_n$.
\end{defn}
\begin{corollary}(Fundamental Theorem on Symmetric Functions).
    Any symmetric function in the variables $x_1,x_2,\ldots,x_n$ is a rational function in the elementary symmetric functions $s_1,s_2,\ldots,s_n$.
\end{corollary}
\begin{theorem}
    The general polynomial $x^n-s_1x^{n-1}+s_2x^{n-2}+\cdots+(-1)^ns_n$ over the field $F(s_1,s_2,\ldots,s_n)$ is separable with Galois group $S_n$.
\end{theorem}
\begin{defn}(Discriminant).
    Define the \textit{discriminant} $D$ of $x_1,x_2,\ldots,x_n$ by the formula \[D=\prod_{i<j}(x_i-x_j)^2\]
\end{defn}
\begin{prop}
    If $\ch(F)\neq2$ then the permutation $\sigma\in S_n$ is an element of $A_n$ if and only if it fixes the square root of the discriminant $D$.
\end{prop}
\begin{prop}
    The Galois group of $f(x)\in F[x]$ is a subgroup of $A_n$ if and only if the discriminant $D\in F$ is the square of an element of $F$.
\end{prop}
\begin{theorem}(Fundamental Theorem of Algebra).
    Every polynomial $f(x)\in\mathbb{C}[x]$ of degree $n$ has precisely $n$ roots in $\mathbb{C}$ (counted with multiplicity). Equivalently, $\mathbb{C}$ is algebraically closed.
\end{theorem}
\begin{defn}(Cyclic extension).
    The extension $K/F$ is said to be \textit{cyclic} if it is Galois with a cyclic Galois group.
\end{defn}
\begin{prop}
    Let $F$ be a field of characteristic not dividing $n$ which contains the $n$th roots of unity. Then the extension $F(\sqrt[n]{a})$ for $a\in F$ is cyclic over $F$ of degree dividing $n$.
\end{prop}
\begin{defn}(Lagrange resolvent).
    For $\alpha\in K$ and any $n$th root of unity $\zeta$, define the \textit{Lagrange resolvent} $(\alpha,\zeta)\in K$ by $(\alpha,\zeta)=\alpha+\zeta\sigma(\alpha)+\zeta^2\sigma^2(\alpha)+\cdots+\zeta^{n-1}\sigma^{n-1}(\alpha)$.
\end{defn}
\begin{prop}
    Any cyclic extension of degree $n$ over a field $F$ of characteristic not dividing $n$ which contains the $n$th roots of unity is of the form $F(\sqrt[n]{a})$ for some $a\in F$.
\end{prop}
\begin{defn}(Expressed by radicals, root extension, solved by radicals).
    \begin{enumerate}
        \item An element $\alpha$ which is algebraic over $F$ can be \textit{expressed by radicals} or \textit{solved for in terms of radicals} if $\alpha$ is an element of a field $K$ which can be obtained by a succession of simple radical extensions $F=K_0\subset K_1\subset\cdots\subset K_i\subset K_{i+1}\subset\cdots\subset K_s=K$ where $K_{i+1}=K_i(\sqrt[n_i]{a_i})$ for some $a_i\in K,i=0,1,\ldots,s-1$. Here $\sqrt[n_i]{a_i}$ denotes some root of the polynomial $x^{n_i}-a_i$. Such a field $K$ will be called a \textit{root extension} of $F$.
        \item A polynomial $f(x)\in F[x]$ can be \textit{solved by radicals} if all its roots can be solved for in terms of radicals.
    \end{enumerate}
\end{defn}
\begin{lemma}
    If $\alpha$ is contained in a root extension $K$ as in Definition 1.517, then $\alpha$ is contained in a root extension which is Galois over $F$ and where each extension $K_{i+1}/K_i$ is cyclic.
\end{lemma}
\begin{theorem}
    The polynomial $f(x)$ can be solved by radicals if and only if its Galois group is a solvable group.
\end{theorem}
\begin{corollary}
    The general equation of degree $n$ cannot be solved by radicals for $n\geq5$.
\end{corollary}
\begin{theorem}
    For any prime $p$ not dividing the discriminant $D$ of $f(x)\in\mathbb{Z}[x]$, the Galois group over $\mathbb{F}_p$ of the reduction $\overline{f}(x)=f(x)\mod p$ is permutation group isomorphic to a subgroup of the Galois group over $\mathbb{Q}$ of $f(x)$.
\end{theorem}
\begin{corollary}
    For any prime $p$ not dividing the discriminant of $f(x)\in\mathbb{Z}[x]$, the Galois group of $f(x)$ over $\mathbb{Q}$ contains an element with cycle decomposition $(n_1,n_2,\ldots,n_k)$ where $n_1,n_2,\ldots,n_k$ are the degrees of the irreducible factors of $f(x)$ reduced modulo $p$.
\end{corollary}
\begin{prop}
    For each $n\in\mathbb{Z}^+$ there exist infinitely many polynomials $f(x)\in\mathbb{Z}[x]$ with $S_n$ as Galois group over $\mathbb{Q}$.
\end{prop}
\begin{theorem}
    The density of primes $p$ for which $f(x)$ splits into type $T$ modulo $p$ is precisely $d_T$.
\end{theorem}
\begin{defn}(Algebraically independent, independent transcendentals, transcendence base).
    \begin{enumerate}
        \item A subset $\{a_1,a_2,\ldots,a_n\}$ of $E$ is called \textit{algebraically independent} over $F$ if there is no nonzero polynomial $f(x_1,x_2,\ldots,x_n)\in F[x_1,x_2,\ldots,x_n]$ such that $f(a_1,a_2,\ldots,a_n)=0$. An arbitrary subset $S$ of $E$ is called \textit{algebraically independent} over $F$ if every finite subset of $S$ is algebraically independent. The elements of $S$ are called \textit{independent transcendentals} over $F$.
        \item A \textit{transcendence base} for $E/F$ is a maximal subset (with respect to inclusion) of $E$ which is algebraically independent over $F$.
    \end{enumerate}
\end{defn}
\begin{theorem}
    The extension $E/F$ has a transcendence base and any two transcendence bases of $E/F$ have the same cardinality.
\end{theorem}
\begin{defn}(Transcendence degree).
    The cardinality of a transcendence base for $E/F$ is called the \textit{transcendence degree} of $E/F$.
\end{defn}
\begin{defn}(Purely transcendental).
    An extension $E/F$ is called \textit{purely transcendental} if it has a transcendence base $S$ such that $E=F(S)$.
\end{defn}
\begin{theorem}
    Let $t$ be transcendental over $F$.
    \begin{enumerate}
        \item (Lüroth). If $F\subseteq K\subseteq F(t)$, then $K=F(r)$ for some $r\in F(t)$. In particular, every nontrivial extension of $F$ contained in $F(t)$ is purely transcendental over $F$.
        \item If $P=P(t),Q=Q(t)$ are nonzero relatively prime polynomials in $F[t]$ which are not both constant, $[F(t):F(P/Q)]=\max(\deg{P},\deg{Q})$.
    \end{enumerate}
\end{theorem}
\begin{theorem}(Hilbert).
    Let $x_1,x_2,\ldots,x_n$ be independent transcendentals over $\mathbb{Q}$, let $E=\mathbb{Q}(x_1,\ldots,x_n)$ and let $G$ be a finite group of automorphisms of $E$ with fixed field $K$. If $K$ is a purely transcendental extension of $\mathbb{Q}$ with transcendence basis $a_1,a_2,\ldots,a_n$, then there are infinitely many specializations of $a_1,\ldots,a_n$ in $\mathbb{Q}$ such that $E$ specializes to a Galois extension of $\mathbb{Q}$ with Galois group isomorphic to $G$.
\end{theorem}
\begin{corollary}
    $S_n$ is a Galois group over $\mathbb{Q}$ for all $n$.
\end{corollary}
\begin{defn}(Purely inseparable).
    An algebraic extension $E/F$ is called \textit{purely inseparable} if for each $\alpha\in E$ the minimal polynomial of $\alpha$ over $F$ has only one distinct root.
\end{defn}
\begin{prop}
    If $E_1$ and $E_2$ are subfields of $E$ which are both separable (or both purely inseparable) extensions of $F$, then their composite $E_1E_2$ is separable (purely inseparable, respectively) over $F$.
\end{prop}
\begin{prop}
    Let $E/F$ be an algebraic extension. Then there is a unique field $E_{\textrm{sep}}$ with $F\subseteq E_{\textrm{sep}}\subseteq E$ such that $E_{\textrm{sep}}$ is separable over $F$ and $E$ is purely inseparable over $E_{\textrm{sep}}$. The field $E_{\textrm{sep}}$ is the set of elements of $E$ which are separable over $F$.
\end{prop}
\begin{corollary}
    Separable degrees (respectively inseparable degrees) are multiplicative.
\end{corollary}
\begin{prop}
    If $E$ is a finitely generated extension of a perfect field $F$, then there is a transcendence base $T$ of $E/F$ such that $E$ is a separable (algebraic) extension of $F(T)$.
\end{prop}
\begin{prop}
    Let $E/F$ be an arbitrary algebraic extension and let $\Omega$ be an algebraic closure of $E$. The following are equivalent:
    \begin{enumerate}
        \item $E/F$ is a normal extension (i.e., the splitting field over $F$ of some set of polynomials in $F[x]$).
        \item Whenever $\sigma:E\rightarrow\Omega$ is an embedding such that $\sigma|_F$ is the identity, $\sigma(E)=E$.
        \item Whenever an irreducible polynomial $f(x)\in F[x]$ has one root in $E$, it has all its roots in $E$.
    \end{enumerate}
\end{prop}
\begin{prop}
    If $E/F$ is normal with $[E:F]_s<\infty$, then $E=E_{\textrm{sep}}E_{\textrm{pi}}$, where $E_{\textrm{pi}}$ is a purely inseparable extension of $F$ ($E_{\textrm{pi}}$ consists of all purely inseparable elements of $E$ over $F$) and $E_{\textrm{sep}}\cap E_{\textrm{pi}}=F$.
\end{prop}
\begin{defn}(Galois extension).
    An extension $E/F$ is called \textit{Galois} if it is algebraic, normal and separable. In this case $\Aut(E/F)$ is called the \textit{Galois group} of the extension and is denoted by $\Gal(E/F)$.
\end{defn}
\begin{theorem}(Krull).
    Let $E/F$ be a Galois extension with Galois group $G$. Topologize $G$ by taking as a base for the closed sets the subgroups of $G$ which are the fixing subgroups of the finite extensions of $F$ in $E$, together with all left and right cosets of these subgroups. Then with this ("Krull") topology the closed subgroups of $G$ correspond bijectively with the subfields of $E$ containing $F$ and the corresponding lattices are dual. Closed normal subgroups of $G$ correspond to normal extensions of $F$ in $E$.
\end{theorem}
\section{Serre: Linear Representations of Finite Groups}
\subsection{Representations and Characters}
\subsubsection{Generalities on Linear Representations}
\begin{theorem}
    Let $\rho:G\rightarrow\GL(V)$ be a linear representation of $G$ in $V$ and let $W$ be a vector subspace of $V$ stable under $G$. Then there exists a complement $W^0$ of $W$ in $V$ which is stable under $G$.
\end{theorem}
\begin{theorem}
    Every representation is a direct sum of irreducible representations.
\end{theorem}
\subsubsection{Character Theory}
\begin{prop}
    If $\chi$ is the character of a representation $\rho$ of degree $n$, we have:
    \begin{enumerate}
        \item $\chi(1)=n$.
        \item $\chi(s^{-1})=\chi(s)^*$ for $s\in G$.
        \item $\chi(tst^{-1})=\chi(s)$ for $s,t\in G$.
    \end{enumerate}
\end{prop}
\begin{prop}
    Let $\rho^1:G\rightarrow\GL(V_1)$ and $\rho^2:G\rightarrow\GL(V_2)$ be two linear representations of $G$, and let $\chi_1$ and $\chi_2$ be their characters. Then:
    \begin{enumerate}
        \item The character $\chi$ of the direct sum representation $V_1\otimes V_2$ is equal to $\chi_1+\chi_2$.
        \item The character $\psi$ of the tensor product representation $V_1\otimes V_2$ is equal to $\chi_1\cdot\chi_2$.
    \end{enumerate}
\end{prop}
\begin{prop}
    Let $\rho:G\rightarrow\GL(V)$ be a linear representation of $G$, and let $\chi$ be its character. Let $\chi_{\sigma}^2$ be the character of the symmetric square $\Sym^2(V)$ of $V$, and let $\chi_{\alpha}^2$ be that of $\Alt^2(V)$. For each $s\in G$, we have $\chi_{\sigma}^2=\frac{1}{2}(\chi(s)^2+\chi(s^2))$, $\chi_{\alpha}^2(s)=\frac{1}{2}(\chi(s)^2-\chi(s^2))$ and $\chi_{\sigma}^2+\chi_{\alpha}^2=\chi^2$.
\end{prop}
\begin{lemma}(Schur's Lemma).
    Let $\rho^1:G\rightarrow\GL(V_1)$ and $\rho^2:G\rightarrow\GL(V_2)$ be two irreducible representations of $G$, and let $f$ be a linear mapping of $V_1$ into $V_2$ such that $\rho_s^2\circ f=f\circ\rho_s^1$ for all $s\in G$. Then:
    \begin{enumerate}
        \item If $\rho^1$ and $\rho^2$ are not isomorphic, we have $f=0$.
        \item If $V_1=V_2$ and $\rho^1=\rho^2$, $f$ is a homothety (i.e., a scalar multiple of the identity).
    \end{enumerate}
\end{lemma}
\begin{corollary}
    Let $h$ be a linear mapping of $V_1$ into $V_2$, and put: \[h_0=\frac{1}{g}\sum_{t\in G}(\rho_t^2)^{-1}h\rho_t^1\] Then:
    \begin{enumerate}
        \item If $\rho^1$ and $\rho^2$ are not isomorphic, we have $h^0=0$.
        \item If $V_1=V_2$ and $\rho^1=\rho^2$, $h^0$ is a homothety of ratio $(1/n)\tr(h)$, with $n=\dim(V_1)$.
    \end{enumerate}
\end{corollary}
\begin{corollary}
    If $\rho^1$ and $\rho^2$ are not isomorphic we have: \[\frac{1}{g}\sum_{t\in G}r_{i_2j_2}(t^{-1})r_{j_1i_1}(t)=0\] for arbitrary $i_1,i_2,j_1,j_2$.
\end{corollary}
\begin{corollary}
    If $V_1=V_2$ and $\rho^1=\rho^2$ we have: \[\frac{1}{g}\sum_{t\in G}r_{i_2j_2}(t^{-1})r_{j_1i_1}(t)=\frac{1}{n}\delta_{i_2i_1}\delta_{j_2j_1}=\begin{cases}1/n&\textrm{if}\;i_1=i_2\;\textrm{and}\;j_1=j_2\\0&\textrm{otherwise}\end{cases}\]
\end{corollary}
\begin{theorem}
    \begin{enumerate}
        \item If $\chi$ is the character of an irreducible representation, we have $(\chi|\chi)=1$ (i.e., $\chi$ is 'of norm 1').
        \item If $\chi$ and $\chi'$ are the characters of two nonisomorphic irreducible representations, we have $(\chi|\chi')=0$ (i.e., $\chi$ and $\chi'$ are orthogonal).
    \end{enumerate}
\end{theorem}
\begin{theorem}
    Let $V$ be a linear representation of $G$ with character $\phi$, and suppose $V$ decomposes into a direct sum of irreducible representations: $V=W_1\otimes\cdots\otimes W_k$. Then, if $W$ is an irreducible representation with character $\chi$, the number of $W_i$ isomorphic to $W$ is equal to the scalar product $(\phi|\chi)=\langle\phi,\chi\rangle$.
\end{theorem}
\begin{corollary}
    The number of $W_i$ isomorphic to $W$ does not depend on the chosen decomposition.
\end{corollary}
\begin{corollary}
    Two representations with the same character are isomorphic.
\end{corollary}
\begin{theorem}
    If $\phi$ is the character of a representation $V$, $(\phi|\phi)$ is a positive integer and we have $(\phi|\phi)=1$ if and only if $V$ is irreducible.
\end{theorem}
\begin{prop}
    The character $r_G$ of the regular representation is given by the formulas: $r_G(s)=\begin{cases}g&s=1\\0&s\neq1\end{cases}$.
\end{prop}
\begin{corollary}
    Every irreducible representation $W_i$ is contained in the regular representation with multiplicity equal to its degree $n_i$.
\end{corollary}
\begin{corollary}
    \begin{enumerate}
        \item The degrees $n_i$ satisfy the relation $\sum_{i=1}^{i=h}n_i^2=g$.
        \item If $s\in G$ is different from 1, we have $\sum_{i=1}^{i=h}n_i\chi_i(s)=0$.
    \end{enumerate}
\end{corollary}
\begin{prop}
    Let $f$ be a class function on $G$ and let $\rho:G\rightarrow\GL(V)$ be a linear representation of $G$. Let $\rho_f$ be the linear mapping of $V$ into itself defined by: \[\rho_f=\sum_{t\in G}f(t)\rho_t\] If $V$ is irreducible of degree $n$ and character $\chi$, then $\rho_f$ is a homothety of ratio $\lambda$ given by: \[\lambda=\frac{1}{n}\sum_{t\in G}f(t)\chi(t)=\frac{g}{n}(f|\chi^*)\]
\end{prop}
\begin{theorem}
    The characters $\chi_1,\ldots,\chi_h$ form an orthonormal basis of $H$.
\end{theorem}
\begin{theorem}
    The number of irreducible representations of $G$ (up to isomorphism) is equal to the number of classes of $G$.
\end{theorem}
\begin{prop}
    Let $s\in G$, and let $c(s)$ be the number of elements in the conjugacy class of $s$.
    \begin{enumerate}
        \item We have $\sum_{i=1}^{i=h}\chi_i(s)^*\chi_i(s)=g/c(s)$.
        \item For $t\in G$ not conjugate to $s$, we have $\sum_{i=1}^{i=h}\chi_i(s)^*\chi_i(t)=0$.
    \end{enumerate}
\end{prop}
\begin{theorem}
    \begin{enumerate}
        \item The decomposition $V=V_1\otimes\cdots\otimes V_h$ does not depend on the initially chosen decomposition of $V$ into irreducible representations.
        \item The projection $p_i$ of $V$ onto $V_i$ associated with this decomposition is given by the formula: \[\rho_i=\frac{n_i}{g}\sum_{t\in G}\chi_i(t)^*\rho_t\]
    \end{enumerate}
\end{theorem}
\begin{prop}
    \begin{enumerate}
        \item The map $p_{\alpha\alpha}$ is a projection; it is zero on the $V_j,j\neq i$. Its image $V_{i,\alpha}$ is contained in $V_i$, and $V_i$ is the direct sum of the $V_{i,\alpha}$ for $1\leq\alpha\leq n$. We have $p_i=\sum_{\alpha}p_{\alpha\alpha}$.
        \item The linear map $p_{\alpha\beta}$ is zero on the $V_j,j\neq i$, as well as on the $V_{i,\gamma}$ for $\gamma\neq\beta$; it defines an isomorphism from $V_{i,\beta}$ onto $V_{i,\alpha}$.
        \item Let $x_1$ be a nonzero element of $V_{i,1}$ and let $x_{\alpha}=p_{\alpha1}(x_1)\in V_{i,\alpha}$. The $x_{\alpha}$ are linearly independent and generate a vector subspace $W(x_1)$ stable under $G$ and of dimension $n$. For each $s\in G$, we have $\rho_s(x_{\alpha})=\sum_{\beta}r_{\beta\alpha}(s)x_{\beta}$ (in particular, $W(x_1)$ is isomorphic to $W_i$).
        \item If $(x_1^{(1)},\ldots,x_1^{(m)})$ is a basis of $V_{i,1}$, the representation $V_i$ is the direct sum of the subrepresentations $W(x_1^{(1)}),\ldots,W(x_1^{(m)})$ defined in (3).
    \end{enumerate}
\end{prop}
\subsubsection{Subgroups, Products, Induced Representations}
\begin{prop}
    The following properties are equivalent:
    \begin{enumerate}
        \item $G$ is abelian.
        \item All the irreducible representations of $G$ have degree 1.
    \end{enumerate}
\end{prop}
\begin{corollary}
    Let $A$ be an abelian subgroup of $G$, let $a$ be its order and let $g$ be that of $G$. Each irreducible representation of $G$ has degree at most $g/a$.
\end{corollary}
\begin{theorem}
    \begin{enumerate}
        \item If $\rho^1$ and $\rho^2$ are irreducible, $\rho^1\otimes\rho^2$ is an irreducible representation of $G_1\times G_2$.
        \item Each irreducible representation of $G_1\times G_2$ is isomorphic to a representation $\rho^1\otimes\rho^2$, where $\rho^i$ is an irreducible representation of $G_i$ for $i=1,2$.
    \end{enumerate}
\end{theorem}
\begin{defn}(Induced representation).
    We say that the representation $\rho$ of $G$ in $V$ is \textit{induced} by the representation $\theta$ of $H$ in $W$ if $V$ is equal to the sum of the $W_{\sigma}$ ($\sigma\in G/H$) and if this sum is direct (that is, if $V=\otimes_{\sigma\in G/H}W_{\sigma}$).
\end{defn}
\begin{lemma}
    Suppose that $(V,\rho)$ is induced by $(W,\theta)$. Let $\rho':G\rightarrow\GL(V')$ be a linear representation of $G$, and let $f:W\rightarrow V'$ be a linear map such that $f(\theta_tw)=\rho_t'f(w)$ for all $t\in H$ and $w\in W$. Then there exists a unique linear map $F:V\rightarrow V'$ which extends $f$ and satisfies $F\circ\rho_s=\rho_s'\circ F$ for all $s\in G$.
\end{lemma}
\begin{theorem}
    Let $(W,\theta)$ be a linear representation of $H$. There exists a linear representation $(V,\rho)$ of $G$ which is induced by $(W,\theta)$, and it is unique up to isomorphism.
\end{theorem}
\begin{theorem}
    Let $h$ be the order of $H$ and let $R$ be a system of representatives of $G/H$. For each $u\in G$, we have \[\chi_{\rho}(u)=\sum_{\substack{r\in R\\r^{-1}ur\in H}}\chi_{\theta}(r^{-1}ur)=\frac{1}{h}\sum_{\substack{s\in G\\s^{-1}us\in H}}\chi_{\theta}(s^{-1}us)\]
\end{theorem}
\subsection{Representations in Characteristic Zero}
\subsubsection{The Group Algebra}
\begin{prop}
    If $K$ is a field of characteristic zero, the algebra $K[G]$ is semisimple.
\end{prop}
\begin{corollary}
    The algebra $K[G]$ is a product of matrix algebras over skew fields of finite degree over $K$.
\end{corollary}
\begin{prop}
    The homomorphism $\tilde{\rho}$ defined by \[\tilde{\rho}:\mathbb{C}[G]\rightarrow\prod_{i=1}^{i=h}\End(W_i)\simeq\prod_{i=1}^{i=h}M_{n_i}(\mathbb{C})\] is an isomorphism.
\end{prop}
\begin{prop}(Fourier inversion formula).
    Let $(u_i)_{1\leq i\leq h}$ be an element of $\prod\End(W_i)$, and let $u=\sum_{s\in G}u(s)s$ be the element of $\mathbb{C}[G]$ such that $\tilde{\rho}_i(u)=u_i$ for all $i$. The $s$th coefficient $u(s)$ of $u$ is given by the formula \[u(s)=\frac{1}{g}\sum_{i=1}^{i=h}n_i\tr_{W_i}(\rho_i(s^{-1})u_i)\] where $n_i=\dim(W_i)$.
\end{prop}
\begin{prop}
    The homomorphism $\tilde{\rho}_i$ maps the center of $\mathbb{C}[G]$ into the set of homotheties of $W_i$ and defines an algebra homomorphism $\omega_i:\Cent(\mathbb{C}[G])\rightarrow\mathbb{C}$. If $u=\sum u(s)s$ is an element of $\Cent(\mathbb{C}[G])$, we have \[\omega_i(u)=\frac{1}{n_i}\tr_{W_i}(\tilde{\rho}_i(u))=\frac{1}{n_i}\sum_{s\in G}u(s)\chi_i(s)\]
\end{prop}
\begin{prop}
    The family $(\omega_i)_{1\leq i\leq h}$ defines an isomorphism of $\Cent(\mathbb{C}[G])$ onto the algebra $\mathbb{C}^h=\mathbb{C}\times\cdots\times\mathbb{C}$.
\end{prop}
\begin{prop}
    Let $x$ be an element of a commutative ring $R$. The following properties are equivalent:
    \begin{enumerate}
        \item $x$ is integral over $\mathbb{Z}$.
        \item The subring $\mathbb{Z}[x]$ of $R$ generated by $x$ is finitely generated as a $\mathbb{Z}$-module.
        \item There exists a finitely generated sub-$\mathbb{Z}$-module of $R$ which contains $\mathbb{Z}[x]$.
    \end{enumerate}
\end{prop}
\begin{corollary}
    If $R$ is a finitely generated $\mathbb{Z}$-module, each element of $R$ is integral over $\mathbb{Z}$.
\end{corollary}
\begin{corollary}
    The elements of $R$ which are integral over $\mathbb{Z}$ form a subring of $R$.
\end{corollary}
\begin{prop}
    Let $\chi$ be the character of a representation $\rho$ of a finite group $G$. Then $\chi(s)$ is an algebraic integer for each $s\in G$.
\end{prop}
\begin{prop}
    Let $u=\sum u(s)s$ be an element of $\Cent(\mathbb{C}[G])$ such that the $u(s)$ are algebraic integers. Then $u$ is integral over $\mathbb{Z}$.
\end{prop}
\begin{corollary}
    Let $\rho$ be an irreducible representation of $G$ of degree $n$ and character $\chi$. If $u=\sum_{i=1}^{i=h}u(s_i)e_i$, then the number $(1/n)\sum_{s\in G}u(s)\chi(s)$ is an algebraic integer.
\end{corollary}
\begin{corollary}
    The degrees of the irreducible representations of $G$ divide the order of $G$.
\end{corollary}
\begin{prop}
    Let $C$ be the center of $G$. The degrees of the irreducible representations of $G$ divide $(G:C)$.
\end{prop}
\section{Fulton and Harris: Representation Theory}
\subsection{Lie Groups and Lie Algebras}
\subsubsection{Lie Groups}
\begin{prop}
    Let $G$ be a Lie group, $H$ a connected manifold, and $\varphi:H\rightarrow G$ a covering space map. Let $e'$ be an element lying over the identity $e$ of $G$. Then there is a unique Lie group structure on $H$ such that $e'$ is the identity and $\varphi$ is a map of Lie groups; and the kernel of $\varphi$ is in the center of $H$.
\end{prop}
\begin{prop}
    Let $H$ be a Lie group, and $\Gamma\subset Z(H)$ a discrete subgroup of its center. Then there is a unique Lie group structure on the quotient group $G=H/\Gamma$ such that the quotient map $H\rightarrow G$ is a Lie group map.
\end{prop}
\subsubsection{Lie Algebras and Lie Groups}
\begin{theorem}(First Principle).
    Let $G$ and $H$ be Lie groups, with $G$ connected. A map $\rho:G\rightarrow H$ is uniquely determined by its differential $d\rho_e:T_eG\rightarrow T_eH$ at the identity.
\end{theorem}
\begin{theorem}(Second Principle).
    Let $G$ and $H$ be Lie groups, with $G$ connected and simply connected. A linear map $T_eG\rightarrow T_eH$ is the differential of a homomorphism $\rho:G\rightarrow H$ if and only if it preserves the bracket operation, i.e., $d\rho_e([X,Y])=[d\rho_e(X),d\rho_e(Y)]$.
\end{theorem}
\begin{defn}(Lie algebra).
    A \textit{Lie algebra} $\mathfrak{g}$ is a vector space together with a skew-symmetric bilinear map $[\;,\;]:\mathfrak{g}\times\mathfrak{g}\rightarrow\mathfrak{g}$ satisfying the Jacobi identity.
\end{defn}
\begin{defn}(Representation of a Lie algebra).
    A \textit{representation} of a Lie algebra $\mathfrak{g}$ on a vector space $V$ is simply a map of Lie algebras $\rho:\mathfrak{g}\rightarrow\GL(V)=\End(V)$, i.e., a linear map that preserves brackets, or an action of $\mathfrak{g}$ on $V$ such that $[X,Y](v)=X(Y(v))-Y(X(v))$.
\end{defn}
\begin{prop}
    The exponential map is the unique map from $\mathfrak{g}$ to $G$ taking 0 to $e$ whose differential at the origin $(\exp_*)_0:T_0\mathfrak{g}=\mathfrak{g}\rightarrow T_eG=\mathfrak{g}$ is the identity, and whose restrictions to the lines through the origin in $\mathfrak{g}$ are one-parameter subgroups of $G$.
\end{prop}
\begin{prop}
    Let $G$ be a Lie group, $\mathfrak{g}$ its Lie algebra, and $\mathfrak{h}\subset\mathfrak{g}$ a Lie subalgebra. Then the subgroup of the group $G$ generated by $\exp(\mathfrak{h})$ is an immersed subgroup $H$ with tangent space $T_eH=\mathfrak{h}$.
\end{prop}
\subsubsection{Initial Classification of Lie Algebras}
\begin{defn}(Nilpotent Lie algebra).
    We say that $\mathfrak{g}$ is \textit{nilpotent} if $\mathscr{D}_k\mathfrak{g}=0$ for some $k$.
\end{defn}
\begin{defn}(Solvable Lie algebra).
    We say that $\mathfrak{g}$ is \textit{solvable} if $\mathscr{D}^k\mathfrak{g}=0$ for some $k$.
\end{defn}
\begin{defn}(Perfect Lie algebra).
    We say that $\mathfrak{g}$ is \textit{perfect} if $\mathscr{D}\mathfrak{g}=\mathfrak{g}$.
\end{defn}
\begin{defn}(Semisimple Lie algebra).
    We say that $\mathfrak{g}$ is \textit{semisimple} if $\mathfrak{g}$ has no nonzero solvable ideals.
\end{defn}
\begin{theorem}(Engel's Theorem).
    Let $\mathfrak{g}\subset\GL(V)$ be any Lie subalgebra such that every $X\in\mathfrak{g}$ is a nilpotent endomorphism of $V$. Then there exists a nonzero vector $v\in V$ such that $X(v)=0$ for all $X\in\mathfrak{g}$.
\end{theorem}
\begin{theorem}(Lie's Theorem).
    Let $\mathfrak{g}\subset\GL(V)$ be a complex solvable Lie algebra. Then there exists a nonzero vector $v\in V$ that is an eigenvector of $X$ for all $X\in\mathfrak{g}$.
\end{theorem}
\begin{lemma}
    Let $\mathfrak{h}$ be an ideal in a Lie algebra $\mathfrak{g}$. Let $V$ be a representation of $\mathfrak{g}$ and $\lambda:\mathfrak{h}\rightarrow\mathbb{C}$ a linear function. Set $W=\{v\in V:X(v)=\lambda(X)\cdot v\;\forall X\in\mathfrak{h}\}$. Then $Y(W)\subset W$ for all $Y\in\mathfrak{g}$.
\end{lemma}
\begin{prop}
    Let $\mathfrak{g}$ be a complex Lie algebra, $\mathfrak{g}_{ss}=\mathfrak{g}/\Rad(\mathfrak{g})$. Every irreducible representation of $\mathfrak{g}$ is of the form $V=V_0\otimes L$, where $V_0$ is an irreducible representation of $\mathfrak{g}_{ss}$ (i.e., a representation of $\mathfrak{g}$ that is trivial on $\Rad(\mathfrak{g})$), and $L$ is a one-dimensional representation.
\end{prop}
\begin{theorem}(Complete Reducibility).
    Let $V$ be a representation of the semisimple Lie algebra $\mathfrak{g}$ and $W\subset V$ a subspace invariant under the action of $\mathfrak{g}$. Then there exists a subspace $W'\subset V$ complementary to $W$ and invariant under $\mathfrak{g}$.
\end{theorem}
\begin{theorem}(Preservation of Jordan Decomposition).
    Let $\mathfrak{g}$ be a semisimple Lie algebra. For any element $X\in\mathfrak{g}$, there exist $X_s$ and $X_n\in\mathfrak{g}$ such that for any representation $\rho:\mathfrak{g}\rightarrow\GL(V)$ we have $\rho(X)_s=\rho(X_s)$ and $\rho(X)_n=\rho(X_n)$.
\end{theorem}
\begin{theorem}
    With five exceptions, every simple complex Lie algebra is isomorphic to either $\mathfrak{sl}_n\mathbb{C}$, $\mathfrak{so}_n\mathbb{C}$ or $\mathfrak{sp}_{2n}\mathbb{C}$ for some $n$.
\end{theorem}

\end{document}